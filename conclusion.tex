\label{sec:conclusion}
This paper presents a novel approach to detect false sharing problems by combining
compiler instrumentation technique and a runtime system. 
Compiler instruments every read and write access of global variables
and heap objects by inserting a function call to invoke the runtime
system. 
By collecting and analyzing information passed through instrumented function calls, the runtime system 
detects false sharing based on cache invalidations and only reports those ones  
causing severe performance degradation.
To differentiate false sharing with true sharing, 
\Predator{} further provides word accesses information for those cache lines involved in false sharing, 
which can help users debug and fix the problem.

\Predator{} further predicts potential false sharing caused by the change of cache line size 
or starting addresses of objects. It helps to prevent the predicament of existing tools:
problems may occur in a real environment rather than test environment due to environmental changes.

Our evaluation shows \Predator{} can effectively detect and predict several 
unknown and existing false sharing problems 
in two popular benchmark suites, \texttt{Phoenix} and \texttt{PARSEC}. 
We also evaluate \Predator{} on $6$ different real applications. 
It successfully detects two known false sharing problems inside 
\texttt{MySQL} and \texttt{boost} libraries.
Fixing these fals sharing problems improves the performance by $6\times$ and
$40\%$ respectively.
\Predator{} is ready to be used in real-world deployment environments. 
