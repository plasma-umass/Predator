\label{sec:conclusion}
This paper presents a novel method to detect false sharing problem by combining
compiler instrumentation and runtime system. 
Compiler instruments every read and write access of global variables
and heap objects by inserting a function call to notify corresponding runtime
system. 
By handling those instrumented function calls, runtime system can collect
and analyze false sharing problems based on cache invalidations: those 
false sharing causing severe performance degrading should have a great amount of 
cache invalidations. To differentiate false sharing with true sharing, 
\defaults{} further provides word accesses information for those cache lines involved in false sharing, 
which can help users locate where the problem is and how to fix false sharing problem.

Manifests of false sharing can be affected by alignments between
objects and cache lines, which can be caused by any change of compiler optimization, 
compiler, memory manager, memory allocation order, cache line size or different target binary.
\Defaults{} can further predict potential false sharing caused by the change of cache line size 
or starting address of objects, thus it helps to prevent the predicament of existing tools:
problems may occur in a real environment rather than test environment due to environmental changes.
 
In the actual evaluation, \defaults{} finds some unknown false sharing problems in two popular benchmark
suites, \texttt{Phoenix} and \texttt{PARSEC}. 
\defaults{} are also evaluated on $6$ different real applications and 
successfully finds existing false sharing problems inside \texttt{MySQL} and \texttt{boost} libraries. 
\defaults{] can be utilized to find some unknown false sharing performance problems in real deployments.
