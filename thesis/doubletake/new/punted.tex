\vspace*{\baselineskip} 
During normal execution of an epoch, \doubletake{} intercepts all calls to \pthreads{} 
library functions that may issue system calls. \doubletake{} actually handles much more function calls
than the listed number of system calls. For example, \texttt{open()}, \texttt{fopen()}, \texttt{fdopen}, \texttt{freopen()}, and \texttt{creat()} can possibly call \texttt{open} system calls. 
Then all of these library functions should be intercepted. 
%We assume that other libraries are built 
%on this library: there is no direct system call from other libraries or applications.

To reduce overhead, we should reduce the amount of epochs as much as possible by 
recording, checkpointing and delaying system calls since starting and ending an epoch 
involves large performance overhead, as discussed in Section~\ref{sec:implementation}, 
Currently, \doubletake{} only optimizes those system calls invoked by evaluated applications.
Current category of system calls can be seen in Table~\ref{tbl:syscalls}. Other system calls
not mentioned here belong to the category of irrevocable system calls. 

\begin{comment}
To support multithreading programs, \DoubleTake{} has to record the order of synchronizations 
and replay them in the same order (Section~\ref{sec:multithreading}) and design memory allocator specially (Section~\ref{sec:mtheap}).

The basic idea of \DoubleTake{} is to greatly {\em reduce the number of checkings}:  
instead of checking a possible out-of-bounds error before every memory access, 
\DoubleTake{} checks all possible out-of-bounds errors of the whole program 
before those irrevocable system calls or in the end of an execution.
By checking buffer overflows accumulatively, \DoubleTake{} amortizes 
checking overhead over long executions and achieves much better performance. 
As showed in Figure~\ref{fig:diagram}, if a program does not have buffer overflows, 
it continues to perform those system calls and 
starts a new epoch after them by snapshotting state of this execution. 
If a program is found to have buffer overflows, \DoubleTake{} re-executes it by rolling back 
to last good state after installing hardware watchpoints on those overflowing addresses.
By handling exceptions, 
\DoubleTake{} can pinpoint precisely those memory accesses causing memory overflows without
actually instrumenting every access. 
\end{comment}


\subsection{Canary Mechanism}
\label{sec:canaries}
Canaries, also called as sentinels, are normally put before or after each actual object.
They are initialized to a special value in the beginning so that modifications of those values
indicates the problems.
Canaries have been used by many previous works to
detect buffer overflows ~\cite{overflow:purify, exterminator, overflow:lbc}.
Similarly, \stopgap{} puts canaries before and after heap objects, which can be used .
to detect buffer underflows and overflows.
The size of canary is a balance between memory overhead and precision.
As discusssed by Hasabnis et al. ~\cite{overflow:lbc}, larger size helps to detect more
buffer overflows, but increasing memory overhead.
\stopgap{} chooses the size of a pointer as the size of each word-based canary:
4 bytes for 32 bit binaries and 8 bytes for 64 bit binaries.
For those objects with size not aligned by words, \stopgap{} filles byte-based canaries before a
word-based canary.

\stopgap{} maintains a global bitmap to mark placements of those canaries:
if a word includes canaries, either a word-based canary or multiple
byte-based canaries, its corresponding bit is set to $1$.
Since \stopgap{} only uses a bit for each word, this helps to reduce memory overhead of the bitmap.
Previous works use a bit for each byte in order to capture one-byte buffer overflow~\cite{overflow:lbc, AddressSanitizer}.
However, it is uncessary to do this since \stopgap{} embeds size information of each heap object
into the header, thus by coming with size information and byte-based canaries \stopgap{} doesn't
loose the precision of detecting one-byte buffer overflows.
Reducing the size of bitmap helps improving the speed of heap integrity checking too.

\subsection{Watch Point Mechanism}
\label{sec:watchpoints}
Watch point mechanism relies on hardware debug registers to monitor memory accesses on
specific addresses. Previous work has used this mechanism for their specific 
targets \cite{fastboundschecking, Kivati}.

Debug registers are universally supported by most if not all existing CPUs, such as
X86 and X86-64 architecture, PowerPC, MIPS, ARM, etc. 
The main target of debug registers is to support software debuggers, e.g. gdb. 
In X86 and X86-64 architectures, there are four debug address registers, one debug 
status register and one debug control register. 
A user can set up debug address registers to hold those addresses that want to monitor 
before executions of a program. 
Then this user can be notified by an exception whenever a memory access matches one of those 
debugging address in debug registers. 

%\DoubleTake{} installs the addresses of canaries with buffer overflows and watches memory
%accesses on them in the re-execution phase by handling those exceptions generated by debug registers. 
%By analyzing the context of signal handler, 
%\DoubleTake{} can precisely locate those instructions causing buffer overflows and report to users.
%More implementation details are discussed in Section~\ref{sec:installwatchpoints}. 

\subsection{Customized Heap Allocator}
\label{sec:allocator}
A general memory allocator invokes a big number of \texttt{mmap} or \texttt{sbrk} system calls,
thus it is hard to repeat memory usages of a program without recording all memory allocations. 
%related system calls and repeating
%them in re-execution phase.
\DoubleTake{} utilizes customized memory allocator for its heap allocations to overcome these shortcomings. The memory allocator of \DoubleTake{} is built on Heap Layers ~\cite{heaplayers}, 
%repeat memory allocations in re-execution phase.

%However, it is normally impossible to do this for multithreading programs.
\DoubleTake{} pre-allocates a fixed size of memory 
from its underlying operating system using \texttt{mmap} system calls and 
satisfies memory allocations from this block of memory.
All heap objects has the size of {\it power of $2$}, known as block size. 
When an object is allocated, \doubletake{} adds a object header to each object, including block size,
actual object size and canary.
When an object is deallocated, it is put into a corresponding free list of the memory allocator, 
holding objects with the same block size. 
\DoubleTake{} never changes size of an object so there is no split and merge operation on heap objects.
If size of an allocation is less than {\it power of 2}, 
\DoubleTake{} allocates an object with the size of next {\it power of 2}.
% but putting canaries
%right after this actual object in order to detect .
%The support for multithreading programs can be seen in Section~\ref{sec:mtheap}.

% why it is deterministic?
Since it is conventient to find out memory usage conditions in the beginning of each epoch 
and all memory allocation are deterministic according to the order of program execution from the same heap without involving other system calls,  
it is relatively easy to reproduce memory usage in production runs.
% Because of this design, it is convenient to take a snapshot in production runs 
%and repeat memory usage in reproduction runs. 

%%%%%%%%%%%%%%%%%%%%%%%%%%%%
%%%%%%%%%%%%%%%%%%%%%%%%%%%%

\subsubsection{Micro Benchmarks}
All benchmarks evaluated in this subsection is taken from NIST SAMATE Reference Dataset Project ~\cite{microbenchmarks}.

{\bf Heap Overflows}: We evaluate \doubletake{} on 26 test cases listed in Table~\ref{table:SAMATESRDOverflow}. \doubletake{} can detect all buffer overflows when they corrupted canaries. \doubletake{} does not have any false alarms. Test cases 1955 and 2063 are not detected by \doubletake{} because they are not continuous buffer overflows, which cannot be detected by any canary-based approaches.

{\bf Memory Leakage}: We evaluate 13 test cases listed in Table~\ref{table:SAMATESRDLeak}. \doubletake{} reports memory leakage correctly for all test cases.  

{\bf Memory Usage-After-Free}: 
{Use After Free}
  
\begin{comment}

\begin{table}[!t]
\centering
\begin{tabular}{l|r}
\hline
{\bf \small Vunerbility Class} & {\bf \small SRD Test Case ID} \\
\hline
strcpy & 015 \\
strcpy & 1843 \\
strcpy & 1845 \\ % Array Index 
memory opration & 1952 \\ % Memory operation
Array Boundary & 1955 \\ % Array assignment
Array Boundary & 1958 \\ % Array assignment
Array Index & 1961 \\ % Array assignment
Array Index & 2062 \\ % Array assignment
Array Index & 2063 \\ % Array assignment
Array Index & 2064 \\ % Array assignment
Array Index & 2065 \\ % Array assignment
Array Index & 2145 \\ % Array assignment
strcpy & 2147 \\ % Array assignment

\hline \\ 
% Good Cases
strncpy & 1844 \\
strcpy & 1846 \\ % Array Index, limit string 
strcpy & 1848 \\
strcpy & 1936 \\
Memory & 1953 \\
Array Boundary & 1956 \\ % Array assignment
Array Boundary & 1959 \\ % Array assignment
Array Index & 1962 \\ % Array assignment
Array Index & 2066 \\ % Array assignment
Array Scope & 2067 \\ % Array assignment
Array Index & 2068 \\ % Array assignment
strcpy & 2134 \\ % Array assignment
strcpy & 2148 \\ % Array assignment
strcpy & 2149 \\ % Array assignment
 
\hline
\end{tabular}
\caption{SAMATE Test Cases. 
\label{table:SAMATESRD}}
\end{table}
\end{comment}


\begin{table}[!t]
\centering
\begin{tabular}{l|l}
\hline
{\bf \small Vulnerable Cases } & {\bf \small Good Cases} \\
\hline
015 1843 1845 1952 1955 & 1844 1846 1848 1936 1953\\
1958 1961 2062 2063 2065 & 1956 1959 1962 2066 2067 \\
2145 2147 & 2068 2134 2148 2149\\  
\hline
\end{tabular}
\caption{SAMATE Test Cases for Heap Overflows. 
\label{table:SAMATESRDOverflow}}
\end{table}

\begin{table}[!t]
\centering
\begin{tabular}{l|l}
\hline
{\bf \small Memory Leakage } & {\bf \small Good Cases} \\
\hline
1527 1583 1758 1925 1926 & 1458 1584 1587\\
1933 2056 99201 99202 99203 &   \\
\hline
\end{tabular}
\caption{SAMATE Test Cases for Memory Leakage. 
\label{table:SAMATESRDLeak}}
\end{table}


%%%%%%%%%%
%%%%%%%%%%

\begin{table}[!t]
\small
\centering
\begin{tabular}{c|ccc}
\textbf{Application} & \textbf{Buffer overflow} & \textbf{Memory leak} & \textbf{Use-after-free} \\
\hline
\texttt{bc}			& & & \\
\texttt{bzip2}		& & & \\
\texttt{gcc}		& & & \\
\texttt{gzip2}		& & & \\
\texttt{libhx}		& & & \\
\texttt{polymorphy}	& & & \\
\texttt{vim}		& & & \\
\end{tabular}



\begin{tabular}{l|l|l}
\hline
{\bf \small Buffer Overflows} & {\bf \small Memory Leakage} & {\bf \small Use-after-free}\\
\hline
bc libhx vim bzip2 gzip
polymorphy & gcc bzip2 vim & gcc bzip2 vim\\
\hline
\end{tabular}
\caption{Evaluated Real Applications. 
\label{table:realapps}}
\end{table}

