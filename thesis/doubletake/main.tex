%DIF PREAMBLE EXTENSION ADDED BY LATEXDIFF
%DIF UNDERLINE PREAMBLE %DIF PREAMBLE
\RequirePackage[normalem]{ulem} %DIF PREAMBLE
\RequirePackage{color}\definecolor{RED}{rgb}{1,0,0}\definecolor{BLUE}{rgb}{0,0,1} %DIF PREAMBLE
\providecommand{\DIFadd}[1]{{\protect\color{blue}\uwave{#1}}} %DIF PREAMBLE
\providecommand{\DIFdel}[1]{{\protect\color{red}\sout{#1}}}                      %DIF PREAMBLE
%DIF SAFE PREAMBLE %DIF PREAMBLE
\providecommand{\DIFaddbegin}{} %DIF PREAMBLE
\providecommand{\DIFaddend}{} %DIF PREAMBLE
\providecommand{\DIFdelbegin}{} %DIF PREAMBLE
\providecommand{\DIFdelend}{} %DIF PREAMBLE
%DIF FLOATSAFE PREAMBLE %DIF PREAMBLE
\providecommand{\DIFaddFL}[1]{\DIFadd{#1}} %DIF PREAMBLE
\providecommand{\DIFdelFL}[1]{\DIFdel{#1}} %DIF PREAMBLE
\providecommand{\DIFaddbeginFL}{} %DIF PREAMBLE
\providecommand{\DIFaddendFL}{} %DIF PREAMBLE
\providecommand{\DIFdelbeginFL}{} %DIF PREAMBLE
\providecommand{\DIFdelendFL}{} %DIF PREAMBLE
%DIF END PREAMBLE EXTENSION ADDED BY LATEXDIFF

%\documentclass{sigplanconf}
%\nocaptionrule

% \documentclass[twocolumn,9pt]{article}
% \documentclass[twocolumn,10pt]{acm_proc_article-sp}

% \documentclass{acm_proc_article-sp}
\documentclass[9pt]{sigplanconf}

\date{} % \vspace*{-0.2in}}

% Make sure to put back 

\newcommand{\punt}[1]{}

\punt{

Notes from Daan Leijen:
}

\usepackage{endnotes,xspace}

\newcommand{\footnotenonumber}[1]{{\def\thempfn{}\footnotetext{\small #1}}}
\usepackage[normalem]{ulem}
\usepackage{graphicx}

\usepackage{mathptmx} % rm & math
\usepackage[scaled=0.90]{helvet} % ss
\usepackage{courier} % tt
% \normalfont
\usepackage[T1]{fontenc}

% \usepackage{lmodern}
% \usepackage{times}
\usepackage{subfigure}
\usepackage{url}
\urlstyle{rm}
\usepackage[
      colorlinks=false,    %no frame around URL
      urlcolor=black,    %no colors
      menucolor=black,    %no colors
      linkcolor=black,    %no colors
      pagecolor=black,    %no colors
]{hyperref}

\usepackage{color}
\usepackage{listings}
\usepackage{amsmath}
\usepackage{amsfonts}
\usepackage{amssymb}
\usepackage{comment} 
\usepackage{setspace}
\singlespacing
%\onehalfspacing
\newtheorem{thm}{Theorem}
\newtheorem{prop}[thm]{Proposition}
\newtheorem{cor}[thm]{Corollary}
\newtheorem{lem}[thm]{Lemma}
\newtheorem{defn}[thm]{Definition}

\newcommand{\cfunction}[1]{{\bf \tt #1}}
\newcommand{\malloc}{\cfunction{malloc}}
\newcommand{\realloc}{\cfunction{realloc}}
\newcommand{\free}{\cfunction{free}}
\newcommand{\madvise}{\cfunction{madvise}}
\newcommand{\brk}{\cfunction{brk}}
\newcommand{\sbrk}{\cfunction{sbrk}}
\newcommand{\mmap}{\cfunction{mmap}}
\newcommand{\munmap}{\cfunction{munmap}}
\newcommand{\mprotect}{\cfunction{mprotect}}
\newcommand{\mlock}{\cfunction{mlock}}

\hyphenation{app-li-ca-tion}
\hyphenation{Die-Hard}
\hyphenation{Ar-chi-pe-la-go}
\hyphenation{buf-fer}
\hyphenation{D-threads}
\hyphenation{Heap-Layers}
\hyphenation{wait-Token}
\hyphenation{mul-ti-threa-ded}
\hyphenation{me-m-ory}

\hyphenation{pthread-create}
\hyphenation{pthread-self}
\hyphenation{pthread-mutex-lock}
\hyphenation{pthread-mutex-unlock}

\newcommand{\dthreads}{{\scshape Dthreads}}
\newcommand{\Dthreads}{{\scshape Dthreads}}
\newcommand{\doubletake}{{\scshape DoubleTake}}
\newcommand{\DoubleTake}{{\scshape DoubleTake}}
\newcommand{\stopgap}{{\scshape DoubleTake}}
\newcommand{\Stopgap}{{\scshape DoubleTake}}
\newcommand{\StopGap}{{\scshape DoubleTake}}
\newcommand{\Sheriff}{{\scshape Sheriff}}
\newcommand{\sheriff}{{\scshape Sheriff}}
\newcommand{\Grace}{{\scshape Grace}}
\newcommand{\grace}{{\scshape Grace}}
\newcommand{\SheriffProtect}{\textsc{Sheriff-Protect}}
\newcommand{\sheriffProtect}{\textsc{Sheriff-Protect}}
\newcommand{\sheriffprotect}{\textsc{Sheriff-Protect}}
\newcommand{\SheriffDetect}{\textsc{Sheriff-Detect}}
\newcommand{\sheriffDetect}{\textsc{Sheriff-Detect}}
\newcommand{\sheriffdetect}{\textsc{Sheriff-Detect}}
\newcommand{\pthreads}{\texttt{pthreads}}

\definecolor{lightgray}{rgb}{.9,.9,.9}
\definecolor{darkgray}{rgb}{.4,.4,.4}
\definecolor{purple}{rgb}{0.65, 0.12, 0.82}

\lstdefinelanguage{c++threads}[]{c++}{
  morekeywords={pthread_create,pthread_join},
  keywordstyle=\color{blue}\bfseries,
  ndkeywords={class, export, boolean, throw, implements, import, this},
  ndkeywordstyle=\color{darkgray}\bfseries,
  identifierstyle=\color{black},
  sensitive=false,
  comment=[l]{//},
  morecomment=[s]{/*}{*/},
  commentstyle=\color{purple}\ttfamily,
  stringstyle=\color{red}\ttfamily,
  morestring=[b]',
  morestring=[b]"
}
\lstset{
   language=c++threads,
   backgroundcolor=\color{lightgray},
   extendedchars=true,
   basicstyle=\footnotesize\ttfamily,
   showstringspaces=false,
   showspaces=false,
   numbers=none,
   numberstyle=\footnotesize,
   numbersep=9pt,
   tabsize=2,
   breaklines=true,
   showtabs=false,
   captionpos=b
}
%\lstset{language=c++threads, basicstyle=\ttfamily\scriptsize,frame=trbl,tabsize=4} % ,numbers=left,numberstyle=\tiny}

\definecolor{Gray}{cmyk}{0,0,0,0.5}

\begin{document}

\CopyrightYear{2014}
\copyrightdata{XXX-X-XXXXX-XXX-X/XX/XX}

\title{{\huge \bf \doubletake{}}: Evidence-Based Dynamic Analysis}
% Efficiently and Precisely Locating Buffer Overflows}

% \authorinfo{\emph{authorship list removed for anonymity}}

%\punt{
\authorinfo{Tongping~Liu \and Charlie~Curtsinger \and Emery~D.~Berger}
{School of Computer Science \\
University of Massachusetts Amherst \\
Amherst, MA 01003 \\
{\{tonyliu,charlie,emery\}@cs.umass.edu}
}

\punt{
\numberofauthors{1}
\author{
\alignauthor Tongping~Liu and Emery~D.~Berger \\
\affaddr{Department of Computer Science} \\
\affaddr{University of Massachusetts, Amherst} \\
\affaddr{Amherst, MA 01003} \\
\email{\{tonyliu,emery\}@cs.umass.edu} \\
}
}

\maketitle

\begin{comment}
\end{comment}

\begin{abstract}
%How is existing work?
%Sharing inside mulithreaading programs is not easy, they can easily cause correctness or performance problem. 
%Inappropriate sharing can dramatically degrade the performance of 
%mulithreading programs and seriously affect the scalability. 
%So detecting false sharing accurately and precisely can be helpful for user to fix corresponding performance problem. 


False sharing is a notorious problem for multithreaded applications
that can drastically degrade both performance and
scalability. Existing false sharing detectors can precisely identify
the sources of false sharing, but are limited to reporting false
sharing actually observed during execution: they do not generalize
across executions. Because false sharing is extremely sensitive to
object layout, these detectors can easily miss false sharing problems
that can arise due to slight differences in memory allocation order or
object placement decisions by the compiler. In addition, they cannot
predict the impact of false sharing on hardware with different cache
line sizes.

%objects and cache lines: any change of compiler optimization, compiler, memory manager, 
%memory allocation order, cache line size or different target binary 
%may change alignments, and thus affect occurrences of false sharing, 
%which leaves many of them undetected by existing tools.

This paper presents \Predator{}, a predictive software-based false
sharing detector. \Predator{} generalizes from a single execution to
precisely predict false sharing that is latent in the current
execution. \predator{} tracks accesses within a range that could lead
to false sharing given different object placement. It also tracks
accesses within
\emph{virtual cache lines}, contiguous memory ranges that span actual
hardware cache lines, to predict sharing on hardware platforms with
larger cache line sizes. For each, it reports the exact program
location of predicted false sharing problems, ranked by their
projected impact on performance. We evaluate \Predator{} across a
range of benchmarks and actual applications: \Predator{} identifies
problems undetectable with previous tools, including two
previously-unknown false sharing problems, with no false
positives. \Predator{} located false sharing problems in MySQL and the
Boost library that had eluded detection for years.

%\Predator{} identified two unknown false sharing problems 
%Besides, \Predator{} have successfully detected false sharing of real applications,
%including \texttt{mysql} server application and \texttt{boost} library. Fixing these
%false sharing problems improves performance by $6\times$ and $40\%$ correspondingly.



%False sharing is a notorious performance issue for different software stacks, 
%which can dramatically degrade the performance and seriously affect the scalability of 
%systems.

%Many reserach efforts have been made to detect false sharing. 
%Unfortunately, previous approaches to detect false sharing
%either introduce significant performance overhead, or fail
%to report false sharing accurately and precisely, or have different limitations of usage. 
%\sheriff{}, the prior state-of-the-art tool, 
%can only detect write-write false sharing in applications using \pthreads{} library.
%This paper presents a novel approach, \Predator{}, to combine compiler instrumentation
%and runtime system to detect false sharing. 
%the compiler instruments every memory access and 
%the runtime system collects and analyzes memory accesses to detect false sharing problems.
%Since it does not rely on any hardware, OS or threading library, this approach can be
%applied to the entire software stack without any limitation. 
%\Predator{} can detect false sharing accurately and precisely: it reports no 
%false positives and pinpoints exact objects with false sharing problems.
%Also, unlike previous work, this method can be extended to
%identify false sharing problems across the entire software stack, including 
%hypervisors, operating systems, libraries and applications. 
%Experimental results on two popular benchmark suites 
%show that \Predator{} not only detected all known false sharing problems but also revealed 
%two unknown false sharing problems.
%Besides, \Predator{} have successfully detected false sharing of real applications,
%including \texttt{mysql} server application and \texttt{boost} library. Fixing these
%false sharing problems improves performance by $6\times$ and $40\%$ correspondingly.

%Moreover, existing tools can only detect those manifested false sharing problems.
%However, occurrences of false sharing can be affected by alignments between
%objects and cache lines: any change of compiler optimization, compiler, memory manager, 
%memory allocation order, cache line size or different target binary 
%may change alignments, and thus affect occurrences of false sharing, 
%which leaves many of them undetected by existing tools.
%\Predator{} is the first tool which can accurately predict possible false sharing 
%without the need of occurrences. 
%It can report all false sharing problems with only one execution and with reasonable overhead, 
%around $6.7\times$ performance overhead on average.

%What is novel in our work?
%How is the performance overhead?


\end{abstract}

%  Language-based approaches require programmers to write their code in specialized languages. 


\punt{
\category{D.1.3}{Programming Techniques}{Concurrent Programming--Parallel Programming}
\category{D.2.5}{Software Engineering}{Testing and Debugging--Debugging Aids}

\terms
Design, Reliability, Performance

\keywords
Buffer Overflow, Detection 
}


%%%%%%%%%%%%%%%%%%%%%%%%%%%%%%%%%%%%%%%%%%%%%%%%%%%%%%%%%%%%%%%%%%%%%%%%%%%%%%%%%%%%%%%%%%%%%
%%%%%%%%%%%%%%%%%%%%%%%%%%%%%%%%%%%%%%%%%%%%%%%%%%%%%%%%%%%%%%%%%%%%%%%%%%%%%%%%%%%%%%%%%%%%%

\section{Introduction}
\label{sec:introduction}

The advent of multicore architectures has made multithreaded
programming increasingly necessary, but writing multithreaded programs
remains painful. It is notoriously far more challenging to write
concurrent programs than sequential ones because of the wide range of
errors it can cause, including deadlocks and race
conditions~\cite{havender,76897,130623}. Because thread interleavings
are non-deterministic, different runs of the same multithreaded
program can unexpectedly produce different results. These
``Heisenbugs'' greatly complicate debugging, and eliminating them
requires extensive testing to account for possible thread
interleavings~\cite{DBLP:conf/icse/BallBHMQ09,DBLP:conf/asplos/BurckhardtKMN10}.

% Lots of recent work on bug finding. Getting better, but still difficult.

Instead of testing, one promising alternative approach is to attack
the problem of concurrency bugs by eliminating its source:
non-determinism. A fully \emph{deterministic multithreaded system}
would prevent Heisenbugs by ensuring that executions of the same
program with the same inputs always yield the same results, even in
the face of race conditions in the code. Such a system would not only
dramatically simplify debugging of concurrent
programs~\cite{Carver:1991:RTC:624586.625040} and reduce their
attendant testing overhead, but would also enable a number of other
applications. For example, a deterministic multithreaded system would
greatly simplify record and replay for multithreaded
programs~\cite{Choi:1998:DRJ:281035.281041,LeBlanc:1987:DPP:32387.32396}
and the execution of multiple replicas of multithreaded applications
for fault
tolerance~\cite{deterministic-process-groups,1134000,224058,replicant-hotos}.

Several recent software-only proposals aim at providing
deterministic multithreading, but these all suffer from a variety of
disadvantages. Language-based approaches are effective at removing
determinism but require programmers to write code in specialized
languages, which can be
impractical~\cite{Bocchino:2009:TES:1640089.1640097,Burckhardt:2010:CPR:1869459.1869515,Simpson:1999:SEE:330346.330357}. Recent
deterministic systems that target legacy programming languages
(especially C/C++) are either incomplete or impractical. Kendo ensures
determinism of synchronization operations with low overhead, but does
not guarantee determinism in the presence of data
races~\cite{1508256}. Grace prevents all concurrency errors but is
limited to fork-join programs, and although it is efficient, it can require
code modifications to avoid large runtime
overheads~\cite{grace}. CoreDet, a compiler and runtime system,
enforces deterministic execution for arbitrary multithreaded C/C++
programs~\cite{Bergan:2010:CCR:1736020.1736029}. However, it exhibits
prohibitively high overhead (running up to $8\times$ slower
than \pthreads{}; see Section~\ref{sec:evaluation}) and generates
thread interleavings at arbitrary points
in the code, complicating program debugging and testing.

\hspace{1em} \\
\noindent
\textbf{Contributions:}
This paper presents \textbf{\dthreads{}}, an efficient deterministic runtime
system for multithreaded C/C++ applications. \dthreads{} guarantees
deterministic execution of multithreaded programs even in the presence
of data races (notwithstanding external sources of non-determinism
like I/O): given the same sequence of inputs, a program
using \dthreads{} always produces the same output. \dthreads{}'
deterministic commit protocol not only eliminates data races but also
prevents lock-based deadlocks.

\dthreads{} is easy to deploy: it works as a direct replacement for
the \pthreads{} library, requiring no code modifications or
recompilation. \dthreads{} is also efficient. Its software
architecture avoids the need for expensive write buffers, and 
its commit protocol eliminates cache-line based false sharing, a
notorious performance problem for multithreaded programs. These two
features enable \dthreads{} to nearly match or even exceed the
performance of \pthreads{} for the majority of the benchmarks examined
here. \dthreads{}
thus is a significant improvement over the state of the art in
deployability and performance, and provides evidence that fully
deterministic multithreaded programming may be practical.

% XXX borrowed from Grace, XXX borrowed from Treadmarks, etc.


%Deployable: directly replaces
%the \pthreads{} library, requiring no code modifications.

% Summary of results. Improvements over state-of-the-art (CoreDet).

The remainder of this paper is organized as
follows. Section~\ref{sec:dthreads-architecture} describes
the \dthreads{} architecture and algorithms in depth, and
Section~\ref{sec:discussion} discusses key
limitations. Section~\ref{sec:evaluation} evaluates \dthreads{}
experimentally, comparing its performance and scalability
to \pthreads{} and CoreDet. Section~\ref{sec:related-work} provides an
overview of related work, Section~\ref{sec:future-work} describes
future directions, and Section~\ref{sec:conclusion} concludes.


\section{Overview}
\label{sec:overview}

\begin{figure}[!t]
\begin{center}
\includegraphics[width=3.3in]{figure/overview}
\end{center}
\caption{
Overview of \doubletake{}: execution is divided into epochs at the boundary of irrevocable system calls. 
\label{fig:overview}}
\end{figure}

\doubletake{} is a high performance dynamic analysis framework for a class of errors that share a \emph{monotonicity} property: evidence of the error is persistent and can be gathered after-the-fact. As Figure~\ref{fig:overview} depicts, program execution is divided into epochs, during which execution proceeds at full speed. At the end of each epoch, marked by irrevocable system calls, \doubletake{} checks program state for evidence of memory errors. If an error is found, the epoch is re-executed with additional instrumentation to pinpoint the exact cause of the error. To demonstrate \doubletake{}'s effectiveness, we have implemented detection tools for heap buffer overflows, use-after-free errors, and memory leaks, which we describe in detail in Section~\ref{sec:applications}.  All detection tools share the following core infrastructure that \doubletake{} provides.

\subsection{Efficient Recording}

At the beginning of every epoch, \doubletake{} saves a snapshot of program registers, and all writable memory. The epoch ends when the program attempts to issue an irrevocable system call, but most system calls do not end the current epoch. \doubletake{} also records the order of thread synchronization operations to support re-execution of parallel programs. \doubletake{} records minimal system state at the beginning of each epoch (like file offsets), which allows system calls that modify this state to be undone if re-execution is required. As a result, most programs require very few epochs and program state checks. We describe the details of each application's state checks in Section~\ref{sec:applications}.

\subsection{Precise Replay}

When program state checks detect an error, \doubletake{} replays the previous epoch to pinpoint the error's root cause. \doubletake{} ensures that all program-visible state, system call results, memory allocations, and the order of all thread synchronization operations are identical to the original run. During replay, \doubletake{} returns saved return values for most system calls, with special handling for some cases. Section~\ref{sec:implementation/normalexecution} describes \doubletake{}'s recording and re-execution of system calls and synchronizations.

\subsection{Custom Heap Allocator}

\doubletake{} replaces the default heap allocator with a new heap built using Heap Layers~\cite{heaplayers}. Detection tools can interpose on heap operations to alter memory allocation requests or defer reuse of freed memory, and can access the heap's map of allocated memory. Memory leak detection uses this map to identify unreachable memory. Buffer overflow and dangling pointer (use-after-free) detection both use heap canaries to detect errors. \doubletake{}'s heap includes a bitmap to track the locations of heap canaries, and automatically checks the state of canaries at the end of each epoch. Section~\ref{sec:heapallocator} presents further details.

\subsection{Watchpoints}

\doubletake{} lets detection tools set hardware watchpoints during re-execution. A small number of watchpoints are available on modern architectures (four on x86). Each watchpoint can be configured to pause program execution when a specific byte or word of memory is accessed. Watchpoints are primarily used by debuggers, but previous approaches have used watchpoints for error detection as well~\cite{Kivati,fastboundschecking}. \doubletake{}'s watchpoints are particularly useful in combination with heap canaries. During re-execution, our buffer overflow and use-after-free detectors place a watchpoint at the location of the overwritten canary to trap the instruction(s) responsible for the error.


\section{Applications}
\label{sec:applications}

\doubletake{} implements three important applications based on its lightweight 
dynamic analysis framework. 
Those applications are implemented in a best-effort way to show the efficiency of our framework.
They are not targeted for a complete and noval solution for these problems, and most of
mechanisms are borrowed from previous work. 

For these applications, \doubletake{} runs a program at full speed during an epoch and 
finds evidence of possible memory problems in the end of each epoch. 
If \doubletake{} detects memory problems, it rolls back the program and re-execute it 
in order to collect detailed information problems and report to users. 
Different applications has different implementations, which are discussed in more detailed 
in the following. 

\subsection{Detection of Heap Buffer Overflows}
\label{sec:overflow}
Buffer overflows can greatly impact programs' robustness and security. When a program has 
buffer overflows, a program can exit unexpectedly with segmentation fault errors or other errors. 
Even worse, overflows can be maliciously exploited, loosing security and privacy guarantees. 
Buffer overflows are still one of dominant errors even after decades of research in this field.
In 2012, four types of the CWE/SANS "Top 25 Most Dangerous Software Errors" and 
over one-third valnuerabilites of 1818 high serverity problems reported from 
NIST are related to buffer overflows~\cite{overflows1, overflows2, overflow:lbc}. 
Among them, a big signifiant of problems are heap buffer overflows. 

\subsubsection{Detection}
To detect heap buffer overflows, \DoubleTake{} puts canaries before and after actual heap 
objects in normal executions ~\cite{overflow:lbc, AddressSanitizer}. 
At the end of each epoch, \doubletake{} verifies integrities of canaries by traversing 
the bitmap, which marks the placement of canaries and discussed in Section~\ref{sec:canaries}.  
Corrupted canaries indicates heap buffer overflows. 
To detecting those one-byte overflows, \doubletake{} embeds actual size information of objects 
into headers of objects and places byte-based canaries for non-aligned objects.

\subsubsection{Reporting}
\label{sec:overflowreport}
When a program is found to have heap overflows, \doubletake{} rolls back the execution 
and re-executes this program.
To precisely identifying the instruction reponsible for buffer overflows, 
\doubletake{} installs watch points at the address of corrupted canary before re-execution, 
which is discussed in Section~\ref{sec:watchpoints}.
When the program is re-executed, any instruction that writes to this address 
will trigger a debug trap (resulting in a SIGTRAP signal). 
\doubletake{} then reports callsite information of trapped instructions. 
This can be used to pinpoint the
exact source location where the buffer overflow occurred.
  
\subsection{Detection of Memory Leakage}
\label{sec:leak}
Memory usage inefficiency can greatly reduce programs performance and availability. 
A system with memory leaky programs can greatly degrade reponsiveness 
if some programs has to be paged out, caused by excessive memory consumption by leaky programs. 
Memory leaks grows memory consumption over time, which can make a program runing slower 
and slower, or even unreponsively in the end. 
Memory leakage is still one of common classes of reported bugs.

\subsubsection{Detection}
\doubletake{} detects memory leakage using the same marking mechanism as 
conservative garbage collection ~\cite{Wilson:1992:UGC:645648.664824}.
Starting from roots (globals, stack, and registers), \doubletake{}
computes all reachable heap objects. Any unreachable object that
has not been freed must be leaked.

More specifically, \doubletake{} checks all values of globals, stack and registers. 
If a value falls into the range of the allocated heap, 
then we adds it to a {\it root list}.
Then \doubletake{} utilizes a BFS search algorithm to verify each value of the {\it root list}. 
For each value, \doubletake{} verifies whether this value is a valid address inside a heap object.
If this value is inside an actual heap object and this object is not checked, 
\doubletake{} checks all values inside this heap objects and puts those possible values 
in the range of allocated heap objects into {\it root list} too. 
After a heap object is verified, it is marked as a reachable object. 
To mark those reachable objects, \doubletake{} utilizes the least significant bit
of block size in headers of each objects. 
Since the size of every blocks in \doubletake{} are always \texttt{power of 2},
there is no usage on least significant bit.

For most of values appearing in the {\it root list}, they are possibly starting addresses of 
heap objects. \doubletake{} also leverages the same bitmap of canaries to locate the 
starting address of this object when a value is not the starting address of an object. 

In the end, \doubletake{} traverses the whole heap to find those leaked heap objects.
Because its customized memory allocator has embedded the actual size information into object headers 
and those actual size are set to 0 when an object is freed, 
\doubletake{} identify a leaked heap object if this object is not freed and it is not reachable. 
Also, \doubletake{} clears the least significant bit of all objects 
in this check procedure for the future scanning. 
%\doubletake{} can also identify
%reachable memory that has been freed, which could potentially
%lead a use after free violation.

\subsubsection{Reporting}
\label{sec:leakreport}
\doubletake{} uses re-execution to find allocation sites for leaked heap objects. 
Re-execution proceeds as normal, with an added check in each \texttt{malloc()}.
When a memory allocation matches the actual size and address of any leaked heap object, 
\doubletake{} reports its call stack, obtaining by using \texttt{backtrace} 
functions of \texttt{glibc} library. 

\subsection{Detection of Use-after-free Errors}
\label{sec:danglingpointer}
Memory use-after-free errors occur when an application continue to use pointers, 
where those objects pointing by them have been deallocated.  
Use-after-free errors can induce unexpected behavior of programs execution 
if an de-allocated object is allocated for other purposes, 
violating correctness and security guarantee of programs. 

\subsubsection{Detection}
To detect memory use-after-free errors, 
\doubletake{} firstly delays memory reusage of all freed objects 
by putting them into a quarantine list, 
same as AddressSanitizer does ~\cite{AddressSanitizer}. 
Those objects in the quarantine list are actually returned back
to the program heap when the total size of freed objects in the quarantine list 
are larger then a pre-defined threshold or the quarantine list is full.
In order to find evidence of memory use-after-free problems, 
\doubletake{} fills the first 128 bytes of an object, which can be adjustable, 
with canaries. 

Before an object is returned back to the program heap,
all canaries inside an object are checked. 
In the end of an epoch, all heap objects inside the quarantine list are also checked. 
Same as detection of buffer overflows, 
a corrupted canary indicates a use-after-free memory error and must be reported to
user. 

\subsubsection{Reporting}
When a program has use-after-free memory errors, 
\doubletake{} re-executes this program to find out the allocation and deallocation site
of corresponding objects and those instructions actually writing them.
It is possible that multiple instructions can access an object after deallocation.  
To obtain the allocation and deallocation callsite information, 
\doubletake{} leverages the same mechansim used in
detection of memory leakage, which is discussed in Section~\ref{sec:leakreport}.
To find out those instructions writting a freed object, 
\doubletake{} installs hardware watch points on those violating addresses, which
shares the same mechanism as overflow detection (Section~\ref{sec:overflowreport}).


\section{Implementation}
\label{sec:implementation}

\doubletake{} is implemented as a library and can be linked directly to programs, or can be injected into unmodified binaries by setting the \texttt{LD\_PRELOAD} environment variable on Linux.

At startup, \doubletake{} begins a new epoch. The epoch continues until the program issues an \emph{irrevocable} system call (see Section~\ref{sec:syscalls} for details). Before this call is issued, \doubletake{} scans program state for evidence of errors. The details of this scan are presented in Section~\ref{sec:applications}.

If no errors are found \doubletake{} ends the epoch, issues the irrevocable system call, and begins a new epoch. If any detection tools have found evidence of an error, \doubletake{} enters re-execution mode. The remainder of this section describes the implementation of \doubletake{}'s core functionality.

\subsection{Epoch Start}
\label{sec:implementation/start}

At the beginning of each epoch, \doubletake{} takes a snapshot of program state. \doubletake{} saves all writable memory (stack, heap, and globals) from the main program and any linked libraries, and saves register state of each thread with the \texttt{getcontext} function. Read-only memory does not need to be saved. To identify all writable mapped memory, \doubletake{} reads the Linux \texttt{/proc/self/map} file. \doubletake{} also saves file positions of all open files. This lets programs issue \texttt{read} and \texttt{write} system calls without ending the current epoch. \doubletake{} uses the saved memory state and file offset to ``undo'' these calls if the epoch needs to be re-executed when an error is found.

%%%%%%%%%%%%%%%%%%%%%%%%%%%

\subsection{Normal Execution}
\label{sec:implementation/normalexecution}

Once a snapshot has been saved, \doubletake{} lets the program execute normally. Most program operations proceed normally, but \doubletake{} interposes on heap allocations and system calls in order to set tripwires and support re-execution.

%%%%%%%%%%%%%%

\subsubsection*{System Calls}
\label{sec:syscalls}
\begin{table}[t]
	\centering
	\small
	\renewcommand{\arraystretch}{1.5}
	\begin{tabular}{l|p{6cm}}
		\textbf{Category} & \textbf{Functions} \\
		\hline
		
		Repeatable		& \texttt{getpid}, \texttt{sleep}, \texttt{pause}\\
		
		Recordable		& \texttt{mmap}, \texttt{gettimeofday}, \texttt{time}, 
						  \texttt{clone} , \texttt{open}\\
		
		Revocable		& \texttt{write}, \texttt{read} \\
		
		Deferrable		& \texttt{close}, \texttt{munmap} \\
		
		Irrevocable		& \texttt{fork}, \texttt{exec}, \texttt{exit}, \texttt{lseek}, \texttt{pipe}, \texttt{flock}, \texttt{socket related system calls}\\
	\end{tabular}
	\caption{System calls handled by \doubletake{}. All unlisted system calls are conservatively treated as irrevocable, and will end the current epoch. Section~\ref{sec:syscalls} describes how \doubletake{} handles calls in each category.\label{table:syscalls}}
\end{table}

\doubletake{} ends each epoch when the program attempts to issue an irrevocable system call. However, most system calls can safely be re-executed or undone prior to re-execution. 

\doubletake{} breaks system calls into five categories, shown in Table~\ref{table:syscalls}. System calls could be intercepted using \texttt{ptrace}, but this would add unacceptable overhead during normal execution. Instead, \doubletake{} interposes on all library functions that may issue system calls.

%%%%%%%

\emph{Repeatable system calls} do not modify system state, and return the same result during normal execution and re-execution. No special handling is required for these calls.

\begin{figure*}[ht!]
	\begin{center}
		\includegraphics[width=6.5in]{figure/perf}
	\end{center}
	\caption{Runtime overhead of \doubletake{} (OD = Buffer Overflow Detection, LD = Leak Detection, \doubletake{} = all three detections enabled) and AddressSanitizer, normalized to each benchmark's original execution time. 
%Overhead for Valgrind is reported in Table~\ref{table:valgrind} because the results do not fit on this graph.
\label{fig:perf}}
\end{figure*}

%%%%%%%

\emph{Recordable system calls} may return different results if they are re-executed. \doubletake{} records the result of these system calls during normal execution, and returns the saved result during re-execution. Some recordable system calls, such as \texttt{mmap}, change the state of underlying operating system. Memory mapped with a call to \texttt{mmap} is left mapped for the entire epoch's re-execution; this is safe because the program cannot access this memory until the point at which the \texttt{mmap} call is replayed.

\emph{Revocable system calls} modify system state, but \doubletake{} can save the original state beforehand and restore it prior to re-execution. Most file I/O fall into this category.

For example, \texttt{write} modifies file contents, \doubletake{} can write the same content during re-execution. \texttt{write} also changes the current file position, which \doubletake{} restores to the saved file position using \texttt{lseek} prior to re-execution. \doubletake{} saves all file descriptors of opened files in a hash table at the beginning of each epoch. In addition, \doubletake{} must save stream contents returned by \texttt{fread}. Calls to \texttt{read} and \texttt{write}  on normal files, which can be identified by check the hash map, don't need to be handled. But those calls on socket files are treated as irrevocable system calls.
	
\emph{Deferrable system calls} will irrevocably change program state, but can safely be delayed until the end of the current epoch. \doubletake{} delays all calls to \texttt{munmap} and \texttt{close}, and executes these system calls before exiting or starting a new epoch.
	
\emph{Irrevocable system calls} change internally-visible program state, and cannot be undone. \doubletake{} must end the current epoch before these system calls are allowed to proceed. Note that for \doubletake{}, the meaning of ``irrevocable'' is different from that used in transactional memory systems~\cite{Irrevocabletrans}. Unlike in transactions, we expect re-execution to be identical to the epoch's original execution. It is safe for system calls to affect externally-visible state as long as the effect on internal state can be hidden or undone.

Note that in the presence of multiple threads, an error may fail to appear in the re-execution because of data races (synchronization ordering is already tracked and replayed). \doubletake{} then re-executes the code in an attempt to reveal the error. If it fails to reveal the error on replay, \doubletake{} has effectively tolerated the error and continues execution.

%%%%%%%%%%%%%%

\subsubsection*{Multithreaded Support}
We have implemented support for multiple threads, but the recording and re-execution of thread synchronizations is not yet stable. \doubletake{} records the sequence of system calls and results separately for each thread.

Every mutex records the order of threads that acquire it, and condition variables record the order of thread wakeups. \doubletake{} does not enforce a total global order on lock acquisitions. Operations within a single thread are totally-ordered, and \doubletake{} enforces local order at each synchronization point. In the absence of data races, this is sufficient to ensure deterministic re-execution.

Calls to \texttt{pthread\_create} are recorded with the same mechanism as recordable system calls. When a new thread starts, \doubletake{} takes a snapshot of the thread's stack and registers to enable re-execution from the beginning of the thread's execution. As with synchronization operations, \doubletake{} logs thread creation order and enforces this order during re-execution. Calls to \texttt{pthread\_exit} are deferred until the end of the epoch. Because \texttt{pthread\_exit} is deferred, \texttt{pthread\_join} is effectively deferred as well.

%%%%%%%%%%%%%%

\subsubsection*{Heap Allocator}
\label{sec:heapallocator}

Heap allocators typically issue a large number of \texttt{mmap} or \texttt{sbrk} system calls, which would complicate \doubletake{}'s logging re-execution. \doubletake{} replaces the default heap with a fixed-size BiBOP-style allocator with per-thread subheaps and power-of-two size classes, built using Heap Layers~\cite{heaplayers}. \doubletake{}'s heap is completely deterministic, so no logging is required to ensure that allocations do not change during re-execution.

When an object is freed, the allocator checks which subheap it is allocated from. If the object comes from the freeing thread's subheap, the \texttt{free} call proceeds uninterrupted. If the object was originally allocated by a different thread, the \texttt{free} is deferred. When the epoch ends, each object whose \texttt{free} was deferred is returned to its source thread's freelist.

During replay, \doubletake{}'s heap allocator checks to see if the object being allocated or freed contains the address where an error was detected. If so, \doubletake{} calls the \texttt{backtrace()} function to obtain a call stack for the allocation and deallocation sites.

\doubletake{} lets error detection tools traverse the set of all allocated objects during error checking. Objects are marked as allocated in object headers, including a size of the \emph{requested size}, which may be less than the power-of-two size class for object. All three detection tools use this size during scanning.

\doubletake{} also maintains a bitmap to record the locations of heap canaries. The bitmap records every word of heap memory that contains a canary. \doubletake{} notifies the detection tool when any of the bytes do not contain canaries. Buffer overflow detection places canaries only outside the requested object size. Re-execution is only started if the detection tool finds that canaries between allocated objects have been overwritten.

%%%%%%%%%%%%%%

\subsubsection*{Epoch End}

The epoch ends when any thread issues an irrevocable system call. All other threads are notified with a signal. Once all threads have stopped, \doubletake{} checks the program state for errors. The application-specific error checks are described in Section~\ref{sec:applications}. If an error is found, \doubletake{} immediately switches to re-execution mode. If not, the runtime issues any deferred system calls and clears the logs for all recorded system calls.

%%%%%%%%%%%%%%%%%%%%%%%%%%%

\subsection{Re-Execution}
\label{sec:implementation/re-execution}

Before re-executing the current epoch, \doubletake{} must roll back program state. Restoring saved memory will overwrite the current stack, so \doubletake{} switches to a temporary stack during rollback. The saved state of all writable memory is copied back, and any revocable system calls are undone (see Section~\ref{sec:implementation/normalexecution} for details). Before restoring register state, \doubletake{} must allow detection tools to place watchpoints.

\subsubsection*{Watchpoints}
Debug registers are not accessible in user-mode, so \doubletake{} must use \texttt{ptrace} to set watchpoints. \doubletake{} forks a child process and attaches to it using \texttt{ptrace} to load watched addresses into the debug registers and enable the watchpoints.

Once watchpoints have been placed, \doubletake{} uses the \texttt{setcontext} call to restore register state and begin re-execution. During re-execution, \doubletake{} replays the saved results of system calls from the log collected during normal execution. All deferred system calls are converted to no-ops while the program is re-executing.

\subsection*{Synchronization Replay}
\doubletake{} enforces the recorded order of synchronization operations during re-execution. A thread can only acquire a mutex if it is the next thread in the acquisition log, regardless of whether the mutex is currently locked. \doubletake{} uses semaphores to wake threads from condition variables in the recorded order. When a condition variable is signaled, the signaling thread notifies next waking thread that it can resume. If this thread has not yet arrived at the condition variable, it will wake immediately after it arrives.


%\section{\doubletake{} for Multithreading Programs}
%\label{sec:multithreading}

This section describes the mulithreading support of \doubletake{}.
A thread is a basic execution unit from the point of view of underlying operating system. 
The order of an execution, greatly affecting memory usage, 
is highly depending on timing, synchronization order and internal scheduling algorithm.   
Thus, it is much more difficult to achieve the target of repeatable memory 
usage for multithreading programs, which is crucial to 
precisely detect buffer overflows or other memory errors.
This section first discusses how to handle epochs in multithreading programs.
After this, it describes the design of heap allocator, suitable for repeatable memory usage.
It then discusses how to handle thread creation and exits specially. 
In the end, it describes how to guarantee deterministic synchronization in the re-execution phase,
which is also crucial for repeatable memory usage.


\subsection{Overview}
\label{sec:mtoverview}

As described in Section~\ref{sec:overview}, \doubletake{} uses irrevocable system calls as 
boundaries for epochs for multithreading programs. 
To simplify description in the following sections, a thread encountering an irrevocalbe system call is 
called as the ``Triggering-Thread''. 

When encountering an irrevocable system call, this Triggering-Thread 
has to stop all existing threads so that all other threads are in a quiecent state, which
has been described in Section~\ref{sec:stopepoch}.
Then it performs memory checkings on the heap as described in Section~\ref{sec:epochend}. 
If there is no buffer overflow, it can perform this irrevocable system call and 
start a new epoch after this system call. 
Before waking up other threads, the Triggering-Thread takes a snapshot for the shared memory at first,
including the heap and globals. 
After a thread is waken up, it only needs to take a snapshot on its own state, 
including its stack and its hardware registers. 

If there are buffer overflows, the Triggering-Thread sets up the shared memory 
for all threads at first.
It recovers the heap and globals by copying from the saved snapshot. 
Then it can wake up other threads. 
However, if the Triggering-Thread is spawed newly in current epoch, 
it has to wait for its parent to start its execution. 

\subsubsection{Epoch}
\label{sec:stopepoch}

It is the duty of a Triggering-Thread to close an epoch.
Whenever this thread meets an irrevocable system call, it has to stop other threads.
\doubletake{} utilizes the ``signal'' mechanism to stop other threads asynchronously.
It signals other threads using SIGUSR2 signal when a thread is in a safe state. 
A thread is considered to be in a unsafe state before this thread finishs a snapshot for itself,
discussed in Section~\ref{sec:threadcreation}.
After sending out all signals, this thread is waiting on a internal conditional variable.
The Triggering-Thread only starts checking buffer overflows after all threads are in quiescence.

However, this SIGUSR2 signal can also be used by user programs. 
In order to differentiate this, \doubletake{} specifically marks on 
a shared flag before signalling so that signal handler can check this in the beginning. 
If this flag is marked, this signal is issued by \doubletake{}. Otherwise, it is issued by
a user program and we can call user registered program instead. 

When other threads receive the signal from the Triggering-Thread, 
they are waiting on an internal conditional variable for instructions from the Triggering-Thread:
it can move forward to next epoch or rollback.
It is worthy noting that inside signal handler we have to utilize a different lock that has not been
used by other places. Otherwise, it is easy to cause deadlock. 

\subsubsection{Customized Heap Allocator}
\label{sec:mtheap}
In order to achive the target of repeatable memory usage, the heap allocator must be designed 
carefully. \doubletake{} first borrows a ``per-thread-heap'' idea from Hoard~\cite{Hoard}. 
\doubletake{} keeps a 1-to-1 mapping between threads and sub-heaps of customized memory allocator. 
The total number of threads and sub-heaps are pre-defined. 
A thread can only allocate memory from its own sub-heap, 
where those sub-heaps can get the memory from a pre-allocated heap 
by allocating a huge block of memory each time. 
After a sub-heap gets a block of memory, its corresponding thread always owns all objects
of this block, called as ``owner'' of this block.
This surely can cause memory blowup problem resolved by Hoard. However, it is 
not the focus of \doubletake{}.   

The ``per-thread-heap'' idea is not enough to guarantee the repeatable memory usage. 
\doubletake{} imposes several additional rules besides this.
Firstly, when \doubletake{} acquires a block of memory from the global pre-allocated heap, it must 
acquire a lock at first, which is guaranteed to be deterministic according to mechanisms 
discussed in Section~\ref{sec:sync}.
This guarantee that every new blocks of each sub-heap is repeatable for re-execution. 
Secondly, when there is a memory deallocation, this freed object can only be returned back  
to its original owner in a safe state. 
If this memory deallocation is issued by the same thread as the owner, then this freed object
can be putted into the owner's free list and be utilized immediately. 
If this memory deallocation is issued by a different thread with the owner, 
which indicates a cross-thread communication,  
then this memory
deallocation are cached into a global list, which only issued in the end of this epoch after 
all threads has been stopped. 
By doing this, we can guarantee all memory usage inside an epoch is repeatable in re-execution phase.
 
\subsection{Thread Creation and Exit}
\label{sec:threadcreation}
Tracking creations and exits of threads is very important because of the following reasons.
First, \doubletake{} has to take snapshots for different threads in the beginning. 
Second, terminination of a thread invokes \texttt{munmap} system calls directly by \pthreads{}, which
can not be intercepted.  
Third, thread creation is considered as a synchronization and has to be recorded. 
Thus, \doubletake{} intercepts \texttt{pthread\_create} calls and changes its start routine 
to a customized function. 
In this customized function, \doubletake{} can record thread creation, take a snapshot and delay thread exit to the end of current epoch. 

\subsubsection{Normal Execution}

\doubletake{} makes \texttt{pthread\_create} to call its customized function as the start routine. 
In this start routine, \doubletake{} first puts this new thread into a global map, which maintains
status of all threads. 
Then it takes a snapshot for this new thread, including stack and hardware registers. 
After this, \doubletake{} can invoke the original routine to actually perform user-defined 
thread function. 

After this user-defined thread function finishes, the control flow returns back to \doubletake{}. 
Basically, \doubletake{} should check whether this thread's parent is joining on this thread or not. 
If this thread's parent is already waiting for its termination, it simply marks the status of 
this thread to be joined and wakes up the joining thread. 
If not, this thread can wait on a thread-private conditional variables. 
\doubletake{} delays a thread exit to the end of current epoch.

\subsubsection{Re-execution}
A thread is waiting for its turn to run if a thread is created in current epoch.    

\subsection{Thread Synchronizations}

\label{sec:sync}

Different order of thread synchronizations can lead to totally different memory uage. 
In order to guarantee deterministic replay of thread synchronizations, previous work
actually forces threads to do synchronizations in a global order and
recordes both lock and unlock operations ~\cite{TERN, PRES}. 
However, forcing a global order of synchronizations can greatly 
reduce parallelism and introduce significant performance overhead.
Also, it is unnecessary to record unlock order too.

Unlike previous approaches, \doubletake{} only records local orders of synchronizations.
Synchronizations on two different synchronization variables can be performaed
in parallel. From a thread's point of view, if a program do not have a race and
all synchronizations of a thread are repeated deterministically, then
\doubletake{} can guarantee memory usage of this thread, which also guarantees to 
repeat the same buffer overflows in re-execution phase. 
If a program do have a race, forcing a global order of synchronizations in the production 
run also can not completely avoid races. This also implies that a global order 
can not always guarantee determinstic memory uage.   
\doubletake{} prefers performance for racey programs, while relying on multiple re-executions 
to repeat buffer overflows if a program does have a race problem. 

\doubletake{} records the order of \texttt{pthread\_mutex\_lock}, conditional wakenup and
different signalling functions.
Signalling functions actually calls system calls, which are handled by the procedure discussed 
in Section~\ref{sec:inepoch}.
 Conditional wakenup is actually related to \texttt{pthread\_cond\_wait},
which actually includes conditional wait and conditional wakenup phases. 
Conditional wait atomically releases mutex and waits on corresponding conditional variable, while conditional wakenup actually locks corresponding mutex before returning. 
Thus, we can turn \texttt{pthread\_cond\_wait} to two operations, \texttt{pthread\_mutex\_unlock} and
\texttt{pthread\_mutex\_lock} correspondingly. So we only record the order of conditional wakenup in
production runs. 
 
\doubletake{} also provides an option to record the order of passing a specified barrier, which it is 
not necessary to do this by default.
It is noted that \doubletake{} do not record the order of unlock operations
and conditional signal operations.
It is totally unnecessary to record unlock operations since recording the order of actual 
lock aquiring operations is enough to guarantee a deterministic replay of critical sections.  
Conditional signal and broadcast operations are skiped for the same reason. 

%why two different synchronization is not important?
How to replay this?
How to handle nested locks? 
The replaying is considerred to be two steps: we first advance thread's entry when we met a .

Maybe pseudo code for this.
 
\subsubsection{Normal Execution}
In production runs, \doubletake{} intercepts all synchronizations and 
records orders of synchronizations, such as lock, conditional waken up and signals, 
based on different synchronization variables. 
It maintains a list for each synchronization variable and records synchronization
events on its corresponding list. 
In order to quickly locate its list when a synchronization is intercepted, \doubletake{}
utilizes original synchronization variables to store addresses of list and actual synchronization
variable. 

For a synchronization event like lock, \doubletake{} records the following information:
which thread issues this synchronization event; what is the result of this synchronization.
A naive implementation is to allocate memory from internal memory allocator every time. 
However, for some applications having significant amount of synchronizations,
memory allocations to record synchronization events contributes much performance overhead. 
For example, \texttt{fluidanimate} runs several times slower because of huge amounts
of synchronizations inside. \doubletake{} uses a pre-allocated list for those recordings. 
More specifically, each thread has a pre-allocated list in the beginning of an epoch. 
When a synchronization occurs on this thread, it can get an entry from this thread and 
record a synchronization event on this entry.
Since \doubletake{} always gets an entry from current thread issuing a synchronization,
there is no need to utilize a lock, which also helps reducing overhead. 
By doing this, \doubletake{} greatly reduce performance overhead of logging synchronization
events. For example, performance overhead of \texttt{fludidanimate} are reduced to around 40\%.

 
\subsubsection{Re-execution}
As described above, for a synchronization event in a thread,
\doubletake{} allocates an entry from current thread to record this synchronization event 
and inserted it into synchronization variable's corresponding list. 
This implies that a synchronization event belongs to two lists, 
a list for all synchronizations of this synchronization variable (SyncVariableList) and 
a list for all synchronizations in a thread (ThreadSyncList). 

\begin{figure}[!ht]
{\centering
\subfigure{\lstinputlisting[numbers=none,frame=none,boxpos=t]{figure/lockunlock.pseudocode}}
\caption{Lock and unlock of reproduction runs. \label{fig:lockunlock}}
}
\end{figure}

Reproducing synchronizations involves in manipulating these two lists using 
{\it sempaphore replay}, similar to TERN ~\cite{TERN}.
We listed the pseudocode of ``lock'' of reproduction runs in Figure~\ref{fig:lockunlock}.
\doubletake{} assigns a semaphore for each thread and controls the order 
of synchronizations based on semaphores: in lock acquisitions, 
a thread waits on its semaphore and advances ThreadSyncList after this semaphore; 
In lock releases, a thread increments the semaphore of next thread on the same 
synchronization variable.
However, \doubletake{} only records local synchronization order, instead of global order,
synchronizations replaying of \doubletake{} is much more subtle.
In order to handle those unsuccessful lock acquisitions, \doubletake{} only waits for a semaphore 
if this lock is successfully acquired in the production run. 
Also, to support nesting locks, in lock acquisitions after {\it advanceThreadSyncList()}, 
\doubletake{} signals current thread if next event of this thread is already 
in its pending list, which means that this thread should have its turn.
For lock releases, \doubletake{} adds next event of SyncVariableList to corresponding thread's 
pending list if the event is not the first event of corresponding thread instead of incrementing
its semaphore directly. 
Since performance of reproduction runs is not the main focus, only occurring for those programs 
having buffer overflows, \doubletake{} are using the same lock for all lists' manipulations to
avoid races.  




%\section{Optimization}
%\input{optimization}

\section{Evaluation}
\label{sec:evaluation}

All evaluations are performed on a quiescent Intel Core 2 dual-processor system equipped with 
16GB RAM. 
Each processor is a 4-core 64-bit Intel Xeon running at 2.33 GHz with a 4MB
shared L2 cache and 32KB private L1 data cache. 
The underlying operating system is unmodified CentOS 5.5, running with Linux kernel
version 2.6.18-194.17.1.el5. The glibc version is 2.5. 
In order to compare the performance fairly, all benchmarks were built as 64-bit executables 
using LLVM compiler (version 3.2). The compiler optimization level is set to ``-O1'' 
so memory allocation callsites can be reported precisely.
%since we can not report line number of source code with optimization level larger than
%``-O2''.
In our evaluations of this section, 
we choose two popular benchmark suites, Phoenix~\cite{phoenix-hpca} 
and PARSEC~\cite{parsec}. 
Some programs of PARSEC cannot be compiled or run succesfully with LLVM are excluded here.
For example, \texttt{facesim} can't be compiled 
because \texttt{llvm} forces a much stricter C++ rule. 
\texttt{canneal} can't be run successfully even it is compiled by original \texttt{llvm} compiler.

In this section, our evaluations aim to answer the following questions:
\begin{itemize}
\item
  How effective is \Predator{} on detecting and predicting false sharing problem (Section ~\ref{sec:effective})?

\item
  What is the performance overhead of \Predator{} with and without prediction
  (Section ~\ref{sec:perfoverhead})?

\item
  What is the memory overhead of \Predator{}~ (Section~\ref{sec:memoverhead})?
\end{itemize}


\subsection{Detection and Prediction Effectiveness}
\label{sec:effective}

\subsubsection{Benchmarks}
\label{sec:benchmarks}
Results on two benchmark suites, Phoenix and PARSEC, 
are listed in the Table~\ref{table:detection}. 

%Our results show that \Predator{} not only capture previously-discovered
%false sharing, but also many detect new false sharing places. The results
%are listed in Table~\ref{table:detection}. 

%http://www.technovelty.org/tips/getting-a-tick-in-latex.html
%http://tex.stackexchange.com/questions/42619/x-mark-to-match-checkmark
%\begin{comment}
\begin{table*}[ht!]
{
\centering
\begin{tabular}{l|r|r|r}
\hline
{\bf \small Benchmark} & {\bf \small Source Code} & {\bf \small New} & {\bf \small Improvement} \\
%{\bf \small Benchmark} & {\bf \small Source Code} & {\bf \small Type of False Sharing} & {New} & {\bf \small Improvement} \\
\hline
\small \textbf{histogram} & {\small histogram-pthread.c:213} & \cmark{} & 46.22\%\\
\small \textbf{reverse\_index} & {\small reverseindex-pthread.c:511} & \xmark{} & 0.09\%\\
\small \textbf{word\_count} & {\small word\_count-pthread.c:136} & \xmark{} & 0.14\%\\
\hline
\small \textbf{streamcluster} & {\small streamcluster.cpp:985} & \xmark{} & 7.52\% \\
\small \textbf{streamcluster} & {\small streamcluster.cpp:1907} & \cmark{} & 4.77\%\\
%\small \textbf{bodytrack} & {\small TrackingModel.cpp:59} & 0 & \cmark{} & \\
\hline
\hline
\small \textbf{linear\_regression} & {\small linear\_regression-pthread.c:133} & \xmark{} & 1206.93\%\\
\hline
\end{tabular}
\caption{Detection results of \Predator{} on Phoenix and PARSEC benchmark suites. \label{table:detection}}
}
\end{table*}

In this table, the first column lists those programs with false sharing problems. 
The second column shows precisely where the problem is. Since all found false sharing 
occurs inside the same heap object, we listed callsite source code information in this column.
The third column ``New'' marks whether this false sharing is newly found by \Predator{} or not.
False sharing problems found by previous work are marked with a crossmark(\xmark{}) and those 
newly found ones are marked with a tick mark(\cmark{}). 
The last column ``Improvement'' shows the performance improvement after fixing false sharing. 
The number is based on the average runtime of $10$ runs. 

\begin{comment}
\textbf{Tongping: how to say the following sentences}\\
The improvement rate is calculated by substracting $1$ from normalized runtime of orignal
runtime and new runtime. 
Taking \texttt{histogram} for an example here, original runtime of \texttt{histogram} is $0.75s$
and new runtime is $0.51s$, then the performance improvement is $(0.75/0.51) - 1$.
\end{comment}

As shown in the table, \Predator{} reveals several unknown false sharing problems. 
It is the first tool to detect two false sharing problems: one in \texttt{histogram} 
and one in line $1908$ of \texttt{streamcluster}. 
In \texttt{histogram}, multiple threads repeatedly modify different locations of the same heap object. 
Padding the data structure \texttt{thread\_arg\_t} fixes the false sharing problem and 
helps to improve the performance around 46\%.
In \texttt{streamcluster}, multiple threads are simultaneously accessing and updating 
the same \texttt{bool} array, \texttt{switch\_membership}. 
By simply changing this array to \texttt{long} type contributes to about 4.7\% performance improvement.

%, although it is not a complete fix of false sharing. 
%None of these two false sharing problems has been reported by previous tools.
All other false sharing problems have been discovered by one of previous tools, 
\sheriff{}~\cite{sheriff}. 
Same as \sheriff{}, we do not see much performance improvement for \texttt{reverse\_index} and 
\texttt{word\_count} since the number of updates inside them is not significantly large. But they
are actual false sharing problems that have been verified manually by us.

\texttt{streamcluster} has another false sharing problem at line $985$. 
Different threads change the same object (\texttt{work\_mem}) simultaneously. 
Authors of \texttt{streamclsuter} have already realized possible
false shairing problems and meant to utilize a macro \texttt{CACHE\_LINE} to avoid it. Unfortunately,
the defaulted value of this macro is setted to $32$ bytes, which is different with the actual
cache line size of our hardware. By setting to $64$ bytes instead, we achieve around $7.5\%$ performance
improvement.

\texttt{linear\_regression} has a severe false sharing problem, 
while fixing it improves the performance more than $12\times$.
In this benchmark, different threads constantly update their thread specific locations 
inside a heap object(\texttt{tid\_args}), causing a huge amount of cache invalidations. 
As pointed out by Nanavati et al.~\cite{OSdetection}, false sharing 
occurs when using \texttt{clang} compiler and disappears when using \texttt{GCC} with optimization
-O2 and -O3.  
During our evaluation, because our customized memory manager has different allocation 
metadata for each heap object, this false sharing do not 
occur at all when we are compiling this program
using \texttt{clang-3.2} at ``-O1'' optmization level.
Detailed reason of this can be seen in Section~\ref{sec:predicteval}.
Thus, \Predator{} actually can not detect this false sharing problem without enabling 
prediction mechanism. Also, none of existing tools can detect false sharing without occurence.
Using the prediction mechanism discussed in Section~\ref{sec:prediction}, 
\Predator{} can always capture false sharing problems in this benchmark.
This also examplify the importance of prediction tool. 

\subsubsection{Real Applications}
To verify its practicability, we further evaluate \Predator{} 
on several widely-used real applications, which none of previous work has evaluated.  
These real applications include a server application \texttt{MySQL}~\cite{mysql}, 
a common C++ library \texttt{boost}~\cite{libfalsesharing} 
and a distributed memory object caching system \texttt{memcached}, a network retriver \texttt{aget}, 
a parallel bzip2 file compressor \texttt{pbzip2} and a parallel file scanner \texttt{pfscan}.
For \texttt{MySQL} and \texttt{boost},
we evaluate their specific versions, \texttt{MySQL-5.5.32} and
\texttt{boost-1.49.0}, which are known to have some false sharing problems.

The false sharing problem in \texttt{MySQL} has caused significant scalability problem and
it was very difficult to be identified. 
According to the architect of \texttt{MySQL} Mikael Ronstrom, ``we had gathered specialists on 
InnoDB..., participants from MySQL support... and a number of generic specialists on 
computer performance...'', ``the fruit of the meeting ... were able to 
improve \texttt{MySQL} performance by 6$\times$ with those scalability fixes''. 
The false sharing of boost library is caused by the special usage of \texttt{spinlock} pool and fixing
it brings 40\% performance improvement. 
\Predator{} is able to succesfully detect false sharing locations
in both \texttt{MySQL} and \texttt{boost} library. 
For the other four applications, \Predator{} doest not find serere false sharing problems.

\subsubsection{Prediction Effectiveness}
\label{sec:predicteval}
For prediction effectiveness, we evaluate whether \Predator{} can always capture a false sharing
problem without occurrence.
\texttt{linear\_regression} benchmark is selected here because of the following two reasons:
\begin{enumerate}
\item
False sharing problem of this benchmark can not be detected without prediction, see Section~\ref{sec:benchmarks}. 

\item
False sharing severely degrades performance when it actually occurs. 
Hence, it is a serious problem that should be always detected. 
\end{enumerate}

\begin{figure}[!h]
{\centering
\subfigure{\lstinputlisting[numbers=none,frame=none,boxpos=t]{fig/linearregression.psedocode}}
\caption{False sharing problem inside \texttt{linear\_regression} benchmark.
\label{fig:linearregression}}
}
\end{figure}

Figure~\ref{fig:linearregression} shows the data structure and source code
experiencing false sharing.
The size of this data structure, \texttt{lreg\_args}, is $64$ bytes 
when we use \texttt{clang} compiler to compile $64$bit binary at ``-O1'' optimization level.
For this false sharing problem, the main thread allocates an array with the number of elements equals
to that  of underlying hardware cores.
Each element is a \texttt{lreg\_args} type with $64$ bytes. 
Then this array is passed to different threads (\texttt{lreg\_thread}) 
so that each thread only updates its thread-dependent area, see Figure~\ref{fig:linearregression}.
False sharing occurs if two threads happens to update data in the same cache line. 
However, different fields of \texttt{lreg\_args} has different access pattern:
only those fields between $SX$ and $SXY$ (totally around $40$ bytes) are constantly read and updated.
Consequently, the performance of \texttt{linear\_regression} is very sensitive to 
the starting address of false sharing object (see Figure~\ref{fig:perfsensitive}),
which can be changed by many dynamic properties according
to the discussion in Section~\ref{sec:intro}.

Figure~\ref{fig:perfsensitive} shows performance sensitivity to 
offsets of the starting address between the false sharing object and corresponding cache lines. 
When the offset is $0$ or $56$ bytes, this benchmark achieves its optimal performance 
and has no false sharing at all.
When the offset is $24$ bytes, this benchmarks runs around $15$ times slower 
than its optimal performance.
When we eveluate detection effectiveness on its original code, 
our customized memory manager happens to make the offset $56$ bytes. 
As a result, \Predator{} can not detect false sharing in this benchmark 
without enabling prediction because of no occurrence of false sharing problem.
This situation happens to all existing tools: they can only detect false sharing problems when
they occur. 

In contrast, the prediction mechanism designed in \predator{} 
amis to address this problem.  Results of our evaluations shows 
that \Predator{} can always predict the false sharing problem in this
benchmark no matter what the offset value is. 
This explains the effectiveness of \Predator{}.

\subsection{Performance Overhead}
\label{sec:perfoverhead}

\begin{figure*}[!ht]
\begin{center}
\includegraphics[width=6.5in]{fig/perf}
\end{center}
\caption{
Performance overhead of \Predator{} with and without prediction.
\label{fig:perf}}
\end{figure*}

To avoid the effect caused by extreme outliers, all performance data shown in this section
are based on the average result of $10$ runs while excluding the maximum and minimum values.
Actual performance overhead with and without prediction 
can be seen in the following figure~\ref{fig:perf}. 

From this figure, we can see for all $16$ benchmarks from Phoenix and PARSEC
benchmark suites that \Predator{} with prediction imposes around $6.7\times$
performance overhead. 
If we remove prediction from \Predator{}, we cannot observe significant performance difference.
This means that prediction of \Predator{} only introduces very minimum performance overhead. 

Among these programs, five of them have more than $8\times$ performance overhead, 
including \texttt{histogram, kmeans, bodytrack, ferret} and \texttt{swaptions}. 
Program \texttt{histogram} has the most performance overhead and 
runs more than $26$ slower than original executions. 
It has a severe false sharing problem inside, and tracking detailed access for those
problematic cache lines exacerbates the 
false sharing effect (see more discussion on this in Section~\ref{sec:sample}). 
For \texttt{bodytrack} and \texttt{ferret}, \Predator{} found a large amount of cache lines with 
writes larger than {\it Tracking-Threshold}. 
Tracking all accesses details for those cache lines 
imposes significant performance overhead. 
Currently, we have not identified the reasons 
why \texttt{kmeans} runs much slower in \Predator{}.   

In our evaluation, we do not observe signficant performance overhead on 
\texttt{matrix\_multiply, blackscholes} and 
\texttt{x264}.
Possibly a large portion of computations operates on stack variables, which are
not tracked by \Predator{}. 

\subsection{Memory Overhead}
\label{sec:memoverhead}
Since \Predator{} pre-allocates a huge block of memory ({\it virtual memory}) 
using \texttt{mmap} system call for its heap usage, 
virtual memory can not be used to evaluate actual memory overhead imposed by our tool. 
Hence, we only evaluate physical memory used for each application. 
According to the discussion of Justin et al. ~\cite{memusage}, proportional set size (PSS) 
in \texttt{/proc/self/smaps} is a suitable number since it reflects memory increase to the system
by running this application. 

To get PSS data, we start a script program to save 
corresponding \texttt{smaps} files periodically.
For each \texttt{smaps} file, we calculate the sum of PSSs for different
memory mappings and uses it as total physical memory usage for this application.
Among all collected \texttt{smaps} files, we choose the maximum number of
different files for comparison since it represents the maximum memory overhead to run this application.
%It is noted that we remove the physical memory usage of   
Results of maximum memory usage is shown in Figure~\ref{fig:memusage}. As we can see,
\Predator{} does not introduce substantial memory usage overhead 
for all eveluated bencharmks, except for \texttt{swaptions}. 
Removing \texttt{swaptions} from comparisons reduces 
the average memory overhead from 64\% to 22\%. 

The reason why \texttt{swaptions} introduces $7.8\times$ memory overhead is that 
its original memory usage is too small (only $3KB$).
Adding the code of detection, prediction and
reporting contributes to a large portion of memory overhead. 

\begin{figure*}
\begin{center} 
\includegraphics[width=6.5in]{fig/memusage}
\end{center}
%\includegraphics{fig/potential.pdf}
\caption{Memory usage overhead}
\label{fig:memusage}
\end{figure*}




\section{Discussion}
\subsection{Limitations}
\doubletake{} can detect those heap buffer overflows overwritting ``guard zones'', 
either caused by direct memory access or caused by
incorrect library calls (e.g. \texttt{memcpy} or \texttt{strcpy}). 
\doubletake{} can detect most of heap underflows: they can be detected if they are
aligned by \texttt{power of 2} or they are freed in the end of program. 
For those objects that they are not freed and not aligned, \doubletake{} can not detect them.
 
\doubletake{} can not detect buffer overflows on the globals and the stack 
since \doubletake{} can not change the layout of globals by inserting guard zones around them.
AddressSanitizer is a complementary approach for this: we can rely on AddressSanitizer 
to put guard zones for globals, then \doubletake{} can check the overflow of the globals accumulatively. 

\doubletake{} can not detect the usage-after-free memory error, which already implemented by 
AddressSanitizer. But their mechanism can be implemented in our framework easily and provide
this detection probability. 

\textbf{We need multiple runs for programs with race conditions}.
Although the idea of using re-execution and watch point can be applied to multithreaded programs,
\doubletake{} has not targeted for multithreaded programs currently. 

\doubletake{} can not detect those non-continguous buffer overflows. 
\subsection{Future Work} 
The first possible work of \doubletake{} is to extend it on multithreaded programs because
multithreaded programs is universal these days. 
However, there are some challenges to do this: how we can repeat the execution of 
a multithreaded program given random memory accesses from different threads;
how to handle those synchronizations without violating the shared-memory
semantics. 

Another possible work is to extend \doubletake{} to find more memory errors, such as dangling pointer 
errors, memory usage-after-free errors.
We possibly have to change the memory allocator in order to achieve memory usage-after-free errors
since we are utilizing the first words of freed memory blocks in the free list management. 

The last but not least, we can combine the \doubletake{} mechanism and \texttt{gdb} mechanism
to provide an interactive debugging system without actually using \texttt{gdb}. It can invoke this 
interactive debugging mechanism whenever bugs are found. 


\section{Related Work}
\label{sec:relatedwork}

The proliferation of multicore systems has increased interest in tool
support to detect false sharing, since standard profilers like
OProfile~\cite{oprofile} or gprof~\cite{gprof} only report overall
cache misses.

\paragraph{Simulation and Instrumentation Approaches:} Schindewolf describes a system based on the SIMICS functional
simulator that reports full cache miss information, including
invalidations caused by
sharing~\cite{falseshare:simulator}. Pluto
uses the Valgrind framework to track the sequence of load and store
events on different threads and reports a worst-case estimate of
possible false sharing~\cite{falseshare:binaryinstrumentation1}. Similarly,
Liu uses Pin to collect
memory access information and then reports total false sharing miss
information~\cite{falseshare:binaryinstrumentation2}.

Independently and in parallel with this work, Zhao et al.\ developed a
tool designed to detect false sharing and other sources of cache
contention in multithreaded applications~\cite{zhao:vee:2011}. This
tool uses shadow memory to track ownership of cache lines and cache
access patterns. It is currently limited to at most 8 simultaneous
threads. Like Liu above, it only reports an overall false sharing rate
for the whole program, placing the burden on programmers to examine
the entire source base to locate any instances of false
sharing.

Unlike \sheriffdetect{}, these systems generally suffer from high
performance overhead ($5-200\times$ slower) or memory overheads.
They cannot differentiate true sharing
from false sharing, yielding numerous false positives. Because they
operate at the binary level, they can all be misled by aliasing due to
memory object reuse. Finally, and most importantly, they do not point
to the objects responsible for false sharing, limiting their value to
the programmer.

\paragraph{Sampling-Based Approaches:}
Intel's performance tuning utility (PTU)~\cite{detect:ptu,
detect:intel} uses event-based sampling, allowing it to operate
efficiently. PTU can discover shared physical cache lines, and can
identify possible false sharing at the grain of individual function
calls. PTU suffers from a high false positive rate caused by aliasing
due to reuse of heap objects, and reports false sharing instances that
have no impact on performance. PTU cannot differentiate true from
false sharing or pinpoint the source of false sharing problems,
unlike \sheriffdetect{}. Section~\ref{sec:evaluation} contains an
extensive empirical comparison of PTU to \sheriffdetect{}
demonstrating PTU's relative shortcomings.

Pesterev et al.\ describe DProf, a tool that leverages AMD's
instruction-based sampling hardware to help programmers identify the
sources of cache misses~\cite{DProf}. DProf requires manual annotation
to locate data types and object fields, and cannot detect false
sharing when multiple objects reside on the same cache line. By
contrast, \sheriffdetect{} is architecture independent, requires no
manual intervention, and precisely identifies false sharing regardless
of the flow of data or which data types are involved.

\paragraph{False Sharing Avoidance: }
In some restricted cases, it is possible to eliminate false sharing,
obviating the need for detection. Jeremiassen and Eggers describe a
compiler transformation that adjusts the memory layout of applications
through padding and alignment~\cite{falseshare:compile}.  Chow et al.\
describe an approach that alters parallel loop scheduling to avoid
sharing~\cite{falseshare:schedule}. The effectiveness of these
static analysis based approaches is primarily limited to regular,
array-based scientific codes, while \sheriffprotect{} can prevent false
sharing in any application.

Berger et al.\ describe Hoard, a scalable memory allocator can reduce
the likelihood of false sharing of distinct heap
objects~\cite{BergerMcKinleyBlumofeWilson:ASPLOS2000}. Hoard limits
accidental false sharing of entire heap objects by making it unlikely
that two threads will use the same cache lines to satisfy memory
requests, but this has no effect on false sharing within individual heap
objects, which \sheriffprotect{} avoids.

\paragraph{Other Related Work:}
\sheriff{} borrows the
process-as-thread model pioneered by Grace~\cite{grace}, but otherwise
differs from it in its semantics, generality, and goals.

Grace is a process-based approach designed to prevent concurrency
errors, such as deadlock, race conditions, and atomicity errors by
imposing a sequential semantics on speculatively-executed
threads. Grace supports only fork-join programs without inter-thread
communication (e.g., condition variables or barriers), and rolls back
threads when their effects would violate sequential semantics.

\sheriff{} extends Grace to handle
arbitrary multithreaded programs; for example, Grace is incompatible
with any applications that employ inter-thread communication,
including the PARSEC benchmarks we examine here. \sheriff{} applies
diffs at synchronization points in the program to enable updates
without rollback, giving it far greater performance (but different
semantics) than Grace. \sheriff{} does not eliminate concurrency
errors, but instead allows applications to selectively track updates
and isolate memory, enabling tools like \sheriffdetect{}
and \sheriffprotect{}.


% eliminate global lock
% possibly adopting a page-ownership protocol as used by ...

\section{Conclusion}
\label{sec:conclusion}
This paper presents Sheriff, one software-only system to precisely detect false sharing problems
in multithreaded programs. 
Sheriff's key contribution is its runtime system framework, which simulating the running 
of threads using process. 
Sheriff uses the process's complete separation mechanism and page fault mechanism to capture writes
in one transaction by combining with twin page mechanism. 
In order to capture continuous writes in the same transaction, 
Sheriff introduce the sampling mechanism for those long-running
transactions. 
Unlike previous tools, Sheriff intercepts those memory allocation and de-allocation function calls so that
those false positives caused by re-use of dynamic objects can be avoided completely in Sheriff and also
Sheriff can attach callsite on those memory allocations, which can help to report those callsite 
information about one false sharing objects and save manual effort to locate the false sharing problems.

Sheriff can use preload attributes of library, which can work on unaltered binary files and makes the deployment
simple. Since Sheriff don't use the advanced hardware support, which can be used to find false sharing problems
for those legacy applications running on original hardware. 
For most of applications, sheriff only incurs reasonable runtime overhead, making it a 
practical choice even for those complex system which should run long time. 


\section{Acknowledgements}
\begin{comment}
We want to thank Qiang Zeng, Dinghao Wu and Peng Liu for providing
their test cases used in their Cruiser paper.  We also thank Scott Kaplan for his
suggestions and comments in the development of \doubletake{}.
\end{comment}

{
\bibliographystyle{abbrv}
\bibliography{refs}
}

\end{document}
