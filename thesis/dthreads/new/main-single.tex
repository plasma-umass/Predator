%DIF PREAMBLE EXTENSION ADDED BY LATEXDIFF
%DIF UNDERLINE PREAMBLE %DIF PREAMBLE
\RequirePackage[normalem]{ulem} %DIF PREAMBLE
\RequirePackage{color}\definecolor{RED}{rgb}{1,0,0}\definecolor{BLUE}{rgb}{0,0,1} %DIF PREAMBLE
\providecommand{\DIFadd}[1]{{\protect\color{blue}\uwave{#1}}} %DIF PREAMBLE
\providecommand{\DIFdel}[1]{{\protect\color{red}\sout{#1}}}                      %DIF PREAMBLE
%DIF SAFE PREAMBLE %DIF PREAMBLE
\providecommand{\DIFaddbegin}{} %DIF PREAMBLE
\providecommand{\DIFaddend}{} %DIF PREAMBLE
\providecommand{\DIFdelbegin}{} %DIF PREAMBLE
\providecommand{\DIFdelend}{} %DIF PREAMBLE
%DIF FLOATSAFE PREAMBLE %DIF PREAMBLE
\providecommand{\DIFaddFL}[1]{\DIFadd{#1}} %DIF PREAMBLE
\providecommand{\DIFdelFL}[1]{\DIFdel{#1}} %DIF PREAMBLE
\providecommand{\DIFaddbeginFL}{} %DIF PREAMBLE
\providecommand{\DIFaddendFL}{} %DIF PREAMBLE
\providecommand{\DIFdelbeginFL}{} %DIF PREAMBLE
\providecommand{\DIFdelendFL}{} %DIF PREAMBLE
%DIF END PREAMBLE EXTENSION ADDED BY LATEXDIFF

%\documentclass{sigplanconf}
%\nocaptionrule

% \documentclass[twocolumn,9pt]{article}
% \documentclass[twocolumn,10pt]{acm_proc_article-sp}

% \documentclass{acm_proc_article-sp}
% \documentclass[9pt]{sigplanconf}
\documentclass{acm_proc_onecol}

\date{} % \vspace*{-0.2in}}

% Make sure to put back 

\newcommand{\punt}[1]{}

\punt{

Notes from Daan Leijen:

Generally, I think the paper could have been clearer to me if it would have 
separated out more the actual implementation from the intended semantics.
Currently I found myself trying to decode the implementation to figure out
the semantics. Also, in the introduction you remark “Dthreads guarantees
deterministic execution of multithreaded programs even in the presence of
data races (notwithstanding external sources of non-determinism like I/O):
given the same sequence of inputs, a program using dthreads always produces
the same output”. This is somewhat ambiguous to me. Perhaps it is good to
remark that a Dthread program is deterministic with respect to any
particular scheduling of the threads? Clearly, any other source of
non-determinism will cause a dthread program to be non-deterministic (like I/O).

}

\usepackage{endnotes,xspace}

\newcommand{\footnotenonumber}[1]{{\def\thempfn{}\footnotetext{\small #1}}}
\usepackage[normalem]{ulem}
\usepackage{graphicx}

\usepackage{mathptmx} % rm & math
\usepackage[scaled=0.90]{helvet} % ss
\usepackage{courier} % tt
% \normalfont
\usepackage[T1]{fontenc}

% \usepackage{lmodern}
% \usepackage{times}
\usepackage{subfigure}
\usepackage{url}
\urlstyle{rm}
\usepackage[
      colorlinks=false,    %no frame around URL
      urlcolor=black,    %no colors
      menucolor=black,    %no colors
      linkcolor=black,    %no colors
      pagecolor=black,    %no colors
]{hyperref}

\usepackage{color}
\usepackage{listings}
\usepackage{amsmath}
\usepackage{amsfonts}
\usepackage{amssymb}
\usepackage{comment} 
\usepackage{setspace}
\singlespacing
%\onehalfspacing
\newtheorem{thm}{Theorem}
\newtheorem{prop}[thm]{Proposition}
\newtheorem{cor}[thm]{Corollary}
\newtheorem{lem}[thm]{Lemma}
\newtheorem{defn}[thm]{Definition}

\newcommand{\cfunction}[1]{{\bf \tt #1}}
\newcommand{\malloc}{\cfunction{malloc}}
\newcommand{\realloc}{\cfunction{realloc}}
\newcommand{\free}{\cfunction{free}}
\newcommand{\madvise}{\cfunction{madvise}}
\newcommand{\brk}{\cfunction{brk}}
\newcommand{\sbrk}{\cfunction{sbrk}}
\newcommand{\mmap}{\cfunction{mmap}}
\newcommand{\munmap}{\cfunction{munmap}}
\newcommand{\mprotect}{\cfunction{mprotect}}
\newcommand{\mlock}{\cfunction{mlock}}

\hyphenation{app-li-ca-tion}
\hyphenation{Die-Hard}
\hyphenation{Ar-chi-pe-la-go}
\hyphenation{buf-fer}
\hyphenation{D-threads}
\hyphenation{Heap-Layers}
\hyphenation{wait-Token}
\hyphenation{mul-ti-threa-ded}
\hyphenation{me-m-ory}

\hyphenation{pthread-create}
\hyphenation{pthread-self}
\hyphenation{pthread-mutex-lock}
\hyphenation{pthread-mutex-unlock}

\newcommand{\dthreads}{{\scshape Dthreads}}
\newcommand{\Dthreads}{{\scshape Dthreads}}
\newcommand{\pthreads}{\texttt{pthreads}}

\lstdefinelanguage{c++threads}[]{c++}{morekeywords={pthread_create,pthread_join}}

\lstset{language=c++threads, basicstyle=\ttfamily\scriptsize,frame=trbl,tabsize=4} % ,numbers=left,numberstyle=\tiny}

\definecolor{Gray}{cmyk}{0,0,0,0.5}

\begin{document}

%\conferenceinfo{SOSP 2011,} {October 22--26, Cascais, Portugal.}
%\CopyrightYear{2011}
%\copyrightdata{XXX-X-XXXXX-XXX-X/XX/XX}

\title{{\huge \bf \dthreads{}}: Efficient Deterministic Multithreading}

% \authorinfo{\emph{authorship list removed for anonymity}}

\punt{
\authorinfo{Tongping~Liu \and Charlie~Curtsinger \and Emery~D.~Berger}
{Dept.\ of Computer Science \\
University of Massachusetts, Amherst \\
Amherst, MA 01003}
%{\{tonyliu,charlie,emery\}@cs.umass.edu}
}

% \punt{
\numberofauthors{1}
\author{
\alignauthor Tongping~Liu, Charlie~Curtsinger, and Emery~D.~Berger \\
\affaddr{Department of Computer Science} \\
\affaddr{University of Massachusetts, Amherst} \\
\affaddr{Amherst, MA 01003} \\
\email{\{tonyliu,charlie,emery\}@cs.umass.edu} \\
%\alignauthor Charlie~Curtsinger \\
%\affaddr{Dept.\ of Computer Science} \\
%\affaddr{Univ. of Massachusetts, Amherst} \\
%\affaddr{Amherst, MA 01003} \\
%\email{charlie@cs.umass.edu} \\
%\alignauthor Emery~D.~Berger \\
%\affaddr{Dept.\ of Computer Science} \\
%\affaddr{Univ. of Massachusetts, Amherst} \\
%\affaddr{Amherst, MA 01003} \\
%\email{emery@cs.umass.edu} \\
% }
}

\maketitle

\begin{comment}
\end{comment}

\begin{abstract}
Multithreaded programming is notoriously difficult to get right.  A key problem
is non-determinism, which complicates debugging, testing, and reproducing
errors. One way to simplify multithreaded programming is to enforce
deterministic execution, but current deterministic systems for C/C++ are
incomplete or impractical. These systems require program modification, do not
ensure determinism in the presence of data races, do not work with
general-purpose multithreaded programs, or run up to $8.4\times$ slower than
\pthreads{}.

This paper presents \dthreads{}, an efficient deterministic multithreading
system for unmodified C/C++ applications that replaces the \pthreads{} library.
\Dthreads{} enforces determinism in the face of data races and deadlocks. 
\dthreads{} works by exploding multithreaded applications into multiple
processes, with private, copy-on-write mappings to shared memory.  It uses
standard virtual memory protection to track writes, and deterministically orders
updates by each thread. By separating updates from different threads,
\dthreads{} has the additional benefit of eliminating false sharing.
Experimental results show that \dthreads{} substantially outperforms a
state-of-the-art deterministic runtime system, and for a majority of the
benchmarks evaluated here, matches and occasionally exceeds the performance of
\pthreads{}.
\end{abstract}

%  Language-based approaches require programmers to write their code in specialized languages. 


\punt{
\category{D.1.3}{Programming Techniques}{Concurrent Programming--Parallel Programming}
\category{D.2.5}{Software Engineering}{Testing and Debugging--Debugging Aids}

\terms
Design, Reliability, Performance

\keywords
Deterministic Multithreading, Determinism, Parallel Programming, Concurrency, Debugging,
Multicore
}


%%%%%%%%%%%%%%%%%%%%%%%%%%%%%%%%%%%%%%%%%%%%%%%%%%%%%%%%%%%%%%%%%%%%%%%%%%%%%%%%%%%%%%%%%%%%%
%%%%%%%%%%%%%%%%%%%%%%%%%%%%%%%%%%%%%%%%%%%%%%%%%%%%%%%%%%%%%%%%%%%%%%%%%%%%%%%%%%%%%%%%%%%%%

\section{Introduction}
\label{sec:introduction}

The advent of multicore architectures has made multithreaded
programming increasingly necessary, but writing multithreaded programs
remains painful. It is notoriously far more challenging to write
concurrent programs than sequential ones because of the wide range of
errors it can cause, including deadlocks and race
conditions~\cite{havender,76897,130623}. Because thread interleavings
are non-deterministic, different runs of the same multithreaded
program can unexpectedly produce different results. These
``Heisenbugs'' greatly complicate debugging, and eliminating them
requires extensive testing to account for possible thread
interleavings~\cite{DBLP:conf/icse/BallBHMQ09,DBLP:conf/asplos/BurckhardtKMN10}.

% Lots of recent work on bug finding. Getting better, but still difficult.

Instead of testing, one promising alternative approach is to attack
the problem of concurrency bugs by eliminating its source:
non-determinism. A fully \emph{deterministic multithreaded system}
would prevent Heisenbugs by ensuring that executions of the same
program with the same inputs always yield the same results, even in
the face of race conditions in the code. Such a system would not only
dramatically simplify debugging of concurrent
programs~\cite{Carver:1991:RTC:624586.625040} and reduce their
attendant testing overhead, but would also enable a number of other
applications. For example, a deterministic multithreaded system would
greatly simplify record and replay for multithreaded
programs~\cite{Choi:1998:DRJ:281035.281041,LeBlanc:1987:DPP:32387.32396}
and the execution of multiple replicas of multithreaded applications
for fault
tolerance~\cite{deterministic-process-groups,1134000,224058,replicant-hotos}.

Several recent software-only proposals aim at providing
deterministic multithreading, but these all suffer from a variety of
disadvantages. Language-based approaches are effective at removing
determinism but require programmers to write code in specialized
languages, which can be
impractical~\cite{Bocchino:2009:TES:1640089.1640097,Burckhardt:2010:CPR:1869459.1869515,Simpson:1999:SEE:330346.330357}. Recent
deterministic systems that target legacy programming languages
(especially C/C++) are either incomplete or impractical. Kendo ensures
determinism of synchronization operations with low overhead, but does
not guarantee determinism in the presence of data
races~\cite{1508256}. Grace prevents all concurrency errors but is
limited to fork-join programs, and although it is efficient, it can require
code modifications to avoid large runtime
overheads~\cite{grace}. CoreDet, a compiler and runtime system,
enforces deterministic execution for arbitrary multithreaded C/C++
programs~\cite{Bergan:2010:CCR:1736020.1736029}. However, it exhibits
prohibitively high overhead (running up to $8\times$ slower
than \pthreads{}; see Section~\ref{sec:evaluation}) and generates
thread interleavings at arbitrary points
in the code, complicating program debugging and testing.

\hspace{1em} \\
\noindent
\textbf{Contributions:}
This paper presents \textbf{\dthreads{}}, an efficient deterministic runtime
system for multithreaded C/C++ applications. \dthreads{} guarantees
deterministic execution of multithreaded programs even in the presence
of data races (notwithstanding external sources of non-determinism
like I/O): given the same sequence of inputs, a program
using \dthreads{} always produces the same output. \dthreads{}'
deterministic commit protocol not only eliminates data races but also
prevents lock-based deadlocks.

\dthreads{} is easy to deploy: it works as a direct replacement for
the \pthreads{} library, requiring no code modifications or
recompilation. \dthreads{} is also efficient. Its software
architecture avoids the need for expensive write buffers, and 
its commit protocol eliminates cache-line based false sharing, a
notorious performance problem for multithreaded programs. These two
features enable \dthreads{} to nearly match or even exceed the
performance of \pthreads{} for the majority of the benchmarks examined
here. \dthreads{}
thus is a significant improvement over the state of the art in
deployability and performance, and provides evidence that fully
deterministic multithreaded programming may be practical.

% XXX borrowed from Grace, XXX borrowed from Treadmarks, etc.


%Deployable: directly replaces
%the \pthreads{} library, requiring no code modifications.

% Summary of results. Improvements over state-of-the-art (CoreDet).

The remainder of this paper is organized as
follows. Section~\ref{sec:dthreads-architecture} describes
the \dthreads{} architecture and algorithms in depth, and
Section~\ref{sec:discussion} discusses key
limitations. Section~\ref{sec:evaluation} evaluates \dthreads{}
experimentally, comparing its performance and scalability
to \pthreads{} and CoreDet. Section~\ref{sec:related-work} provides an
overview of related work, Section~\ref{sec:future-work} describes
future directions, and Section~\ref{sec:conclusion} concludes.


\section{Related Work}
\label{sec:related-work}

The area of deterministic multithreading has seen considerable recent
activity. Due to space limitations, we focus here on software-only,
non language-based approaches.

% ~\cite{Bocchino:2009:TES:1640089.1640097,Burckhardt:2010:CPR:1869459.1869515,Simpson:1999:SEE:330346.330357}

Grace prevents a wide range of concurrency errors, including
deadlocks, race conditions, ordering and atomicity violations by imposing
sequential semantics on threads with speculative execution~\cite{grace}.  \dthreads{} borrows Grace's
threads-as-processes paradigm to provide memory isolation, but differs from Grace in terms of semantics, generality, and performance.

Because it provides the effect of a serial execution of all threads,
one by one, Grace rules out all interthread communication, including
updates to shared memory, condition variables, and barriers. Grace
supports only a restricted class of multithreaded programs: fork-join
programs (limited to thread create and join). Unlike
Grace, \dthreads{} can run most general-purpose multithreaded programs
while guaranteeing deterministic execution.

\dthreads{} enables far higher performance than Grace for several reasons:
It deterministically resolves conflicts, while Grace must rollback and re-execute threads that update any shared pages (requiring code modifications to avoid serialization); 
\dthreads{} prevents false sharing while Grace exacerbates it; and 
\dthreads{} imposes no overhead on reads.

CoreDet is a compiler and runtime system that represents the current
state-of-the-art in deterministic, general-purpose software
multithreading~\cite{Bergan:2010:CCR:1736020.1736029}. It uses
alternating parallel and serial phases, and a token-based global
ordering that we adapt for \dthreads{}. Like \dthreads{}, CoreDet
guarantees deterministic execution in the presence of races, but with
different mechanisms that impose a far higher cost: on average
$3.5\times$ slower and as much as $11.2\times$ slower than \dthreads{} (see
Section~\ref{sec:evaluation}). The CoreDet compiler instruments all
reads and writes to memory that it cannot prove by static analysis to
be thread-local.  CoreDet also serializes \emph{all} external library
calls, except for specific variants provided by the CoreDet runtime.

% \dthreads{} does not serialize library calls unless they perform 
% synchronization operations, and only traps on the first write to a page during
% a transaction.  Because of these differences, CoreDet runs 

CoreDet and \dthreads{} also differ semantically. \dthreads{}
only allows interleavings at synchronization points, but CoreDet relies on
the count of instructions retired to form quanta. This approach makes
it impossible to understand a program's behavior by examining the
source code---the only way to know what a program does in CoreDet (or
dOS and Kendo, which rely on the same mechanism) is to execute it on
the target machine. This instruction-based commit schedule is also
brittle: even small changes to the input or program can cause a
program to behave differently, effectively ruling out {\tt printf}
debugging. \dthreads{} uses synchronization operations as
boundaries for transactions, so changing the code or input does not
affect the schedule as long as the sequence of synchronization
operations remains unchanged.  We call this more stable form of determinism {\em robust determinism}.

% The use of synchronization points as commit boundaries also makes \dthreads{}
% code relatively \emph{robust}: when updates occur after a given number of 
% instructions retired (as in CoreDet and Kendo), it is impossible for 
% programmers to know when interleavings can occur. Such boundaries could vary 
% depending on the underlying architecture and would also be input-dependent, 
% meaning that slightly different inputs could lead to dramatically different
% thread interleavings. By contrast, \dthreads{} guarantees that only changes to
% the sequence of synchronization operations affect the order in which updates 
% are applied.

dOS~\cite{deterministic-process-groups} is an extension to CoreDet
that uses the same deterministic scheduling framework.  dOS provides
deterministic process groups (DPGs), which eliminate all internal
non-determinism and control external non-determinism by recording and
replaying interactions across DPG boundaries. dOS is orthogonal and
complementary to \dthreads{}, and in principle, the two could be
combined.

Determinator is a microkernel-based operating system that enforces
system-wide determinism~\cite{efficient-system-enforced}.  Processes
on Determinator run in isolation, and are able to communicate only at
explicit synchronization points.  For programs that use condition variables,
Determinator emulates a legacy thread API with quantum-based determinism similar
to CoreDet.  This legacy support suffers from the same performance and robustness problems as CoreDet.

Like Determinator, \dthreads{} isolates threads by running them in separate processes, but natively supports all \pthreads{} communication primitives.  \dthreads{} is a drop-in replacement for \pthreads{} that needs no special operating system support.

Finally, some recent proposals provide limited determinism. Kendo
guarantees a deterministic order of lock acquisitions on commodity
hardware (``weak determinism''); Kendo only enforces full (``strong'')
determinism for race-free
programs~\cite{1508256}. TERN~\cite{stable-deterministic} uses code
instrumentation to memoize safe thread schedules for applications, and
uses these memoized schedules for future runs on the same
input. Unlike these systems, \dthreads{} guarantees full determinism even
in the presence of races.

% \subsection{Other Related Work}

% Behavior-oriented parallelism (BOP)~\cite{1250760}.

% Transactional memory?

% Deterministic Record/Replay system: it is a different part, most of replay
% system's target is for debugging. In Deterministic programming language:
% Race detection.

%Languages.



\section{{\bf \Large \Dthreads{}} Overview}
\begin{figure*}[!ht]
{\centering
%\fbox{
\subfigure{\lstinputlisting[numbers=none,frame=none,boxpos=t]{fig/mainthread.example.pseudocode}}
\hspace{50pt}
\subfigure{\lstinputlisting[numbers=none,frame=none,boxpos=t]{fig/thread1.example.pseudocode}}
\hspace{50pt}
\subfigure{\lstinputlisting[numbers=none,frame=none,boxpos=t]{fig/thread2.example.pseudocode}}
%}
\caption{A simple multithreaded program with data races on \texttt{a} and \texttt{b}. With \pthreads{}, the output is non-deterministic, but \dthreads{} guarantees the same output on every execution.\label{fig:sample}}
}
\end{figure*}

\label{sec:dthreads-overview}

% \subsection{Overview}

We begin our discussion of how \dthreads{} works with an example execution of a
simple, racy multithreaded program, and explain at a high level how
\dthreads{} enforces deterministic execution.

Figure~\ref{fig:sample} shows a simple multithreaded program that,
because of data races, non-deterministically produces the outputs
``1,0,'' ``0,1'' and ``1,1.''  With \pthreads{}, the order in
which these modifications occur can change from run to run, resulting
in non-deterministic output. 

With \dthreads{}, however, this program \emph{always} produces the
same output, (``1,1''), which corresponds to exactly one possible
thread interleaving. \dthreads{} ensures determinism using the
following key approaches, illustrated in Figure~\ref{fig:architecture}:

\textbf{Isolated memory access:}
In \dthreads{}, threads are implemented using separate processes with
private and shared views of memory, an idea introduced by
Grace~\cite{grace}.  Because processes have separate address spaces,
they are a convenient mechanism to isolate memory accesses between
threads.  \dthreads{} uses this isolation to control the visibility of
updates to shared memory, so each ``thread'' operates independently
until it reaches a synchronization point (see
below). Section~\ref{sec:threadsasprocs} discusses the implementation
of this mechanism in depth.

\textbf{Deterministic memory commit:} 
Multithreaded programs often use shared memory for communication, so \dthreads{} must propagate one thread's writes to all other threads. To ensure deterministic
execution, these updates must be applied at deterministic times, and in a deterministic order.

\dthreads{} updates shared
state in sequence at synchronization points. These points
include thread creation and exit; mutex lock and unlock; condition
variable wait and signal; posix sigwait and signal; and barrier waits. Between
synchronization points, all code effectively executes within an
atomic \emph{transaction}. This combination of memory isolation between
synchronization points with a deterministic commit protocol guarantees
deterministic execution even in the presence of data races.

\textbf{Deterministic synchronization:}
\dthreads{} supports the full array of \pthreads{} synchronization
primitives.  Because current operating systems make no guarantees about the
order in which threads will acquire locks, wake from condition
variables, or pass through barriers, \dthreads{} re-implements these
primitives to guarantee a deterministic ordering.
Details on the \dthreads{} implementations of these primitives are
given in Section~\ref{sec:synchronization}.

\textbf{Twinning and diffing:}
Before committing updates, \dthreads{} first compares each modified
page to a ``twin'' (copy) of the original shared page, and then writes
only the modified bytes (diffs) into shared state (see
Section~\ref{sec:dthreads-optimizations} for optimizations that avoid
copying and diffing).  This algorithm is adapted from the distributed
shared memory systems TreadMarks and
Munin~\cite{dsm:munin,dsm:treadmarks}. The order in which threads
write their updates to shared state is enforced by a single global
token passed from thread to thread; see
Section~\ref{sec:sharedmem} for full details.

%%% \vfill %%% EDB

\subsection*{Fixing the data race example}
Returning to the example program in Figure~\ref{fig:sample}, we can
now see how \dthreads{}' memory isolation and a deterministic
commit order ensure deterministic output. \dthreads{} effectively
isolates each thread from each other until it completes, and then
orders updates by thread creation time using a deterministic
last-writer-wins protocol.

At the start of execution, thread 1 and thread 2 have the same view of
shared state, with $a = 0$ and $b = 0$.  Because changes by one thread
to the value of $a$ or $b$ will not be made visible to the other until
thread exit, both threads' checks on line 2 will be true.  Thread 1
sets the value of $a$ to 1, and thread 2 sets the value of $b$ to 1.
These threads then commit their updates to shared state and exit, with
thread 1 always committing before thread 2.  The main thread then has
an updated view of shared memory, and prints ``1, 1'' on every
execution.

This determinism not only enables record-and-replay and replicated
execution, but also effectively converts Heisenbugs into ``Bohr''
bugs, making them reproducible. In addition, \dthreads{} optionally
reports any conflicting updates due to racy writes, further
simplifying debugging.



\section{{\bf \Large \Dthreads{}} Architecture}
\label{sec:dthreads-architecture}

This section describes \dthreads{}' key algorithms---mem\-ory
isolation, deterministic (diff-based) memory commit, deterministic
synchronization, and deterministic memory allocation---as well as
other implementation details.

\subsection{Isolated Memory Access}

\label{sec:threadsasprocs}
To achieve deterministic memory access, 
\dthreads{} isolates memory accesses among different
threads between commit points, and commits the updates of each thread deterministically.

\dthreads{} achieves cross-thread memory isolation by 
replacing threads with processes.  In a multithreaded program running
with \pthreads{}, threads share all memory except for the stack.  Changes
to memory immediately become visible to all other threads.  Threads share
the same file descriptors, sockets, device handles, and windows.
By contrast, because \dthreads{} runs threads in separate processes, it must manage these shared
resources explicitly.

\begin{figure}[h]
{\centering
\includegraphics{fig/architecture-diagram}
\caption{An overview of \dthreads{} execution.\label{fig:architecture}}
}
\end{figure}

\subsubsection{Thread Creation}

\dthreads{} replaces the \texttt{pthread\_create()} function with
the \texttt{clone} system call provided by Linux. To create
processes that have disjoint address spaces but share the same file descriptor table, \dthreads{} uses the \texttt{CLONE\_FILES} flag.
\dthreads{} shims the \texttt{getpid()} function to return a single, globally-shared identifier.

\subsubsection{Deterministic Thread Index}
\label{sec:threadindex}
POSIX does not guarantee deterministic process or thread identifiers;
that is, the value of a process id or thread id is not deterministic.
To avoid exposing this non-determinism to threads running as
processes, \dthreads{} shims
\texttt{pthread\_self()} to return an internal thread index.
The internal thread index is managed using a single global variable
that is incremented on thread creation.  This unique thread index is
also used to manage per-thread heaps and as an offset into an array of
thread entries.

%\subsubsection{Stack and Heap} 
% TTT: Change the title since we don't care about stack and add the mapping information, otherwise, it is not clear how we
% do that, it is very important to do so.
\subsubsection{Shared Memory}
\label{sec:stackandheap}

To create the illusion of different threads sharing the same
address space, \dthreads{} uses memory mapped files to share memory
across processes (globals and the heap, but not the stack; see
Section~\ref{sec:discussion}).

\dthreads{} creates two different mappings for both the heap and the
globals.  One is a \emph{shared} mapping, which is used to hold shared state.
The other is a \emph{private}, copy-on-write (COW) per-process mapping that
each process works on directly.  Private mappings are linked to the
shared mapping through a single fixed-size memory-mapped file.
Reads initially go directly to the shared mapping,
but after the first write operation,
both reads and writes are entirely private.

Memory allocations from the shared heap use a scalable
per-thread heap organization loosely based on
Hoard~\cite{BergerMcKinleyBlumofeWilson:ASPLOS2000} and built using
HeapLayers~\cite{BergerZornMcKinley:2001}.  \dthreads{} divides the
heap into a fixed number of sub-heaps (currently 16).  Each thread
uses a hash of its deterministic thread index to find the appropriate sub-heap.

\subsection{Deterministic Memory Commit}
\label{sec:sharedmem}

\begin{figure}[b]
{\centering 
\includegraphics{fig/phase-diagram}
\caption{An overview of \dthreads{} phases. Program execution with \dthreads{} alternates between parallel and serial phases.\label{fig:phase}}
}
\end{figure}

Figure~\ref{fig:phase} illustrates the progression of parallel and serial phases. 
To guarantee determinism, \dthreads{} isolates memory accesses in the parallel phase. These accesses work on private copies of memory; that is, updates are not shared between threads during the parallel phase.
When a synchronization point is reached, updates are applied (and made visible) in a deterministic order.  
This section describes the mechanism used to alternate between parallel and serial execution phases 
and guarantee deterministic commit order, and the details of commits to shared memory.

\subsubsection{Fence and Token}
\label{sec:schedule}

The boundary between the parallel and serial phases is the internal
fence. We implement this fence with a custom barrier, because the
standard \pthreads{} barrier does not support a dynamic thread count
(see Section~\ref{sec:synchronization}).

\punt{
\label{sec:token}
\begin{figure}[ht]
\begin{lstlisting}
void waitFence(void) {
	lock();
	while(!isArrivalPhase()) { 
		CondWait();
	}

	waiting_threads++;
	if(waiting_threads < live_threads) {
		while(!isDeparturePhase()) {
			CondWait();
		}
	} else {
		setDeparturePhase();
		CondBroadcast();
	}

	waiting_threads--;
	if (waiting_threads == 0) {
		setArrivalPhase();
		CondBroadcast();
	}
	unlock();
}
\end{lstlisting}
\caption{Pseudocode for \dthreads{}' internal fence. ($\S$ \ref{sec:schedule})
\label{fig:internalFence}}
\end{figure}
}

\punt{
Figure~\ref{fig:internalFence} presents pseudocode for the internal fence.
Threads must wait at the fence 
until all threads from the previous 
If the thread is the last to enter the fence, it sets the departure phase and wakes the waiting threads (lines 12-15).  
As threads leave the fence, they decrement the waiting thread count.  The last thread to leave sets the fence to the arrival phase and wakes any waiting threads (lines 17-21).
}

Threads wait at the internal fence until all threads from the
previous fence have departed. Waiting threads must block until
the departure phase. If the thread is the last to enter the fence, it
initiates the departure phase and wakes all waiting threads.  As threads
leave the fence, they decrement the waiting thread count.  The last
thread to leave sets the fence to the arrival phase and wakes any
waiting threads.

To reduce overhead, whenever the number of running threads is less than or equal
to the number of cores, waiting threads block by spinning rather
than by invoking relatively expensive cross-process \pthreads{}
mutexes. When the number of threads exceeds the number of
cores, \dthreads{} falls back to using \pthreads{} mutexes.

\punt{
\begin{figure}[!t]
\begin{lstlisting}
void waitToken() {
	waitFence();
	while(token_holder != thread_id) {
		yield();
	}
}
void putToken() {
	token_holder = token_queue.nextThread();
}
\end{lstlisting}
\caption{Pseudocode for token management ($\S$ \ref{sec:schedule}).
\label{fig:token}}
\end{figure}
}

A key mechanism used by \dthreads{} is its global token.
To guarantee determinism, each thread must wait for the token
before it can communicate with other threads. 
The token is a shared pointer that points to the next runnable thread entry.
Since the token is unique in the entire system, waiting for the token guarantees a global order for
all operations in the serial phase. 

\dthreads{} uses two internal subroutines to manage tokens. The \texttt{waitToken} function
first waits at the internal fence and then waits to acquire the global token
before entering serial mode. The \texttt{putToken} function passes the token to
the next waiting thread.

To guarantee determinism (see Figure~\ref{fig:phase}), threads leaving the parallel phase 
must wait at the internal fence before they can enter into the serial phase (by calling
\texttt{waitToken}).
Note that it is crucial that threads wait at the fence even for a thread which is guaranteed to obtain the token next,
since one thread's commits can affect another threads' behavior if there is no fence.

% XXX EDB FIX ME -- commented out because I do not get this.

%In the serial phase, each thread can perform those commits for parallel phase and actual synchronizations
%before they passed the token to next thread. 
%At the end of serial phase, each thread must wait at the fence before resuming execution.


%\textbf{CCC: Edited}

\subsubsection{Commit Protocol}
\label{sec:protocol}

Figure~\ref{fig:architecture} shows the steps taken by \dthreads{} to
capture modifications to shared state and expose them in a
deterministic order.  At the beginning of the parallel phase, threads
have a read-only mapping for all shared pages.  If a thread writes to
a shared page during the parallel phase, this write is trapped and
re-issued on a private copy of the shared page.  Reads go directly to
shared memory and are not trapped.  In the serial phase, threads
commit their updates one at a time.  The first thread to commit to a
page can directly copy its private copy to the shared state, but
subsequent commits must copy only the modified bytes.  \dthreads{} 
computes diffs from a twin page, an unmodified copy of the shared page created at 
the beginning of the serial phase.  At the end of the serial phase,
private copies are released and these addresses are restored to
read-only mappings of the shared memory.

\punt{
\begin{figure}[!t]
\begin{lstlisting}
void atomicBegin() {
	foreach(page in modifiedPages) {
		page.writeProtect();
		page.privateCopy.free();
	}
	modifiedPages.emptyList()
}
\end{lstlisting}
\caption{Pseudocode for \texttt{atomicBegin} ($\S$ \ref{sec:atomicbegin}).\label{fig:atomicbegin}}
\end{figure}
}

\punt{
Figure~\ref{fig:atomicbegin} presents pseudocode for
\texttt{atomicBegin}.  First, all previously-written
pages are write-protected (line 3).  The old working copies of these
pages are then discarded, and mappings are updated to reference the
shared state (line 4).
}

\label{sec:atomicbegin}

At the start of every transaction (that is, right after a
synchronization point), \dthreads{} starts by write-protecting all
previously-written pages. The old working copies of these pages are
then discarded, and mappings are then updated to reference the shared
state.

\punt{
\begin{figure}[!t]
\begin{lstlisting}
void atomicEnd() {
	foreach(page in modifiedPages) {
		if(page.writers > 1 && !page.hasTwin()) {
			page.createTwin();
		}
		if(page.version == page.localCopy.version) {
			page.copyCommit();
		} else {
			page.diffCommit();
		}
		page.writers--;
		if(page.writers == 0 && page.hasTwin()) { 
			page.twin.free();
		}
		page.version++;
	}
}
\end{lstlisting}
\caption{Pseudocode for \texttt{atomicEnd} ($\S$ \ref{sec:atomicbegin}).
\label{fig:atomicend}}
\end{figure}
}

\punt{
Figure~\ref{fig:atomicend} presents pseudocode for
\texttt{atomicEnd}.  \texttt{atomicEnd} commits all
changes from the current transaction to the shared page.  For each
modified page with more than one writer, \dthreads{} ensures that a
twin page is created (lines 3-5).  If the version number of the
private copy matches the shared page, then the current thread must be the first thread to
commit.  In this case, the entire private copy can be copied to the
shared state (lines 7 and 8).  If the version numbers do not match,
then another thread has already committed changes to the page and a
diff-based commit must be used (lines 9-10).
After changes have been committed, the number of writers to the
page is decremented (line 13), and if there are no writers left to
commit, the twin page is freed (lines 14-16).  Finally, the shared
page's version number is incremented (line 17).
}

Just before every synchronization point, \dthreads{} first waits for
the global token (see below), and then commits all changes from the
current transaction to the shared pages in order. \dthreads{} maintains one
``twin'' page (a snapshot of the original) for every modified page
with more than one writer. If the version number of the private copy
matches the shared page, then the current thread must be the first
thread to commit.  In this case, the working copy can be copied directly
to the shared state.  If the version numbers do not match, then
another thread has already committed changes to the page and a
diff-based commit must be used.

Once changes have been committed, the number of writers to the
page is decremented and the shared page's version number is incremented.
If there are no writers left to commit, the twin page is freed.

\subsection{Deterministic Synchronization}
\label{sec:synchronization}

%Comparing to Grace, \dthreads{} supports one complete synchronization
%mechanism.  Grace turns lock operations to no-ops to eliminate
%deadlocks and does not support condition variables or barriers.  The
%only supported synchronization by Grace is thread exit; Grace enforces
%a sequential semantics in thread exit.
%
\dthreads{} enforces determinism for the full range of synchronization operations in the
\pthreads{} API, including locks, condition variables, barriers and
various flavors of thread exit.

\subsubsection{Locks}
\label{sec:lock}

\dthreads{} uses a single global token to guarantee ordering and atomicity during the serial phase.  When acquiring a lock, threads must first wait for the global token.  Once a thread has the token it can attempt to acquire the lock.  If the lock is currently held, the thread must pass the token and wait until the next serial phase to acquire the lock.  It is possible for a program run with \dthreads{} to deadlock, but only for programs that can also deadlock with \pthreads{}.

% EDB: We need to acquire a token before we can execute any
% synchronization operation. We only release the token once our lock
% count is 0 (for instance, lock(A); lock(B); unlock(B); unlock(A);
% we release the token only after unlock(A).

\punt{
\begin{figure}[ht]
\begin{lstlisting}
void mutex_lock() { if(lock_count == 0) { waitToken(); atomicEnd();
	atomicBegin(); } lock_count++; }
\end{lstlisting}
\caption{Pseudocode for \texttt{mutex\_lock} ($\S$ \ref{sec:lock}).
\label{fig:lock}}
\end{figure}
}

\punt{
Figure~\ref{fig:lock} presents the pseudocode for lock acquisition.  First, \dthreads{} checks to see if the current thread is already holding any locks.  If not, the thread first waits for the token, commits changes to shared state by calling \texttt{atomicEnd}, and begins a new atomic section (lines 2-6).  Finally, the thread increments the number of locks it is currently holding.  This count must be kept to ensure that a thread will not pass the token until it has release all of the locks it acquired in the serial phase.
}

Lock acquisition proceeds as follows. First, \dthreads{} checks 
to see if the current thread is already holding any locks.  If not, the thread first waits for the token, commits changes to shared state by calling \texttt{atomicEnd}, and begins a new atomic section.  Finally, the thread increments the number of locks it is currently holding.  The lock count ensures that a thread does not pass the token on until it has released all of the locks it acquired in the serial phase.

\punt{
\begin{figure}[ht]
\begin{lstlisting}
void mutex_unlock(){
	lock_count--;
	if(lock_count == 0) {
		atomicEnd();
		putToken();
		atomicBegin();
		waitFence();
	}
}
\end{lstlisting}
\caption{Pseudocode for \texttt{mutex\_unlock} ($\S$ \ref{sec:lock}).
\label{fig:unlock}}
\end{figure}
}

\punt{
Figure~\ref{fig:unlock} presents the implementation of \texttt{pthread\_mutex\_unlock}.  First, the thread decrements its lock count (line 2).  If no more locks are held, any local modifications are committed to shared state, the token is passed, and a new atomic section is started (lines 3-6).  Finally, the thread waits on the internal fence until the start of the next round's parallel phase (line 7).
}

\texttt{pthread\_mutex\_unlock}'s implementation is similar.  First,
the thread decrements its lock count.  If no more locks are held, any
local modifications are committed to shared state, the token is
passed, and a new atomic section is started.  Finally, the thread
waits on the internal fence until the start of the next round's
parallel phase. If other locks are still held, the lock count is just
decreased and the running thread continues execution with the global
token.


\subsubsection{Condition Variables}
\label{sec:condwait}

Guaranteeing determinism for condition variables is more complex than for mutexes because the operating system does not guarantee that processes will wake up in the order they waited for a condition variable.

\punt{
\begin{figure}[ht]
\begin{lstlisting}
void cond_wait() {
	waitToken();
	atomicEnd();
	
	token_queue.removeThread(thread_id);
	live_threads--;
	cond_queue.addThread(thread_id);
	putToken();	

	while(!threadReady()) {
		real_cond_wait();
	}
	
	while(token_holder != thread_id) {
		yield();
	}
	atomicBegin();
}
\end{lstlisting}
\caption{Pseudocode for \texttt{pthread\_cond\_wait} ($\S$~\ref{sec:condwait}). 
\label{fig:condwait}}
\end{figure}
}

\punt{
Figure~\ref{fig:condwait} presents pseudocode for the \dthreads{} implementation of \texttt{pthread\_cond\_wait}.  When a thread calls \texttt{pthread\_cond\_wait}, it first waits for the token and commits local modifications (lines 2 and 3).  It removes itself from the token queue (line 4) because threads waiting on a condition variable do not participate in the serial phase until they are woken up.  The thread decrements the live thread count (used for the fence between parallel and serial phases), adds itself to the condition variable's queue, and passes the token (lines 6-8).  While threads are waiting on \dthreads{} condition variables, they are suspended on a \pthreads{} condition variable (lines 10-12).  Once a thread is woken up, it busy-waits on the token and finally begins the next transaction (lines 14-17).  Threads must acquire the token (to ensure serial execution) before proceeding because \texttt{pthread\_cond\_wait} is called within a mutex's critical section.
}

When a thread calls \texttt{pthread\_cond\_wait}, it first acquires
the token and commits local modifications.  It then removes itself
from the token queue, because threads waiting on a condition variable
do not participate in the serial phase until they are awakened.  The
thread decrements the live thread count (used for the fence between
parallel and serial phases), adds itself to the condition variable's
queue, and passes the token.  While threads are waiting on \dthreads{}
condition variables, they are suspended on a \pthreads{} condition
variable.  When a thread is awakened (signalled), it busy-waits on the
token before beginning the next transaction.  Threads must
acquire the token before proceeding
because the condition variable wait function must be called within a mutex's
critical section.

\label{sec:condsignal}

\punt{
\begin{figure}[ht]
\begin{lstlisting}
void cond_signal() {
	if(token_holder != thread_id) {
		waitToken();
	}
	atomicEnd();
	
	if(cond_queue.length == 0) {
		return;
	}
	
	lock();
	thread = cond_queue.removeNext();
	token_queue.insert(thread);
	live_threads++;
	thread.setReady(true);
	real_cond_signal();
	unlock();
	atomicBegin();
}
\end{lstlisting}
\caption{Pseudocode for \texttt{pthread\_cond\_signal} ($\S$~\ref{sec:condsignal}). 
\label{fig:condsignal}}
\end{figure}
}

\punt{
The \dthreads{} implementation of \texttt{pthread\_cond\_signal} is presented in Figure~\ref{fig:condsignal}.  The calling thread first waits for the token, and then commits any local modifications (lines 2-5).  If no threads are waiting on the condition variable, this function returns immediately (lines 7-9).  Otherwise, the first thread in the condition variable queue is moved to the head of the token queue and the live thread count is incremented (lines 12-14).  This thread is then marked as ready and woken up from the real condition variable, and the calling thread begins another transaction (lines 15-18).
}

In the \dthreads{} implementation of \texttt{pthread\_cond\_signal},
the calling thread first waits for the token, and then commits any
local modifications.  If no threads are waiting on the condition
variable, this function returns immediately.  Otherwise, the first
thread in the condition variable queue is moved to the head of the
token queue and the live thread count is incremented.  This thread is
then marked as ready and woken up from the real condition variable,
and the calling thread begins another transaction.

To impose an order on signal wakeup, \dthreads{} signals
actually call 
\texttt{pthread\_cond\_broadcast} to wake all waiting threads, but then marks only
the logically next one as ready.  The threads not marked as
ready will wait on the condition variable again.

\subsubsection{Barriers}

\label{sec:barrierwait}

As with condition variables, \dthreads{} must ensure that threads
waiting on a barrier do not disrupt token passing among running
threads. \dthreads{} removes threads entering into the barrier from
the token queue and places them on the corresponding barrier queue.

In \texttt{pthread\_barrier\_wait}, the calling thread first waits for
the token to commit any local modifications. If the current thread is
the last to enter the barrier, then \dthreads{} moves the entire list
of threads on the barrier queue to the token queue, increases the
live thread count, and passes the token to the first thread in the
barrier queue.  Otherwise, \dthreads{} removes the current thread from
the token queue, places it on the barrier queue, and releases
token. Finally, the thread waits on the actual \pthreads{} barrier.

\punt{
\begin{figure}[ht]
\begin{lstlisting}
void barrier_wait() {
	waitToken();
	atomicEnd();
	lock();
	if(barrier_queue.length == barrier_count-1) {
		token_holder = barrier_queue.first();
		live_threads += barrier_queue.length;
		barrier_queue.moveAllTo(token_queue);
	} else {
		token_queue.remove(thread_id);
		barrier_queue.insert(thread_id);
		putToken();
	}
	unlock();
	atomicBegin();
	real_barrier_wait();
}
\end{lstlisting}
\caption{Pseudocode for barrier\_wait ($\S$~\ref{sec:barrierwait}).
\label{fig:barrierwait}}
\end{figure}
}

\punt{
Figure~\ref{fig:barrierwait} presents pseudocode
for \texttt{pthread\_barrier\_wait}. The calling thread first waits for the
token to commit any local modifications in order to ensure
deterministic commit (lines 2 and 3). If the current thread is the
last to enter the barrier, then \dthreads{} moves the entire list of
threads on the barrier queue to the token queue (line 7), increases
the fence's thread count (line 8), and passes the token to the first thread in the
barrier queue (line 9).  Otherwise, \dthreads{} removes the current
thread from the token queue (line 12), places it on the barrier queue
(line 13), and releases token (line 14). Finally, the thread waits on
the actual barrier (line 19).
}

\subsubsection{Thread Creation and Exit}

\label{sec:threadcreation}

To guarantee determinism, thread creation and exit are performed
in the serial phase.  Newly-created threads are added to
the token queue immediately after the parent thread.  Creating a thread does not  release the token; this approach allows a single thread to quickly create multiple child threads without having to wait for a new serial phase for each child thread.

\punt{
\begin{figure}[ht]
\begin{lstlisting}
void thread_create() {
	waitToken();
	clone(CLONE_FS | CLONE_FILES | CLONE_CHILD);
	if(child_process) {
		thread_id = next_thread_index;
		next_thread_index++;
		notifyChildRegistered();
		waitParentProadcast();
	} else {
		waitChildRegistered();
	}
}
\end{lstlisting}
\caption{Pseudocode for \texttt{thread\_create} ($\S$ \ref{sec:threadcreation}).
\label{fig:threadcreate}}
\end{figure}
}

\punt{
Figure~\ref{fig:threadcreate} presents pseudocode for thread
creation. The caller first waits for the token before proceeding (line
2).  It then creates a new process with shared file descriptors but a
distinct address space using the \texttt{clone} system call (line 3).
The newly created child obtains the global thread index (line 5),
places itself in the token queue (line 6), and notifies the parent
that child has registered itself in the active list (line 7). The
child thread then waits for the parent to reach a synchronization
point.
}

When creating a thread, the parent first waits for the token. It then
creates a new process with shared file descriptors but a distinct
address space using the \texttt{clone} system call.  The newly created
child obtains the global thread index, places itself in the token
queue, and notifies the parent that the child has registered itself in the
active list. The child thread then waits for the next parallel phase before proceeding.

\punt{
\begin{figure}[ht]
\begin{lstlisting}
void thread_exit() {
	waitToken();
	atomicEnd();
	token_queue.remove(thread_id);
	live_threads--;
	putToken();
	real_thread_exit();
}
\end{lstlisting}
\caption{Pseudocode for \texttt{thread\_exit} ($\S$ \ref{sec:threadcreation}).
\label{fig:threadexit}}
\end{figure}
}

% \textbf{CCC: Edited}

\punt{
Figure~\ref{fig:threadexit} presents pseudocode for \texttt{thread\_exit}.
When this function is called, the caller first waits for the
token and then commits any local modifications (line 3). It then
removes itself from the token queue (line 4) and decreases the number
of threads required to proceed to the next phase (line 5). Finally,
the thread passes its token to the next thread in the token queue
(line 6) and exits (line 7).
}

Similarly, \dthreads{}' \texttt{pthread\_exit} first waits for the
token and then commits any local modifications to memory. It then
removes itself from the token queue and decreases the number
of threads required to proceed to the next phase. Finally,
the thread passes its token to the next thread in the token queue
 and exits.

%% \vfill %%% EDB

\subsubsection{Thread Cancellation}

\dthreads{} implements thread cancellation in the serial
phase. A thread can only invoke \texttt{pthread\_cancel} while holding the
token.  If the thread being cancelled is waiting on a condition
variable or barrier, it is removed from the queue. Finally, to cancel
the corresponding thread, \dthreads{} kills the target process
with a call to \texttt{kill(tid, SIGKILL)}.


\subsection{Deterministic Memory Allocation}

Programs sometimes rely on the addresses of objects returned by the memory
allocator intentionally (for example, by hashing objects based on
their addresses), or accidentally. A program with a memory error like
a buffer overflow will yield different results for different memory
layouts.

This reliance on memory addresses can undermine other efforts to
provide determinism. For example, CoreDet is unable to fully enforce
determinism because it relies on the Hoard scalable memory
allocator~\cite{BergerMcKinleyBlumofeWilson:ASPLOS2000}. Hoard was not
designed to provide determinism and several of its mechanisms, thread
id based hashing and non-deterministic assignment of memory to
threads, lead to non-deterministic execution in CoreDet for
the \texttt{canneal} benchmark.

To preserve determinism in the face of intentional or inadvertent
reliance on memory addresses, we designed the \dthreads{} memory allocator
to be fully deterministic. \dthreads{} assigns subheaps to each thread
based on its thread index (deterministically assigned; see
Section~\ref{sec:threadindex}). In addition to guaranteeing the same
mapping of threads to subheaps on repeated executions, \dthreads{}
allocates superblocks (large chunks of memory) deterministically by
acquiring a lock (and the global token) on each superblock
allocation. Thus, threads always use the same subheaps, and these
subheaps always contain the same superblocks on each execution. The
remainder of the memory allocator is entirely deterministic. The
superblocks themselves are allocated via \mmap{}: while \dthreads{}
could use a fixed address mapping for the heap, we currently simply disable
ASLR to provide deterministic \mmap{} calls.  If a program does not use the
absolute address of any heap object, \dthreads{} can guarantee determinism even
with ASLR enabled.  Hash functions and lock-free algorithms frequently use
absolute addresses, and any deterministic multithreading system must disable
ASLR to provide deterministic results for these cases.

\subsection{OS Support}

\dthreads{} provides shims for a number of system calls both for
correctness and determinism (although it does not enforce
deterministic arrival of I/O events; see
Section~\ref{sec:discussion}).

System calls that write to or read from buffers on the heap (such
as \texttt{read} and \texttt{write}) will fail if the buffers contain
protected pages. \dthreads{} intercepts these calls and touches each
page passed in as an argument to trigger the copy-on-write operation before issuing the real system
call. \dthreads{} conservatively marks all of these pages as modified
so that any updates made by the system will be committed properly.

\dthreads{} also intercepts other system calls that affect
program execution. For example, when a thread
calls \texttt{sigwait}, \dthreads{} behaves much like it does for
condition variables. It removes the calling thread from the token
queue before issuing the system call, and after being awakened the
thread must re-insert itself into the token queue and wait for the token before proceeding.



\begin{figure*}[!t]
{\centering
\includegraphics[width=5in]{fig/overhead-figure}
\caption{Normalized execution time with respect to \pthreads{} (lower is better). For 9 of the 14 benchmarks, \dthreads{} runs nearly as fast or faster than \pthreads{}, while providing deterministic behavior.\label{fig:performance}}
}
\end{figure*}

\section{Optimizations}
\label{sec:dthreads-optimizations}

\dthreads{} employs a number of optimizations that improve its performance.

\textbf{Lazy commit:}
\dthreads{} reduces copying overhead and the time spent in the serial
phase by \emph{lazily} committing pages. When only one thread has ever
modified a page, \dthreads{} considers that thread to be the page's
owner.  An owned page is committed to shared state only when another
thread attempts to read or write this page, or when the owning thread attempts to modify it in a later phase. \dthreads{} tracks reads with page protection and signals
the owning thread to commit pages on demand. To reduce the number of
read faults, pages holding global variables (which we expect to be
shared) and any pages in the heap that have ever had multiple writers are
all considered unowned and are not read-protected.

\textbf{Lazy twin creation and diff elimination: }
To further reduce copying and memory overhead, a twin page is only created
when a page has multiple writers during the same transaction.
In the commit phase, a single writer can directly copy its working copy to
shared state without performing a diff. \dthreads{} does this by
comparing the local version number to the global page version number
for each dirtied page.  At commit time, \dthreads{} directly copies
its working copy for each page whenever its local version number
equals its global version number.  This optimization saves the cost of a twin 
page allocation, a page copy, and a diff in the common case where just one thread is the sole writer of a page.

\textbf{Single-threaded execution: }
Whenever only one thread is running, \dthreads{} stops using memory protection 
and treats certain synchronization operations (locks and barriers) as no-ops.
In addition, when all other threads are waiting on condition variables, 
\dthreads{} does not commit local changes to the shared mapping or discard  
private dirty pages. Updates are only committed when the thread
issues a signal or broadcast call, which wakes up at least one thread
and thus requires that all updates be committed.

\textbf{Lock ownership: }
\dthreads{} uses lock ownership to avoid unnecessary waiting when threads are
using distinct locks.  Initially, all locks are unowned.  Any thread that
attempts to acquire a lock that it does not own must wait until the serial phase
to do so.  If multiple threads attempt to acquire the same lock, this lock is 
marked as shared.  If only one thread attempts to acquire the lock, this thread 
takes ownership of the lock and can acquire and release it during the parallel
phase.

Lock ownership can result in starvation if one thread continues to re-acquire an owned lock without entering the serial phase.  To avoid this, each lock has a maximum number of times it can be acquired during a parallel phase before a serial phase is required.

\textbf{Parallelization: }
\dthreads{} attempts to expose as much parallelism as possible in the
runtime system itself.  One optimization takes place at the start of trasactions, where
\dthreads{} performs a variety of cleanup tasks. These include releasing private page frames,
and resetting pages to read-only mode by calling the \texttt{madvise}
and \texttt{mprotect} system calls. If all this cleanup work is done
simultaneously for all threads in the beginning of parallel phase
(Figure~\ref{fig:phase}), this can hurt performance for some
benchmarks.

Since these operations do not affect other the behavior of other
threads, most of this work can be parallelized with other threads'
commit operations without holding the global token. With this
optimization, the token is passed to the next thread as soon as
possible, saving time in the serial phase.  Before passing the token,
any local copies of pages that have been modified by other threads
must be discarded, and the shared read-only mapping is restored.  This
ensures all threads have a complete image of this page which later
transactions may refer to.  In the actual
implementation, this cleanup occurs at the end of each transaction.



\section{Evaluation}
\label{sec:evaluation}

% CCC: overhead figure moved to main.tex to place on the same page as evaluation section

\begin{figure*}
{\centering
\includegraphics[width=5in]{fig/scalability-figure}
\caption{
	Speedup with four and eight cores relative to two cores (higher is better).  \dthreads{} generally scales nearly as well or better than \pthreads{} and almost always as well or better than CoreDet.  CoreDet was unable to run \texttt{dedup} with two cores and \texttt{ferret} with four cores, so some scalability numbers are missing.\label{fig:scalability}}
}
\end{figure*}
\label{sec:evaluation}

All evaluations are performed on a quiescent Intel Core 2 dual-processor system equipped with 
16GB RAM. 
Each processor is a 4-core 64-bit Intel Xeon running at 2.33 GHz with a 4MB
shared L2 cache and 32KB private L1 data cache. 
The underlying operating system is unmodified CentOS 5.5, running with Linux kernel
version 2.6.18-194.17.1.el5. The glibc version is 2.5. 
In order to compare the performance fairly, all benchmarks were built as 64-bit executables 
using LLVM compiler (version 3.2). The compiler optimization level is set to ``-O1'' 
so memory allocation callsites can be reported precisely.
%since we can not report line number of source code with optimization level larger than
%``-O2''.
In our evaluations of this section, 
we choose two popular benchmark suites, Phoenix~\cite{phoenix-hpca} 
and PARSEC~\cite{parsec}. 
Some programs of PARSEC cannot be compiled or run succesfully with LLVM are excluded here.
For example, \texttt{facesim} can't be compiled 
because \texttt{llvm} forces a much stricter C++ rule. 
\texttt{canneal} can't be run successfully even it is compiled by original \texttt{llvm} compiler.

In this section, our evaluations aim to answer the following questions:
\begin{itemize}
\item
  How effective is \Predator{} on detecting and predicting false sharing problem (Section ~\ref{sec:effective})?

\item
  What is the performance overhead of \Predator{} with and without prediction
  (Section ~\ref{sec:perfoverhead})?

\item
  What is the memory overhead of \Predator{}~ (Section~\ref{sec:memoverhead})?
\end{itemize}


\subsection{Detection and Prediction Effectiveness}
\label{sec:effective}

\subsubsection{Benchmarks}
\label{sec:benchmarks}
Results on two benchmark suites, Phoenix and PARSEC, 
are listed in the Table~\ref{table:detection}. 

%Our results show that \Predator{} not only capture previously-discovered
%false sharing, but also many detect new false sharing places. The results
%are listed in Table~\ref{table:detection}. 

%http://www.technovelty.org/tips/getting-a-tick-in-latex.html
%http://tex.stackexchange.com/questions/42619/x-mark-to-match-checkmark
%\begin{comment}
\begin{table*}[ht!]
{
\centering
\begin{tabular}{l|r|r|r}
\hline
{\bf \small Benchmark} & {\bf \small Source Code} & {\bf \small New} & {\bf \small Improvement} \\
%{\bf \small Benchmark} & {\bf \small Source Code} & {\bf \small Type of False Sharing} & {New} & {\bf \small Improvement} \\
\hline
\small \textbf{histogram} & {\small histogram-pthread.c:213} & \cmark{} & 46.22\%\\
\small \textbf{reverse\_index} & {\small reverseindex-pthread.c:511} & \xmark{} & 0.09\%\\
\small \textbf{word\_count} & {\small word\_count-pthread.c:136} & \xmark{} & 0.14\%\\
\hline
\small \textbf{streamcluster} & {\small streamcluster.cpp:985} & \xmark{} & 7.52\% \\
\small \textbf{streamcluster} & {\small streamcluster.cpp:1907} & \cmark{} & 4.77\%\\
%\small \textbf{bodytrack} & {\small TrackingModel.cpp:59} & 0 & \cmark{} & \\
\hline
\hline
\small \textbf{linear\_regression} & {\small linear\_regression-pthread.c:133} & \xmark{} & 1206.93\%\\
\hline
\end{tabular}
\caption{Detection results of \Predator{} on Phoenix and PARSEC benchmark suites. \label{table:detection}}
}
\end{table*}

In this table, the first column lists those programs with false sharing problems. 
The second column shows precisely where the problem is. Since all found false sharing 
occurs inside the same heap object, we listed callsite source code information in this column.
The third column ``New'' marks whether this false sharing is newly found by \Predator{} or not.
False sharing problems found by previous work are marked with a crossmark(\xmark{}) and those 
newly found ones are marked with a tick mark(\cmark{}). 
The last column ``Improvement'' shows the performance improvement after fixing false sharing. 
The number is based on the average runtime of $10$ runs. 

\begin{comment}
\textbf{Tongping: how to say the following sentences}\\
The improvement rate is calculated by substracting $1$ from normalized runtime of orignal
runtime and new runtime. 
Taking \texttt{histogram} for an example here, original runtime of \texttt{histogram} is $0.75s$
and new runtime is $0.51s$, then the performance improvement is $(0.75/0.51) - 1$.
\end{comment}

As shown in the table, \Predator{} reveals several unknown false sharing problems. 
It is the first tool to detect two false sharing problems: one in \texttt{histogram} 
and one in line $1908$ of \texttt{streamcluster}. 
In \texttt{histogram}, multiple threads repeatedly modify different locations of the same heap object. 
Padding the data structure \texttt{thread\_arg\_t} fixes the false sharing problem and 
helps to improve the performance around 46\%.
In \texttt{streamcluster}, multiple threads are simultaneously accessing and updating 
the same \texttt{bool} array, \texttt{switch\_membership}. 
By simply changing this array to \texttt{long} type contributes to about 4.7\% performance improvement.

%, although it is not a complete fix of false sharing. 
%None of these two false sharing problems has been reported by previous tools.
All other false sharing problems have been discovered by one of previous tools, 
\sheriff{}~\cite{sheriff}. 
Same as \sheriff{}, we do not see much performance improvement for \texttt{reverse\_index} and 
\texttt{word\_count} since the number of updates inside them is not significantly large. But they
are actual false sharing problems that have been verified manually by us.

\texttt{streamcluster} has another false sharing problem at line $985$. 
Different threads change the same object (\texttt{work\_mem}) simultaneously. 
Authors of \texttt{streamclsuter} have already realized possible
false shairing problems and meant to utilize a macro \texttt{CACHE\_LINE} to avoid it. Unfortunately,
the defaulted value of this macro is setted to $32$ bytes, which is different with the actual
cache line size of our hardware. By setting to $64$ bytes instead, we achieve around $7.5\%$ performance
improvement.

\texttt{linear\_regression} has a severe false sharing problem, 
while fixing it improves the performance more than $12\times$.
In this benchmark, different threads constantly update their thread specific locations 
inside a heap object(\texttt{tid\_args}), causing a huge amount of cache invalidations. 
As pointed out by Nanavati et al.~\cite{OSdetection}, false sharing 
occurs when using \texttt{clang} compiler and disappears when using \texttt{GCC} with optimization
-O2 and -O3.  
During our evaluation, because our customized memory manager has different allocation 
metadata for each heap object, this false sharing do not 
occur at all when we are compiling this program
using \texttt{clang-3.2} at ``-O1'' optmization level.
Detailed reason of this can be seen in Section~\ref{sec:predicteval}.
Thus, \Predator{} actually can not detect this false sharing problem without enabling 
prediction mechanism. Also, none of existing tools can detect false sharing without occurence.
Using the prediction mechanism discussed in Section~\ref{sec:prediction}, 
\Predator{} can always capture false sharing problems in this benchmark.
This also examplify the importance of prediction tool. 

\subsubsection{Real Applications}
To verify its practicability, we further evaluate \Predator{} 
on several widely-used real applications, which none of previous work has evaluated.  
These real applications include a server application \texttt{MySQL}~\cite{mysql}, 
a common C++ library \texttt{boost}~\cite{libfalsesharing} 
and a distributed memory object caching system \texttt{memcached}, a network retriver \texttt{aget}, 
a parallel bzip2 file compressor \texttt{pbzip2} and a parallel file scanner \texttt{pfscan}.
For \texttt{MySQL} and \texttt{boost},
we evaluate their specific versions, \texttt{MySQL-5.5.32} and
\texttt{boost-1.49.0}, which are known to have some false sharing problems.

The false sharing problem in \texttt{MySQL} has caused significant scalability problem and
it was very difficult to be identified. 
According to the architect of \texttt{MySQL} Mikael Ronstrom, ``we had gathered specialists on 
InnoDB..., participants from MySQL support... and a number of generic specialists on 
computer performance...'', ``the fruit of the meeting ... were able to 
improve \texttt{MySQL} performance by 6$\times$ with those scalability fixes''. 
The false sharing of boost library is caused by the special usage of \texttt{spinlock} pool and fixing
it brings 40\% performance improvement. 
\Predator{} is able to succesfully detect false sharing locations
in both \texttt{MySQL} and \texttt{boost} library. 
For the other four applications, \Predator{} doest not find serere false sharing problems.

\subsubsection{Prediction Effectiveness}
\label{sec:predicteval}
For prediction effectiveness, we evaluate whether \Predator{} can always capture a false sharing
problem without occurrence.
\texttt{linear\_regression} benchmark is selected here because of the following two reasons:
\begin{enumerate}
\item
False sharing problem of this benchmark can not be detected without prediction, see Section~\ref{sec:benchmarks}. 

\item
False sharing severely degrades performance when it actually occurs. 
Hence, it is a serious problem that should be always detected. 
\end{enumerate}

\begin{figure}[!h]
{\centering
\subfigure{\lstinputlisting[numbers=none,frame=none,boxpos=t]{fig/linearregression.psedocode}}
\caption{False sharing problem inside \texttt{linear\_regression} benchmark.
\label{fig:linearregression}}
}
\end{figure}

Figure~\ref{fig:linearregression} shows the data structure and source code
experiencing false sharing.
The size of this data structure, \texttt{lreg\_args}, is $64$ bytes 
when we use \texttt{clang} compiler to compile $64$bit binary at ``-O1'' optimization level.
For this false sharing problem, the main thread allocates an array with the number of elements equals
to that  of underlying hardware cores.
Each element is a \texttt{lreg\_args} type with $64$ bytes. 
Then this array is passed to different threads (\texttt{lreg\_thread}) 
so that each thread only updates its thread-dependent area, see Figure~\ref{fig:linearregression}.
False sharing occurs if two threads happens to update data in the same cache line. 
However, different fields of \texttt{lreg\_args} has different access pattern:
only those fields between $SX$ and $SXY$ (totally around $40$ bytes) are constantly read and updated.
Consequently, the performance of \texttt{linear\_regression} is very sensitive to 
the starting address of false sharing object (see Figure~\ref{fig:perfsensitive}),
which can be changed by many dynamic properties according
to the discussion in Section~\ref{sec:intro}.

Figure~\ref{fig:perfsensitive} shows performance sensitivity to 
offsets of the starting address between the false sharing object and corresponding cache lines. 
When the offset is $0$ or $56$ bytes, this benchmark achieves its optimal performance 
and has no false sharing at all.
When the offset is $24$ bytes, this benchmarks runs around $15$ times slower 
than its optimal performance.
When we eveluate detection effectiveness on its original code, 
our customized memory manager happens to make the offset $56$ bytes. 
As a result, \Predator{} can not detect false sharing in this benchmark 
without enabling prediction because of no occurrence of false sharing problem.
This situation happens to all existing tools: they can only detect false sharing problems when
they occur. 

In contrast, the prediction mechanism designed in \predator{} 
amis to address this problem.  Results of our evaluations shows 
that \Predator{} can always predict the false sharing problem in this
benchmark no matter what the offset value is. 
This explains the effectiveness of \Predator{}.

\subsection{Performance Overhead}
\label{sec:perfoverhead}

\begin{figure*}[!ht]
\begin{center}
\includegraphics[width=6.5in]{fig/perf}
\end{center}
\caption{
Performance overhead of \Predator{} with and without prediction.
\label{fig:perf}}
\end{figure*}

To avoid the effect caused by extreme outliers, all performance data shown in this section
are based on the average result of $10$ runs while excluding the maximum and minimum values.
Actual performance overhead with and without prediction 
can be seen in the following figure~\ref{fig:perf}. 

From this figure, we can see for all $16$ benchmarks from Phoenix and PARSEC
benchmark suites that \Predator{} with prediction imposes around $6.7\times$
performance overhead. 
If we remove prediction from \Predator{}, we cannot observe significant performance difference.
This means that prediction of \Predator{} only introduces very minimum performance overhead. 

Among these programs, five of them have more than $8\times$ performance overhead, 
including \texttt{histogram, kmeans, bodytrack, ferret} and \texttt{swaptions}. 
Program \texttt{histogram} has the most performance overhead and 
runs more than $26$ slower than original executions. 
It has a severe false sharing problem inside, and tracking detailed access for those
problematic cache lines exacerbates the 
false sharing effect (see more discussion on this in Section~\ref{sec:sample}). 
For \texttt{bodytrack} and \texttt{ferret}, \Predator{} found a large amount of cache lines with 
writes larger than {\it Tracking-Threshold}. 
Tracking all accesses details for those cache lines 
imposes significant performance overhead. 
Currently, we have not identified the reasons 
why \texttt{kmeans} runs much slower in \Predator{}.   

In our evaluation, we do not observe signficant performance overhead on 
\texttt{matrix\_multiply, blackscholes} and 
\texttt{x264}.
Possibly a large portion of computations operates on stack variables, which are
not tracked by \Predator{}. 

\subsection{Memory Overhead}
\label{sec:memoverhead}
Since \Predator{} pre-allocates a huge block of memory ({\it virtual memory}) 
using \texttt{mmap} system call for its heap usage, 
virtual memory can not be used to evaluate actual memory overhead imposed by our tool. 
Hence, we only evaluate physical memory used for each application. 
According to the discussion of Justin et al. ~\cite{memusage}, proportional set size (PSS) 
in \texttt{/proc/self/smaps} is a suitable number since it reflects memory increase to the system
by running this application. 

To get PSS data, we start a script program to save 
corresponding \texttt{smaps} files periodically.
For each \texttt{smaps} file, we calculate the sum of PSSs for different
memory mappings and uses it as total physical memory usage for this application.
Among all collected \texttt{smaps} files, we choose the maximum number of
different files for comparison since it represents the maximum memory overhead to run this application.
%It is noted that we remove the physical memory usage of   
Results of maximum memory usage is shown in Figure~\ref{fig:memusage}. As we can see,
\Predator{} does not introduce substantial memory usage overhead 
for all eveluated bencharmks, except for \texttt{swaptions}. 
Removing \texttt{swaptions} from comparisons reduces 
the average memory overhead from 64\% to 22\%. 

The reason why \texttt{swaptions} introduces $7.8\times$ memory overhead is that 
its original memory usage is too small (only $3KB$).
Adding the code of detection, prediction and
reporting contributes to a large portion of memory overhead. 

\begin{figure*}
\begin{center} 
\includegraphics[width=6.5in]{fig/memusage}
\end{center}
%\includegraphics{fig/potential.pdf}
\caption{Memory usage overhead}
\label{fig:memusage}
\end{figure*}




\section{Discussion}
\subsection{Limitations}
\doubletake{} can detect those heap buffer overflows overwritting ``guard zones'', 
either caused by direct memory access or caused by
incorrect library calls (e.g. \texttt{memcpy} or \texttt{strcpy}). 
\doubletake{} can detect most of heap underflows: they can be detected if they are
aligned by \texttt{power of 2} or they are freed in the end of program. 
For those objects that they are not freed and not aligned, \doubletake{} can not detect them.
 
\doubletake{} can not detect buffer overflows on the globals and the stack 
since \doubletake{} can not change the layout of globals by inserting guard zones around them.
AddressSanitizer is a complementary approach for this: we can rely on AddressSanitizer 
to put guard zones for globals, then \doubletake{} can check the overflow of the globals accumulatively. 

\doubletake{} can not detect the usage-after-free memory error, which already implemented by 
AddressSanitizer. But their mechanism can be implemented in our framework easily and provide
this detection probability. 

\textbf{We need multiple runs for programs with race conditions}.
Although the idea of using re-execution and watch point can be applied to multithreaded programs,
\doubletake{} has not targeted for multithreaded programs currently. 

\doubletake{} can not detect those non-continguous buffer overflows. 
\subsection{Future Work} 
The first possible work of \doubletake{} is to extend it on multithreaded programs because
multithreaded programs is universal these days. 
However, there are some challenges to do this: how we can repeat the execution of 
a multithreaded program given random memory accesses from different threads;
how to handle those synchronizations without violating the shared-memory
semantics. 

Another possible work is to extend \doubletake{} to find more memory errors, such as dangling pointer 
errors, memory usage-after-free errors.
We possibly have to change the memory allocator in order to achieve memory usage-after-free errors
since we are utilizing the first words of freed memory blocks in the free list management. 

The last but not least, we can combine the \doubletake{} mechanism and \texttt{gdb} mechanism
to provide an interactive debugging system without actually using \texttt{gdb}. It can invoke this 
interactive debugging mechanism whenever bugs are found. 


\punt{
\section{Future Work}
\label{sec:future-work}

We are investigating ways to enhance the performance and determinism
guarantees of \dthreads{}.

While \dthreads{} ensures full \emph{internal} determinism, it does
not currently guarantee determinism if the application is sensitive
to \emph{external non-determinism}, such as the latency of network
connections or time of day. We are examining the use of a shim layer
to enforce deterministic delivery of such events in order to achieve
the same kind of external determinism guarantees provided by
dOS~\cite{deterministic-process-groups}, although without changes to
the underlying operating systems. We also plan to examine
whether \dthreads{} can be used directly in conjunction with dOS.

The key performance problem for \dthreads{} is when an application
modifies an extremely large number of pages. For those
benchmarks, \dthreads{} can incur high overheads. We plan to
incorporate an ownership protocol to limit page copying and thus
improve performance. In many cases, these pages are ``owned'' (read
and modified) by a single thread. Tracking these pages would
allow \dthreads{} to avoid copying modified private pages to the
shared memory space.


% eliminate global lock
% possibly adopting a page-ownership protocol as used by ...
}

\section{Conclusion}
\label{sec:conclusion}
This paper presents Sheriff, one software-only system to precisely detect false sharing problems
in multithreaded programs. 
Sheriff's key contribution is its runtime system framework, which simulating the running 
of threads using process. 
Sheriff uses the process's complete separation mechanism and page fault mechanism to capture writes
in one transaction by combining with twin page mechanism. 
In order to capture continuous writes in the same transaction, 
Sheriff introduce the sampling mechanism for those long-running
transactions. 
Unlike previous tools, Sheriff intercepts those memory allocation and de-allocation function calls so that
those false positives caused by re-use of dynamic objects can be avoided completely in Sheriff and also
Sheriff can attach callsite on those memory allocations, which can help to report those callsite 
information about one false sharing objects and save manual effort to locate the false sharing problems.

Sheriff can use preload attributes of library, which can work on unaltered binary files and makes the deployment
simple. Since Sheriff don't use the advanced hardware support, which can be used to find false sharing problems
for those legacy applications running on original hardware. 
For most of applications, sheriff only incurs reasonable runtime overhead, making it a 
practical choice even for those complex system which should run long time. 


\section{Acknowledgements}
The authors thank Robert Grimm, Sam Guyer, Shan Lu, Tom Bergan, Daan
Leijen, Dan Grossman, Yannis Smaragdakis, the anonymous reviewers, and our
shepherd Steven Hand for their invaluable feedback and suggestions which helped
improve this paper. We acknowledge the support of the Gigascale
Systems Research Center, one of six research centers funded under the
Focus Center Research Program (FCRP), a Semiconductor Research
Corporation entity.  This material is based upon work supported by
Intel, Microsoft Research, and the National Science Foundation under
CCF-1012195 and CCF-0910883. Any opinions, findings, and conclusions
or recommendations expressed in this material are those of the
author(s) and do not necessarily reflect the views of the National
Science Foundation.

{
\bibliographystyle{abbrv}
\bibliography{dthreads,emery}
}

\end{document}
