\date{\vspace*{-0.2in}}

%%%\renewcommand{\baselinestretch}{2}
\documentclass[10pt]{sigplanconf}
\nocaptionrule


\newcommand{\footnotenonumber}[1]{{\def\thempfn{}\footnotetext{\small #1}}}
\usepackage[normalem]{ulem}
\usepackage{graphicx}
\usepackage{times}
\usepackage{subfigure}
\usepackage{url}
\urlstyle{rm}

\usepackage{color}
\usepackage{listings}
\usepackage{amsmath}
\usepackage{amsfonts}
\usepackage{amssymb}
\usepackage{comment} 
\usepackage{setspace}
\singlespacing
%\onehalfspacing
\newtheorem{thm}{Theorem}
\newtheorem{prop}[thm]{Proposition}
\newtheorem{cor}[thm]{Corollary}
\newtheorem{lem}[thm]{Lemma}
\newtheorem{defn}[thm]{Definition}


\newcommand{\cfunction}[1]{{\bf \tt #1}}
\newcommand{\malloc}{\cfunction{malloc}}
\newcommand{\realloc}{\cfunction{realloc}}
\newcommand{\free}{\cfunction{free}}
\newcommand{\madvise}{\cfunction{madvise}}
\newcommand{\brk}{\cfunction{brk}}
\newcommand{\sbrk}{\cfunction{sbrk}}
\newcommand{\mmap}{\cfunction{mmap}}
\newcommand{\munmap}{\cfunction{munmap}}
\newcommand{\mprotect}{\cfunction{mprotect}}
\newcommand{\mlock}{\cfunction{mlock}}

\hyphenation{app-li-ca-tion}
\hyphenation{Die-Hard}
\hyphenation{Archi-pe-la-go}
\hyphenation{buf-fer}

\lstset{language=c++, basicstyle=\small\ttfamily,frame=single,tabsize=4}

\definecolor{Gray}{cmyk}{0,0,0,0.5}

\begin{document}

%\renewcommand{\baselinestretch}{2}

\conferenceinfo{XXXXXXXXXXXXXXXXXXX}

\title{Sheriff: Detecting and Eliminating False Sharing}

\authorinfo{}

%Tongping~Liu \and Emery~D.~Berger}
%{Dept.\ of Computer Science \\ University of Massachusetts, Amherst \\
%Amherst, MA 01003} {\{tonyliu,emery\}@cs.umass.edu}

\maketitle

\begin{comment}
%%%%%%%%%%%%%%%%%%%%%%%%%%%%%%%%%%%%%%%%%%%%%%%%%%%%%%%%%%%%%%%%%%%%%%%%%%%%%%%%%%%%%%%%%%%%%
Story:
Multi-core processors or NUMA are now widely used in order to avoid the physical limits of hardware. 
Multi-threaded is considered as one way to take full advantage of those computation resources. Unfornately, 
writing efficient multi-threaded program is not an easy task; 
false sharing problem can reduce the performance greatly, 
even worse, one program with serious false sharing problem can run slower on multi-core machine that that on 
on single-core machine with the same cpu frequency of every core.

This paper presents Sheriff, a system to detect those false sharing problem in multi-threaded c/c++ programs. 
Comparing to previous tool, Sheriff has a very low performance overhead, 
averagely the performance overhead is about 15\% in our experiments. 
For most programs, Sheriff can run almost the same speed or even faster than original program
using pthreads library. 
The second advantage of Sheriff is that there is no false positives at all. 
The false sharing problems reported by Sheriff are actual false sharing problems.

The third adanvatage of Sheriff is that Sheriff can pinpoint the code to cause the false sharing problems. 
For heap objects, Sheriff can point out the allocation site, 
then it is easy for programmer to fix the false sharing problem given the callsite information,
even for some one that are not familiar with the code. 
For global objects, Sheriff can point out those objects' name and length information which has false sharing problem.

How Sheriff to do that?
First, we are using a runtime system which maintains the semantics of multi-threaded program.
To detect those modifications by different threads, we turn those multi-threaded program into a 
multi-process program and use the page protection mechanism to capture those writes on different threads.
In order to capture those writes of different memory, we introduce a "twin" page in the page fault handler;
then we utilize the lazy differentiating mechanism (which has been widely used in distributed share memory) to 
find acutal writes in each phase. 
In order to capture those cache invalidation of different threads, we maintain a global array which is used to
maintain the last thread id to write on one cache line. We use a 
conservative mechanism to count those invalidation of cache lines, 
only those interleaving writes by different threads are counted as an invalidation of corresponding cache line. 
In order to differentiate the false sharing problem from true sharing problem, 
we also keep a word-based version number array 
which is used to record every writes information on each word, including thread id and version. (We use a special
id to identify multi-threads on one word).
 
Also, this word-based versioning mechanism
can help us to the precisely locate the object that are causing the problem if multiple objects are located in 
the same physical cache line.

In order to catch those false sharing problem caused by long transaction, 
we introduce one "sampling mechanism" which can get continutive writes of different threads.

In order to pingpoint the allocation site, we will attach the callsite information for those allocation. Then we can present those callsite when we find out much invalidation. 
 
What else Sheriff can do?
First, Sheriff can work as a run-time system, which can tolerate some of the false sharing problems. 
Second, Sheriff can work as a system which can detect those race condition problems undetectable using those lock-set based tool.
Third, Sheriff can also work as a system to tolerate race problem, like Isolator.

Future work:
(1) Pinpoint the line number to access the cache line by using the "watch point" technique.
(2) Figure out the problem caused by read-write false sharing problem by using the "watch point" technique.
(3) Design a run-time system which can tolerate the false sharing problem in a very low overhead. Profiling 
on specified input should be very helpful to find the problem.
%%%%%%%%%%%%%%%%%%%%%%%%%%%%%%%%%%%%%%%%%%%%%%%%%%%%%%%%%%%%%%%%%%%%%%%%%%%%%%%%%%%%%%%%%%%%%
\end{comment}

\begin{abstract}
%How is existing work?
%Sharing inside mulithreaading programs is not easy, they can easily cause correctness or performance problem. 
%Inappropriate sharing can dramatically degrade the performance of 
%mulithreading programs and seriously affect the scalability. 
%So detecting false sharing accurately and precisely can be helpful for user to fix corresponding performance problem. 


False sharing is a notorious problem for multithreaded applications
that can drastically degrade both performance and
scalability. Existing false sharing detectors can precisely identify
the sources of false sharing, but are limited to reporting false
sharing actually observed during execution: they do not generalize
across executions. Because false sharing is extremely sensitive to
object layout, these detectors can easily miss false sharing problems
that can arise due to slight differences in memory allocation order or
object placement decisions by the compiler. In addition, they cannot
predict the impact of false sharing on hardware with different cache
line sizes.

%objects and cache lines: any change of compiler optimization, compiler, memory manager, 
%memory allocation order, cache line size or different target binary 
%may change alignments, and thus affect occurrences of false sharing, 
%which leaves many of them undetected by existing tools.

This paper presents \Predator{}, a predictive software-based false
sharing detector. \Predator{} generalizes from a single execution to
precisely predict false sharing that is latent in the current
execution. \predator{} tracks accesses within a range that could lead
to false sharing given different object placement. It also tracks
accesses within
\emph{virtual cache lines}, contiguous memory ranges that span actual
hardware cache lines, to predict sharing on hardware platforms with
larger cache line sizes. For each, it reports the exact program
location of predicted false sharing problems, ranked by their
projected impact on performance. We evaluate \Predator{} across a
range of benchmarks and actual applications: \Predator{} identifies
problems undetectable with previous tools, including two
previously-unknown false sharing problems, with no false
positives. \Predator{} located false sharing problems in MySQL and the
Boost library that had eluded detection for years.

%\Predator{} identified two unknown false sharing problems 
%Besides, \Predator{} have successfully detected false sharing of real applications,
%including \texttt{mysql} server application and \texttt{boost} library. Fixing these
%false sharing problems improves performance by $6\times$ and $40\%$ correspondingly.



%False sharing is a notorious performance issue for different software stacks, 
%which can dramatically degrade the performance and seriously affect the scalability of 
%systems.

%Many reserach efforts have been made to detect false sharing. 
%Unfortunately, previous approaches to detect false sharing
%either introduce significant performance overhead, or fail
%to report false sharing accurately and precisely, or have different limitations of usage. 
%\sheriff{}, the prior state-of-the-art tool, 
%can only detect write-write false sharing in applications using \pthreads{} library.
%This paper presents a novel approach, \Predator{}, to combine compiler instrumentation
%and runtime system to detect false sharing. 
%the compiler instruments every memory access and 
%the runtime system collects and analyzes memory accesses to detect false sharing problems.
%Since it does not rely on any hardware, OS or threading library, this approach can be
%applied to the entire software stack without any limitation. 
%\Predator{} can detect false sharing accurately and precisely: it reports no 
%false positives and pinpoints exact objects with false sharing problems.
%Also, unlike previous work, this method can be extended to
%identify false sharing problems across the entire software stack, including 
%hypervisors, operating systems, libraries and applications. 
%Experimental results on two popular benchmark suites 
%show that \Predator{} not only detected all known false sharing problems but also revealed 
%two unknown false sharing problems.
%Besides, \Predator{} have successfully detected false sharing of real applications,
%including \texttt{mysql} server application and \texttt{boost} library. Fixing these
%false sharing problems improves performance by $6\times$ and $40\%$ correspondingly.

%Moreover, existing tools can only detect those manifested false sharing problems.
%However, occurrences of false sharing can be affected by alignments between
%objects and cache lines: any change of compiler optimization, compiler, memory manager, 
%memory allocation order, cache line size or different target binary 
%may change alignments, and thus affect occurrences of false sharing, 
%which leaves many of them undetected by existing tools.
%\Predator{} is the first tool which can accurately predict possible false sharing 
%without the need of occurrences. 
%It can report all false sharing problems with only one execution and with reasonable overhead, 
%around $6.7\times$ performance overhead on average.

%What is novel in our work?
%How is the performance overhead?

\end{abstract}

\terms
Performance, False sharing

\keywords
Concurrency, False Sharing, Performance, Multi-threaded program

%%%%%%%%%%%%%%%%%%%%%%%%%%%%%%%%%%%%%%%%%%%%%%%%%%%%%%%%%%%%%%%%%%%%%%%%%%%%%%%%%%%%%%%%%%%%%
%%%%%%%%%%%%%%%%%%%%%%%%%%%%%%%%%%%%%%%%%%%%%%%%%%%%%%%%%%%%%%%%%%%%%%%%%%%%%%%%%%%%%%%%%%%%%

\section{Introduction}
%False sharing problem is a cache usage problem. 
%Cache, with much faster access speed than main memory, is normally utilized by CPU
%to accelerate program executions by preloading a fixed size of data into the cache each time, 
%called as a cache line.  

%%%%%%%%%%%%%%%%%%%%%%%%%%%%%%%%%%%%%%%%
% Why 
%%%%%%%%%%%%%%%%%%%%%%%%%%%%%%%%%%%%%%%%
\label{sec:intro} 

While writing correct multithreaded programs is often challenging,
making them scale can present even greater obstacles. Any source of
resource contention can impair scalability or even lead to
applications running slower as the number of threads increases.

False sharing is a particularly insidious form of resource contention.
It occurs when two threads inadvertently contend because the objects
they are accessing happen to reside on the same cache line. When at least
one of the threads frequently updates its object, the resulting cache
coherence traffic can degrade performance by up to an order of
magnitude~\cite{falseshareeffect}. Worse, false sharing is generally
difficult to detect: unlike locks or truly shared objects, there is
often no obvious culprit in the source code.

As cache lines have grown larger and multithreaded applications have
become commonplace, false sharing has become an increasingly important
problem. False sharing has been found across the software stack,
including inside the Linux kernel~\cite{OSfalsesharing}, the Java
virtual machine~\cite{JVMfalsesharing}, common
libraries~\cite{libfalsesharing} and several real
applications~\cite{mysql,appfalsesharing}.

Recent efforts to false sharing detection fall short in a number of
dimensions. Some introduce excessive performance overhead, making them
too slow to use in practice~\cite{falseshare:binaryinstrumentation1,falseshare:binaryinstrumentation2,falseshare:simulator}. Most
do not report false sharing precisely and
accurately~\cite{falseshare:binaryinstrumentation1,detect:ptu,detect:intel,falseshare:binaryinstrumentation2,DProf,qinzhaodetection}, require special OS
support~\cite{OSdetection} or impose limits on their use~\cite{sheriff}.
% or they can only detect one kind of
%false sharing problem on a specific multithreading library~\cite{sheriff}.

In addition, all of these systems share one key limitation: they can
only report \emph{observed} cases of false sharing. As Nanavati et
al.\ point out, false sharing is sensitive to where objects are
placed in cache lines and so can be affected by a wide range of
factors~\cite{OSdetection}. For example, using the gcc compiler
\emph{accidentally} eliminates false sharing in the Phoenix
linear\_regression benchmark at certain optimization levels, while
LLVM does not do so at any optimization level.  A slightly different memory
allocation sequence (or different memory allocator) can reveal or hide
false sharing, depending on where objects end up in memory; using a
different hardware platform with different addressing or cache line
sizes can have the same effect. All of this means that existing
tools cannot root out potentially devastating cases of false sharing
that could arise with different inputs, in different execution
environments, and on different hardware platforms.

%\textsc{Sheriff} avoids these limitations, but  
%can only detect write-write type false sharing and 
%and can break correctness of programs using ad-hoc synchronizations or using stack variables to 
%communicate across different threads~\cite{sheriff}. 
%Also, none of these existing approaches  has been verified to find actual 
%false sharing problems in real applications.

%Despite their different features and limitations, all existing detection tools 
%share two common drawbacks.
%Firstly, existing techniques can not be applied for 
%the entire software stack.
% although they might work on a specific level of software stack.

%To be more specific, 
%\Predator{} can report potential false sharing problems with different cache line size and different starting address of an object 
%without the need of another execution. 

This paper presents \Predator{}, a novel software-only false sharing
detector that not only \emph{detects} all existing false sharing
problems accurately and precisely, but also \emph{predicts} potential
false sharing problems that could appear in a  different execution
environment.

\subsection*{Contributions}

This paper makes the following contributions:

\begin{itemize}


% prediction
\item
\textbf{Predictive False Sharing Detection:} \Predator{} is the first system that can \emph{predict} potential false sharing that does
not manifest in an execution but may appear and greatly degrade the
performance of programs in a slightly different
environment. Predictive false sharing generalizes from a single
execution to identify potential false sharing instances that are
within one cache line of each other, which could be exposed by slight
changes in object placement and alignment. It also can predict false sharing
in hardware platforms with larger cache line sizes by tracking
accesses within \emph{virtual cache lines} that span multiple physical
lines. Predictive false sharing detection thus overcomes a key
limitation of previous detection tools.

%Existing approaches is based on specific hardware,
%runtime environment(using specific libraries and compilers) and specific cache line size, which is OK to detect those 
%existing false sharings. But they fail to capture those variables or objects which can greately slow down performance 

%In order to save memory usage, we propose a threshold invoked detection based on the predefined number of writes on a cache line, which
%can be used to track the .

% effect: can detect all kinds of false sharing problems.
\item
\textbf{A Practical and Effective Detection Tool:} \Predator{} detects both observed and predicted false sharing 
problems accurately and precisely with reasonable overhead (average:
$7\times$ performance, $20\%$ memory).  It reveals unknown false
sharing problems in benchmark suites evaluated by previous approaches. It
is also the first false sharing tool able to automatically and precisely uncover
false sharing problems in real applications, including 
MySQL and the Boost library. Fixing these problems 
dramatically improves performance, by $12\times$ for a benchmark and by $6\times$ for \texttt{MySQL}.
%in the benchmarks, and $40\%$ in the actual applications).


\item
\textbf{A Fully General Approach to False Sharing Detection:} \Predator{} provides the first completely general false sharing detection method by
combining compiler-based instrumentation with a runtime system. While this
paper focuses on user-level multithreaded applications, \Predator{}'s
approach is broadly applicable across the entire software stack
because it does not depend on specific hardware, OS, or library
support.
%Since compiler can be utilized to do selective instrumentation, 
%\Predator{} can be utilized to detect all kinds of false sharing problem with ideal overhead. 

% combine compiler instrumentation with runtime system, so that it can 
%detect all kinds of false sharing problem including write-write false sharing. 
%avoid the shortcomings of runtime-only system, where Sheriff can not detect the read-writee false sharings. 
%Also, because of the limit of their implementation, Sheriff can not support the program using the ad hoc synchronization
%or using the stack variables to communicate among different threads.
%Also, it looks like that we can provide a evenly performance overhead.
%Also, sheriff should only work on specific thread library, currently, it can only work on pthreads library. 
%We are trying to extending the same idea to different thread library. Now we can also run DeFault to detect all 
%false sharings using other threads libraries.

\end{itemize}

\subsection*{Outline}

The remainder of this paper is organized as follows. 
Section~\ref{sec:detection} describes \Predator{}'s detection mechanisms and
algorithms in detail.
Section~\ref{sec:prediction} discusses how to predict potential false sharing accurately 
without actually observing false sharing directly. 
Section~\ref{sec:evaluation} presents experimental results, including using \Predator{} to 
reveal unknown false sharing problems in several benchmarks and real applications. 
While \Predator{} has no false positives, it cannot predict \emph{all} possible false sharing; Section~\ref{sec:discussion} discusses these limitations and outlines directions for future work.
Finally, 
Section~\ref{sec:relatedwork} describes key related work and Section~\ref{sec:conclusion} concludes.


%There are two types of false sharings:
%A. Different threads are accessed different locations according to the definition of flase sahrings. 
%B. Locations with a large amount of reads is placed in the cache line with a large amount of writes.
%For the second type, existing tools may tend to miss that.  



\section{Sheriff Overview}
\label{sec:overview}

\begin{figure}[!t]
\begin{center}
\includegraphics[width=3.3in]{figure/overview}
\end{center}
\caption{
Overview of \doubletake{}: execution is divided into epochs at the boundary of irrevocable system calls. 
\label{fig:overview}}
\end{figure}

\doubletake{} is a high performance dynamic analysis framework for a class of errors that share a \emph{monotonicity} property: evidence of the error is persistent and can be gathered after-the-fact. As Figure~\ref{fig:overview} depicts, program execution is divided into epochs, during which execution proceeds at full speed. At the end of each epoch, marked by irrevocable system calls, \doubletake{} checks program state for evidence of memory errors. If an error is found, the epoch is re-executed with additional instrumentation to pinpoint the exact cause of the error. To demonstrate \doubletake{}'s effectiveness, we have implemented detection tools for heap buffer overflows, use-after-free errors, and memory leaks, which we describe in detail in Section~\ref{sec:applications}.  All detection tools share the following core infrastructure that \doubletake{} provides.

\subsection{Efficient Recording}

At the beginning of every epoch, \doubletake{} saves a snapshot of program registers, and all writable memory. The epoch ends when the program attempts to issue an irrevocable system call, but most system calls do not end the current epoch. \doubletake{} also records the order of thread synchronization operations to support re-execution of parallel programs. \doubletake{} records minimal system state at the beginning of each epoch (like file offsets), which allows system calls that modify this state to be undone if re-execution is required. As a result, most programs require very few epochs and program state checks. We describe the details of each application's state checks in Section~\ref{sec:applications}.

\subsection{Precise Replay}

When program state checks detect an error, \doubletake{} replays the previous epoch to pinpoint the error's root cause. \doubletake{} ensures that all program-visible state, system call results, memory allocations, and the order of all thread synchronization operations are identical to the original run. During replay, \doubletake{} returns saved return values for most system calls, with special handling for some cases. Section~\ref{sec:implementation/normalexecution} describes \doubletake{}'s recording and re-execution of system calls and synchronizations.

\subsection{Custom Heap Allocator}

\doubletake{} replaces the default heap allocator with a new heap built using Heap Layers~\cite{heaplayers}. Detection tools can interpose on heap operations to alter memory allocation requests or defer reuse of freed memory, and can access the heap's map of allocated memory. Memory leak detection uses this map to identify unreachable memory. Buffer overflow and dangling pointer (use-after-free) detection both use heap canaries to detect errors. \doubletake{}'s heap includes a bitmap to track the locations of heap canaries, and automatically checks the state of canaries at the end of each epoch. Section~\ref{sec:heapallocator} presents further details.

\subsection{Watchpoints}

\doubletake{} lets detection tools set hardware watchpoints during re-execution. A small number of watchpoints are available on modern architectures (four on x86). Each watchpoint can be configured to pause program execution when a specific byte or word of memory is accessed. Watchpoints are primarily used by debuggers, but previous approaches have used watchpoints for error detection as well~\cite{Kivati,fastboundschecking}. \doubletake{}'s watchpoints are particularly useful in combination with heap canaries. During re-execution, our buffer overflow and use-after-free detectors place a watchpoint at the location of the overwritten canary to trap the instruction(s) responsible for the error.


\section{Simulation of multi-threaded program}
\label{sec:simulation}

This section we are going to talk how to replace threads with processes while still maintain
the same semantics of multithreaded program.

We are going to answer the following questions in the following subsections:
\begin{itemize}
\item How to turn threads into proceses? 
\item How to maintain the semantics to share the same address space?
\item How to share file access?
\item How to handle synchronization among threads?
\end{itemize} 

In the end, we are going to talk how to simuate the threads in one phase.
\subsection{Thread Creation and Exit}
Sheriff intercepts those function calls of \texttt{pthread\_create()} then replaced that with a \texttt{fork()} system call.
Child processes invokes the routine passed by \texttt{pthread\_create()} calls. 
In this way, Sheriff fork off new processes instead. 

Sheriff use one explicit \texttt{\_exit()} call to terminal current thread.

\subsection{Share Address Space}
\label{simulation:sharememory}
In order to create the illusion of multi-threaded programs that 
different threads are sharing the same address space, 
Sheriff uses the memory mapped files to share the heap and globals across different processes.
It is noted that Sheriff don't try to share the stack across different processes 
because different threads have their own stacks and 
it is un-common to use stack to communicate between different threads.

Sheriff creates two different mappings for both the heap and the globals. 
One is shared mapping, which is working as a backstore using to hold those shared state. 
Another one is private, copy-on-write(COW) mapping (per-process) that each process works on directly.
Private working mapping are connecting to the shared mapping through the one memory mapped file.

Read/write operations are worked only on private working mappings. 
But actually reads or writes on one page have different effects that depending on the state of one page.
For reads on those pages without private copies, 
reads are actually accessing those pages in the shared mapping dirtectly with the help of memory mapped file.
For reads on those pages with private copies, 
reads can only access those private copies which are guranteed by the separation mechanism of process. 
Writes have one different senario. Private working mapping are setted to read-only mode in the beginning. 
First write on one read-only page can invoke a COW operation that can copy the content from 
corresponding shared mapping. 
After than, user program can only access those private mapping. 
It is the duty of Sheriff to commit those changes
of different processes to the shared mapping in the transaction end (more deatails can be seen in
section~\ref{simulation:thread}.

The details to create the memory mappings for globals and heap are described in the following.
\par\vspace{3mm}
\noindent
\textbf{Globals}
\par\vspace{3mm}
\noindent
Sheriff uses a fixed-size(larger than normal globals size) file to hold globals. Sheriff checks the size of actual
globals to guarantee that the specified size is enough to hold all globals, if not, Sheriff can report this anomaly and 
user can fix that easily.
Sheriff tries to get the start and size of globals for one program and uses \texttt{mmap()} to create a shared mapping 
among different threads. That is, user program are still using the address after the compilation.
Since some global variables may has some initialized value, Sheriff copies all contents 
from the private mapping to the shared mapping. 

\par\vspace{3mm}
\noindent
\textbf{Heap}
\par\vspace{3mm}
\noindent
Sheriff also uses a fixed-size mapping to hold the heap for user application, currently we are using 1.6GB. Memory allocations 
requirements from user applications are met from this fixed-size private mapping.

Since different threads can get the memory from this fixed-size mapping, the heap data structure are shared among different
threads and allocations are protected by one process-based mutex.
In order to avoid the false sharing problems brought by the memory allocator, 
Sheriff builds a scalable ``per-thread'' heap organization that is loosely based on Hoard~\cite{BergerMcKinleyBlumofeWilson:ASPLOS2000} and built using HeapLayers~\cite{BergerZornMcKinley:2001}. 
Sheriff divides the heap into a fixed number of sub-heaps(currently 16). Each thread uses a hash of its process id to obtain
the index of the heap which can be used to satisfy memory allocations. 
This origranization has two benefits. First, it can minimize the conflicts caused by using only 
one shared heap between different threads, thus avoiding the performance impact by using the lock to 
maintain the data consistency.
Second, it avoids the false sharing problems caused by heap objects. Since each thread are using different pages 
to satisfy memory allocations, objects allocated by one thread has no chance to be the same 
cache line with objects from another thread. Thus, runtime system build on that can avoid false sharing problems 
and have a much better scalability. 

\begin{comment}
Not all memory allocations are meet from the protected heap to avoid the performance problem. 
Since those accesses on protected heap, the first write on each transaction should pay the overhead to 
create the copies and later should do the comparison and commit in the end of transaction.
Those overhead is additional comparing to normal running of multi-threaded program. 
We believe that false sharing problem has a greater chance to happen in those small allocations (smaller than cores number 
times of cache line size). Huge block of memory normally is not a source of false sharing problem. 
Besides those mapping to work as a shared backstore and a working copy, Sheriff provides an additional MAP\_SHARED mapping 
which is used to meet the allocation requirements of larger objects.
\end{comment}

\subsection{Share File Access}
\label{sec:fileshare}
\begin{figure}[!t]
\small
\begin{lstlisting}[frame=trbl]{}
int spawnWithShareFiles{
  return syscall(SYS_clone, 
   CLONE_FS|CLONE_FILES|SIGCHLD,(void*)0);
}
\end{lstlisting}
\caption{Pseudo-code to create a new process.
\label{fig:newfork}}
\end{figure}
In multi-threaded programs, different threads in the same process share opened files of each thread. 
But multi-processes is a different senario: 
each process is assumed to own all of its resources, including memory, file handles, sockets, device handles, and windows. 

Since Sheriff tries to approximate the semantics of multi-threaded programs.
it is necessary to make different processes to share opened files.
One naiive way is to create a process to work as the proxy for other processes, this proxy can \texttt{open/close} files 
or change attributes of files for other processes. 
After that, the proxy can use a ``Unix domain socket'' to pass the descriptor from the proxy to the initiating process 
and other processes wanting to share this file ~\cite{passfd}. 
But it is not easy to do this since we have to wrap up a lot of file system related system calls. 
Also, there is some additonal overhead of communication.

Sheriff uses a different approach, by setting the flag \texttt{CLONE\_FILES} 
when creating new processes (Fig.~\ref{fig:newfork}), 
child processes can share the same physical file descriptor table with the parent process. 
Then Sheriff can share files among different processes.
\begin{comment}
It is noted that using \texttt{clone()} to create new process will affect the getpid() system call.
Although current thread model is the 1-to-1 model in pthreads library (not n-to-1 model any more), 
glibc's wrapper will always return the process ID of the parent process via the glibc wrapper function. 
Sheriff avoid this predicament by intercept getpid() calls then just give the stored pid for different threads.
\end{comment}

\subsection{Synchronization}
\label{simulation:syn}

\begin{figure}[!t]
\small
\begin{lstlisting}[frame=trbl]{}
  void sync(var) {
    closeTransaction();

	realVar = getRealVariable(var);
	pthread_sync(realVar);
	
    startTransaction();
  }
\end{lstlisting}
\caption{Pseudo-code for all synchronization operations.\label{fig:synccode}}
\end{figure}
Synchronization are used to coordinate the activities and data accesses among different threads. 
Synchronization also means that some shared data needs to be changed or be used across different threads. 
For example, program calls \texttt{mutex\_lock()} when it needs to access some shared data. 
There are some \textbf{synchronization} mechanisms in multithreaded program, 
including: \textit{mutex}, \textit{conditional variable}, \textit{barrier} and \textit{join}.

In order to describe things easier, we introduce the ``transaction'' concept here. 
``transaction'' is used to describe one piece of code which executes in a atomical,  
separated and consistent way. 
It is noted that Sheriff is not one traditional transactional memory system.
Transaction Memory is a concurrency control mechanism 
attempting to simplify the parallel programming by allowing a group of instructions 
to execute in an atomic way.
Traditional transaction memory systems are optimized for short transactions,
but do not effectively support long-lived transactions. They also provides a rollback mechanism
when one transaction fails to commit. 
Sheriff won't support rollback here and supports any length of transaction, 
longer transaction is better to amortize the overhead,
and tends to use for general multi-threaded program, 
no need to annotate the code as ``transaction'' manually.
Also, Sheriff don't replace the lock usage as some log-based mechanism.
But transaction concept in Sheriff has the atomicity, isolation and consistency attributes, which is 
the same as that in transactional memory~\cite{transaction}. 

\begin{comment}
Sheriff read-protects all pages of memory in the beginning of memory so that any intends to write on one 
page can be captured by Sheriff by handling the page fault. 
In the end of transaction, Sheriff tries to publish the modifications
made by current transaction so that other threads in the same application can see the modifications by one thread.
\end{comment}

In order to simulate those multithreaded synchronization, Sheriff intercepts those synchronization object 
initialization function calls, allocates one new synchronization object on a shared mapping (shared by all processes)
and initializes them to be accessed by different processes. Then the new object's address can be saved 
in the header of original object. 
Handling about one synchronization call can be seen in Fig.~\ref{fig:synccode}. 

In order to explain it more clear, we are using the mutex as an example. There are three regions for one mutex usage: 
the first region is before \texttt{mutex\_lock}; the second region is between \texttt{mutex\_lock} and \texttt{mutex\_unlock}; 
 the third region is after \texttt{mutex\_unlock}. Sheriff chooses to use three different transactions for these three different regions.
Although it is unnecessary to do so for the first and third region since they don't need 
to executed in a atomical and isolated way. But it is better to do so for easiness.

For the first and third region, introduction of transaction here won't cause correctness problem 
according to Sheriff's assumption(Section~\ref{overview:assumption}). Other threads are not supposed 
to access the data modified by current process.

For the second region, when there is \texttt{mutex\_lock} and \texttt{mutex\_unlock} call, 
Sheriff are trying to call corresponding
pthreads library's functions but worked on a process-based mutex object. 
According to the semantics of multithreaded program, the modifications happening between \texttt{mutex\_lock} 
and \texttt{mutex\_unlock}
are unseen by other threads and modifications can work on shared memory directly (it is also safe to so).  
In actual implementation, it is relatively expensive to 
change page mapping between \texttt{MAP\_SHARED} and \texttt{MAP\_PRIVATE} 
especially when the memory footprint is very large. To avoid those unnecessary overhead, 
Sheriff starts a new transaction after \texttt{mutex\_lock()} and closes that transaction in case of \texttt{mutex\_unlock()}. 

For conditional variable and barrier, they are using the same mechanism as the example in Fig.~\ref{fig:synccode}.
\texttt{pthread\_join} is a little different. Sheriff just closes current transaction and calls \texttt{waitpid()} 
when there is a \texttt{pthread\_join()} call. 

\subsection{Thread Execution}
\label{simulation:thread}
As what we described above, Sheriff uses the transaction(atomic execution) to simulate 
those synchronization mechanism of multithreaded program. 
The overview of this mechanism can also be seen in Fig.~\ref{fig:overview}.
 
Before the program begins, Sheriff establishes shared and local mappings for the heap and globals. 
\begin{comment}
To improve the performance, Sheriff don't do page protection when there is just one thread; read/write operations
work on shared mappings directly to avoid protection overhead and commit overhead. 
We implement this in the following two cases:
First, Sheriff will start page protection until there is a pthread\_create() function call  
to spawn one child. 
Second, Sheriff will close page protection when there is just one thread.
Sheriff always checks whether current thread is the only thread in the system in pthread\_join() function
and close the page protection timely if it is.

To start the protection, those pages in the protection range will be set to MAP\_PRIVATE and PROT\_READ mode; 
later access on one protected page should invoke a Copy-On-Write operation in the operating system.
To stop the protection, those pages in the protection range will be re-set to MAP\_SHARED and readable/writable; later access 
access on those pages will work on shared mapping directly(through mapping file). 
\end{comment}
\subsubsection{Transaction Begin}
In the beginning of one transaction, Sheriff sets every page in the protection range to \texttt{PROT\_READ} so that
later writes on those pages can be caught by handling \texttt{SEGV} protection faults.
In fact, Sheriff don't need to set on every page, only those pages dirtied in last transaction needs to be
set to \texttt{PROT\_READ}; other clean pages should still in the \texttt{PROT\_READ} 
since the mode bit of those pages are kept the same
in the execution if one page is not modified.

Sheriff clears dirty pages sets after it set every dirty page to \texttt{PROT\_READ}.

\subsubsection{Execution}
\label{simulation:execution}
When no page under protection is written, Sheriff runs almost the same speed as that of multithreaded program. 
When those pages under protection are written, that triggers a page fault and Sheriff can
be involved in by handling \texttt{SEGV} protection faults.
 
The algorithm of page fault handler is listed in the following:
\begin{enumerate}
\item 
Sheriff tries to get the page holding the faulted address and then set this page to write-able so that 
future accesses on this page can run at a full speed (won't invoke page fault any more). 
Thus, one page incurs only one page fault in one transaction. 
Although protection faults and signal faults are expensive, those cost 
can be amoritized for the whole transaction.

\item 
Before the creation of ``twin page'', Sheriff force a Copy-On-Write operation on this page by writing to the start of this page 
with the content getting from the same address. 
This step is very important to get two identical pages for ``twin'' page and working copy 
so that comparison of those two can give actual modifications made by this transaction. 
Since there is a time gap between the creation of ``twin'' pages and that of ``private'' pages, private pages are created 
by OS's COW after the signal handler. 
After the force of a COW, Sheriff creates a copy of current page from share area to a local store(called as ``twin'' page). 
This ``twin'' page mechanism has been discussed in Section~\ref{overview-twinpage}.
\item 
In the end of page fault handler, Sheriff adds page address to the \textit{dirty} set so that 
all dirty pages can be checked in the end of transaction.
\end{enumerate}

\subsubsection{Transaction End}
\label{simulation:endtran}
In the end of each atomically-executed region - the end of each thread, right before and end of those synchronization points, 
right before a thread spawn, and right before joining another thread - 
Sheriff commits those changes from ``private'' pages to ``shared'' mapping, 
then remove those old private pages and twin pages.

Then Sheriff will trying to commit those modifications in current transaction to ``shared'' mapping so that other threads can
see those changes modified by current transaction. 
As what we desribed in Section~\ref{overview-twinpage}, those ``twin'' pages and ``working'' (private) pages
can be compared word by word in order to capture those modifications in current transaction. 
Those new values on ``working'' pages will be committed to the same offsets of corresponding ``shared'' mapping.

After commits, Sheriff issues a madvise (\texttt{MADV\_DONTNEED}) call to discard current physical pages of ``private'' mapping 
. Since Sheriff allocates some physical pages to 
hold those ``twin'' mapping, those pages should be discarded too to avoid the memory leakages. 
Note here that Sheriff will hold a global lock in order to do commits and updations automically. 
\begin{comment}
\begin{figure}[!t]
\small 
\begin{lstlisting}[frame=trbl]{}
//@This function is called by pthread_join
void uniqueChecking (void) {
  // Is it the initial thread?
  if(getpid() == initialPid) {
	// Using waitpid to check the uniqueness.
    if(waitpid(-1, NULL, WNOHANG) == -1 
	   && errno == ECHILD) {
		// Close protetion here.
        closeMemoryProtection();
        _protected = false;
    }
  }
}
\end{lstlisting}
\caption{Pseudo-code for unique checking.\label{fig:unique}}
\end{figure}

\end{comment}


\section{Detection of False Sharing}
\label{sec:falseshare}
From above section, we already know that how to design a runtime system to simulate the running 
of multi-threaded program. 

In this section, we are going to talk how to indentify false sharing problems by recording memory writing using
the runtime system.
This section are trying to answer the following questions:

\begin{itemize}
\item How to capture the memory writes from different process?
\item How to capture the continuous memory writes?
% - sampling mechanism - "temporary twin" and "original twin". 
\item How to capture the interleaving cache invalidation? 
% Use a global array, updating timely when modification is detected. 
\item How to identify objects inside one cache line? 
%Attach the callsite information to capture the allocate sites for heap objects. 
\item How to differentiate true sharing and false sharing?
%detect the combination? An array to get word version and threads working on that. We can detect those fields inside one object causing the problem too.
\item How to report one false sharing problem?
% In the end of program, we traverse the whole global array.
\end{itemize}

\subsection{Capture of Memory Writes}
\label{falseshare:memorywrites}
Process can provide a strong isolation of one thread's running from other threads' running. 
In each transaction, Sheriff runs one thread in a atomical, consistent and isolated way and
won't commit those changes in one transaction until the end of one transaction.
In the end of each transaction, Sheriff can compare ``twin'' page and ``working'' page word by word to find 
those modifications on each dirty page. When the word of ``working'' page is different from that of 
corresponding ``twin'' page, this word is thought to be modified by current thread in current transaction. 
It is reasonable to reach this conclusion since Sheriff can guarantee that originally the content of ``twin'' page 
is the same as that of ``working'' page by forcing a COW explicitely (see ~\ref{simulation:execution}).
\begin{comment}
It is true that the writing of ``A-B-A''  can be missed by simply comparison,
but we believe that ``A-B-A'' writing in one transaction
is not frequent and won't bring any correctness problem.
We don't want to put too much focus on this point.
\end{comment}

Since we can capture the memory writes on every transaction and one thread's running is consisted of multiple transactions, 
we can capture the memory writes from different threads.

\subsection{Capture of Continuous Writes}
\label{detection:sampling}
We already know from above section that Sheriff can capture memory writes on one transaction. 
But it is not good enough when the transaction length is too long (some extreme case can be the whole thread). 
Actually, one serious false sharing problem (\texttt{linear\_regression} benchmark, see Section~\ref{sec:evaluation}) 
which affect the performance 10X can be omitted since there is only one transaction for one thread, 
without any synchronization inside. 

Sheriff use a sampling mechanism to avoid this problem. Sampling is 
to select some of observations in order to acquire some knowledge about the whole.
Although sampling cannot give complete information about memory writes on one transaction,
sampling can be used to capture more writes. More fine sampling can help to find more writes by one transaction.
There is a balance between choosing finer sampling period and performance issue here. 
Sheriff now choose 10 microseconds as a basic interval to do sampling. 

In order to capture continuous writes, Sheriff introduce one ``temporary twin'' page for every shared dirty page
 (see Fig.\ref{fig:overview}). 
Handling of those ``temporary twin'' pages are slightly diferent with those ``original twin'' pages.
First, they are created in the sampling timer handler when one page is found to be shared by multiple threads. 
There is no use to create ``temporary twin'' for those pages only accessed by one thread. We are using a global array to
record users for one page. 
Second, ``temporary twin'' pages are keeping updated to ``working'' version in every timer handler 
in order to capture future writes on the same page.
 
\subsection{Capture of Cache Invalidations}
\label{detection:invalidation}
Just as we talked in Section~\ref{overview:target}, only numerous interleaving writes can bring performance problem. 
Sheriff tries to capture the interleaving writes across different threads in order to capture 
cache invalidations. 

In order to capture interleaving writes on caches, Sheriff introduces 
virtual cache line status words (Fig.~\ref{fig:overview}). 
``virtual'' is used here to differentiate with ``physical'' cache line. 
For every virtual address range (same size with physical cache line) under protection, Sheriff assigns one status word. 
One status word has two fields, the first field points last thread to write on this cache line, 
the second field is used to record times of invalidates (version) on one cache line. 
Every time when one different thread are detected to write on this cache line, Sheriff update both the thread id
to be the new thread and version number. 
In actual implementation, Sheriff introduce two different arrays to avoid using lock. Corresponding code can be seen
in Figure~\ref{fig:capturecacheinvalidation}.
CacheInvalidation array is used to capture those interleaving cache invalidation for all cache lines in protected memory. 
Every cache line have one corresponding counter to indicate the interleaving of cache invalidation for this cache line. 
LastThreadModifyCache array is used to record last thread id to write on its cache line. 

The pseudo code to capture the interleaving cache invalidation is listed in Figure~\ref{fig:capturecacheinvalidation}:
\begin{figure*}[!t]
\begin{lstlisting}
void recordCacheInvalidates(int cacheNo) {
    int myTid = getpid();
    int lastTid;

    // Try to check last thread to modify this cache.
    lastTid = atomic_exchange(&LastThreadModifyCache[cacheNo], myTid);
    if(lastTid != myTid) {
       // Increment cache invalidation only when current thread is different.
       atomic_increment(&cacheInvalidation[cacheNo]);
     }
}
\end{lstlisting}
\caption{Record the cache invalidation atomically.\label{fig:capturecacheinvalidation}}
\end{figure*}

\subsection{Indentify Objects inside Cache Line}
\label{detection:object}
For global object, Sheriff don't need to do anything since debug information can provide
object's information.
Sheriff attaches the call site in the header of each heap object when allocation to indentify objects.
Callsite information can provide objects' request allocation, which is useful for programmer
to fix the false sharing problem (see case study in Section~\ref{evaluation:comparison}).
It is one important feature to differentiate Sheriff from previous tools.
Previous tools using binary instrumentation or hardware performance counter cannot control
the memory allocation, so they cannot provide the callsite information about one object.
Sheriff is a runtime system which intercepts all heap allocations so that Sheriff can tell programmer
about cache line's object information.

Remember that Sheriff have two arrays to capture the cache interleaving invalidation
(see Section~\ref{detection:invalidation}), it is necessary to cleanup those invalid counting
when one object is de-allocated. It is important to avoid the false positives caused by uncorrectly aggregate 
counting when one address is re-used for other objects. 

\subsection{Avoidance of False Positives}
\label{detection:avoidfalsepositive}
To avoid false positives, 
Sheriff introduces another global array to record  
threads writing on each word and version numbers of each word.
Threads writing on each word can tell whether one cache line is false sharing or true sharing. 
Version number on each word can avoid to report those objects which don't contribute much on cache invalidations, 
when there are multiple objects in the same cache line.
In order to save space, Sheriff use one word's higher 16 bit to store the thread id on one word
and use the lower 16 bit to store version number of this word. 
When one word is detected to be modified by more than two threads, we marked specially
on its thread id field.

\subsection{Reporting False Sharing Objects}
In the end of program, Sheriff reports those objects causing false sharing problems. 
Since Sheriff introduce a global array (CacheInvalidationArray) to record those 
cache invalidation (see Section~\ref{detection:invalidation}), Sheriff checks 
CacheInvalidationArray for cache lines with invalidation times larger than one water level. 
After one cache line is found, corresponding invalidation times and offset of this cache line 
will be added into a global link linst sorted by invalidation times. 
Later we can rank the false sharing objects by invalidation times they caused. 

After the traverse of all cache lines, Sheriff tries to get objects information 
for all cache lines in the link list. 
Sheriff uses magic value added in the allocation to differentiate the start of one object. 
Also, the object size information can help to identify the start of one object. 
After finding out those objects inside one cache line, Sheriff should look into the 
array listed in Section~\ref{detection:avoidfalsepositive} to avoid false positives. 

%%%%%%%%%%%%%%%%%%%%%%%%%%%%%%%%%%%%%%%%%%%%%%%%%%%%%%%%%%%%%%%%%%%%%%%%%%%%%
%%%%%%%%%%%%%%%% Where to specify those procedure of timer handler????? LTP
%%%%%%%%%%%%%%%%%%%%%%%%%%%%%%%%%%%%%%%%%%%%%%%%%%%%%%%%%%%%%%%%%%%%%%%


\section{Discussion}
\label{sec:discussion}

\subsection{Limitation on Prediction}
As discussed in Section~\ref{sec:intro}, many dynamic properties
can affect occurrences of false sharing. 
\Predator{} can predict potential false sharing caused by cache line size 
or an object's starting address change for the same binary.
However, it can not predict false sharing in the case of  
memory layout change made by different compiler. 
For example,  two global variables $A$ and $B$, which do not fit into 
the same cache line when compiled using $llvm$, may be placed close to each
other by another compiler, and thus lead to false sharing. 
\Predator {} does not consider such kind of false sharing, because
it is impossible to predict the behavior of different compilers.

\subsection{Application to Kernel and Hypervisor}
\begin{comment}
Although \Predator{} conceptually can be used to detect false sharing in different levels of 
software stack, currently it can not be applied to those levels directly. 
Some components, like customized memory manager, can only work in user level. Also, \Predator{}
has to identify the source of accesses using some system specfic calls. For example, it uses
gettid() to identify accesses from different threads.   
Extending \Predator{} to different levels of software stack will be the future work for us.
\end{comment}

\subsection{Performance Improvement}
\Predator{} runs around $6.6$X slower on average for all evaluated benchmarks. 
In our current implementation, all memory accesses are instrumented with a library call to 
invoke the runtime system for tracking. 
This long jump from user code to library call entails large performance overhead
since it needs to go through Global Offset Table (GOT) and Procedure Linkage Table (PLT).
However, certain logic implemented in library are as simple as reading or
incrementing counters. Two identifed places are shown below:

\begin{itemize}
\item
Upon an access to a cache line, 
\Predator{} simply incrementes this line's write counter for write and
ignore read until the counter value reaches {\it Tracking-Threshold}. 
\item
In sampling mechanism discussed in Section~\ref{sec:sample}, 
most accesses (99\%) only needs to increment access counters of cache lines.
\end{itemize}

Implementing such logic only needs a few low latency instructions instead of expensive
library calls. In the future, \Predator{} should 
selectively expand library calls so that for simple logic, instructions are inserted 
directly into user code. This will certainly improve the performance.



\section{Evaluation}
\label{sec:evaluation}

All evaluations are performed on a quiescent Intel Core 2 dual-processor system equipped with 
16GB RAM. 
Each processor is a 4-core 64-bit Intel Xeon running at 2.33 GHz with a 4MB
shared L2 cache and 32KB private L1 data cache. 
The underlying operating system is unmodified CentOS 5.5, running with Linux kernel
version 2.6.18-194.17.1.el5. The glibc version is 2.5. 
In order to compare the performance fairly, all benchmarks were built as 64-bit executables 
using LLVM compiler (version 3.2). The compiler optimization level is set to ``-O1'' 
so memory allocation callsites can be reported precisely.
%since we can not report line number of source code with optimization level larger than
%``-O2''.
In our evaluations of this section, 
we choose two popular benchmark suites, Phoenix~\cite{phoenix-hpca} 
and PARSEC~\cite{parsec}. 
Some programs of PARSEC cannot be compiled or run succesfully with LLVM are excluded here.
For example, \texttt{facesim} can't be compiled 
because \texttt{llvm} forces a much stricter C++ rule. 
\texttt{canneal} can't be run successfully even it is compiled by original \texttt{llvm} compiler.

In this section, our evaluations aim to answer the following questions:
\begin{itemize}
\item
  How effective is \Predator{} on detecting and predicting false sharing problem (Section ~\ref{sec:effective})?

\item
  What is the performance overhead of \Predator{} with and without prediction
  (Section ~\ref{sec:perfoverhead})?

\item
  What is the memory overhead of \Predator{}~ (Section~\ref{sec:memoverhead})?
\end{itemize}


\subsection{Detection and Prediction Effectiveness}
\label{sec:effective}

\subsubsection{Benchmarks}
\label{sec:benchmarks}
Results on two benchmark suites, Phoenix and PARSEC, 
are listed in the Table~\ref{table:detection}. 

%Our results show that \Predator{} not only capture previously-discovered
%false sharing, but also many detect new false sharing places. The results
%are listed in Table~\ref{table:detection}. 

%http://www.technovelty.org/tips/getting-a-tick-in-latex.html
%http://tex.stackexchange.com/questions/42619/x-mark-to-match-checkmark
%\begin{comment}
\begin{table*}[ht!]
{
\centering
\begin{tabular}{l|r|r|r}
\hline
{\bf \small Benchmark} & {\bf \small Source Code} & {\bf \small New} & {\bf \small Improvement} \\
%{\bf \small Benchmark} & {\bf \small Source Code} & {\bf \small Type of False Sharing} & {New} & {\bf \small Improvement} \\
\hline
\small \textbf{histogram} & {\small histogram-pthread.c:213} & \cmark{} & 46.22\%\\
\small \textbf{reverse\_index} & {\small reverseindex-pthread.c:511} & \xmark{} & 0.09\%\\
\small \textbf{word\_count} & {\small word\_count-pthread.c:136} & \xmark{} & 0.14\%\\
\hline
\small \textbf{streamcluster} & {\small streamcluster.cpp:985} & \xmark{} & 7.52\% \\
\small \textbf{streamcluster} & {\small streamcluster.cpp:1907} & \cmark{} & 4.77\%\\
%\small \textbf{bodytrack} & {\small TrackingModel.cpp:59} & 0 & \cmark{} & \\
\hline
\hline
\small \textbf{linear\_regression} & {\small linear\_regression-pthread.c:133} & \xmark{} & 1206.93\%\\
\hline
\end{tabular}
\caption{Detection results of \Predator{} on Phoenix and PARSEC benchmark suites. \label{table:detection}}
}
\end{table*}

In this table, the first column lists those programs with false sharing problems. 
The second column shows precisely where the problem is. Since all found false sharing 
occurs inside the same heap object, we listed callsite source code information in this column.
The third column ``New'' marks whether this false sharing is newly found by \Predator{} or not.
False sharing problems found by previous work are marked with a crossmark(\xmark{}) and those 
newly found ones are marked with a tick mark(\cmark{}). 
The last column ``Improvement'' shows the performance improvement after fixing false sharing. 
The number is based on the average runtime of $10$ runs. 

\begin{comment}
\textbf{Tongping: how to say the following sentences}\\
The improvement rate is calculated by substracting $1$ from normalized runtime of orignal
runtime and new runtime. 
Taking \texttt{histogram} for an example here, original runtime of \texttt{histogram} is $0.75s$
and new runtime is $0.51s$, then the performance improvement is $(0.75/0.51) - 1$.
\end{comment}

As shown in the table, \Predator{} reveals several unknown false sharing problems. 
It is the first tool to detect two false sharing problems: one in \texttt{histogram} 
and one in line $1908$ of \texttt{streamcluster}. 
In \texttt{histogram}, multiple threads repeatedly modify different locations of the same heap object. 
Padding the data structure \texttt{thread\_arg\_t} fixes the false sharing problem and 
helps to improve the performance around 46\%.
In \texttt{streamcluster}, multiple threads are simultaneously accessing and updating 
the same \texttt{bool} array, \texttt{switch\_membership}. 
By simply changing this array to \texttt{long} type contributes to about 4.7\% performance improvement.

%, although it is not a complete fix of false sharing. 
%None of these two false sharing problems has been reported by previous tools.
All other false sharing problems have been discovered by one of previous tools, 
\sheriff{}~\cite{sheriff}. 
Same as \sheriff{}, we do not see much performance improvement for \texttt{reverse\_index} and 
\texttt{word\_count} since the number of updates inside them is not significantly large. But they
are actual false sharing problems that have been verified manually by us.

\texttt{streamcluster} has another false sharing problem at line $985$. 
Different threads change the same object (\texttt{work\_mem}) simultaneously. 
Authors of \texttt{streamclsuter} have already realized possible
false shairing problems and meant to utilize a macro \texttt{CACHE\_LINE} to avoid it. Unfortunately,
the defaulted value of this macro is setted to $32$ bytes, which is different with the actual
cache line size of our hardware. By setting to $64$ bytes instead, we achieve around $7.5\%$ performance
improvement.

\texttt{linear\_regression} has a severe false sharing problem, 
while fixing it improves the performance more than $12\times$.
In this benchmark, different threads constantly update their thread specific locations 
inside a heap object(\texttt{tid\_args}), causing a huge amount of cache invalidations. 
As pointed out by Nanavati et al.~\cite{OSdetection}, false sharing 
occurs when using \texttt{clang} compiler and disappears when using \texttt{GCC} with optimization
-O2 and -O3.  
During our evaluation, because our customized memory manager has different allocation 
metadata for each heap object, this false sharing do not 
occur at all when we are compiling this program
using \texttt{clang-3.2} at ``-O1'' optmization level.
Detailed reason of this can be seen in Section~\ref{sec:predicteval}.
Thus, \Predator{} actually can not detect this false sharing problem without enabling 
prediction mechanism. Also, none of existing tools can detect false sharing without occurence.
Using the prediction mechanism discussed in Section~\ref{sec:prediction}, 
\Predator{} can always capture false sharing problems in this benchmark.
This also examplify the importance of prediction tool. 

\subsubsection{Real Applications}
To verify its practicability, we further evaluate \Predator{} 
on several widely-used real applications, which none of previous work has evaluated.  
These real applications include a server application \texttt{MySQL}~\cite{mysql}, 
a common C++ library \texttt{boost}~\cite{libfalsesharing} 
and a distributed memory object caching system \texttt{memcached}, a network retriver \texttt{aget}, 
a parallel bzip2 file compressor \texttt{pbzip2} and a parallel file scanner \texttt{pfscan}.
For \texttt{MySQL} and \texttt{boost},
we evaluate their specific versions, \texttt{MySQL-5.5.32} and
\texttt{boost-1.49.0}, which are known to have some false sharing problems.

The false sharing problem in \texttt{MySQL} has caused significant scalability problem and
it was very difficult to be identified. 
According to the architect of \texttt{MySQL} Mikael Ronstrom, ``we had gathered specialists on 
InnoDB..., participants from MySQL support... and a number of generic specialists on 
computer performance...'', ``the fruit of the meeting ... were able to 
improve \texttt{MySQL} performance by 6$\times$ with those scalability fixes''. 
The false sharing of boost library is caused by the special usage of \texttt{spinlock} pool and fixing
it brings 40\% performance improvement. 
\Predator{} is able to succesfully detect false sharing locations
in both \texttt{MySQL} and \texttt{boost} library. 
For the other four applications, \Predator{} doest not find serere false sharing problems.

\subsubsection{Prediction Effectiveness}
\label{sec:predicteval}
For prediction effectiveness, we evaluate whether \Predator{} can always capture a false sharing
problem without occurrence.
\texttt{linear\_regression} benchmark is selected here because of the following two reasons:
\begin{enumerate}
\item
False sharing problem of this benchmark can not be detected without prediction, see Section~\ref{sec:benchmarks}. 

\item
False sharing severely degrades performance when it actually occurs. 
Hence, it is a serious problem that should be always detected. 
\end{enumerate}

\begin{figure}[!h]
{\centering
\subfigure{\lstinputlisting[numbers=none,frame=none,boxpos=t]{fig/linearregression.psedocode}}
\caption{False sharing problem inside \texttt{linear\_regression} benchmark.
\label{fig:linearregression}}
}
\end{figure}

Figure~\ref{fig:linearregression} shows the data structure and source code
experiencing false sharing.
The size of this data structure, \texttt{lreg\_args}, is $64$ bytes 
when we use \texttt{clang} compiler to compile $64$bit binary at ``-O1'' optimization level.
For this false sharing problem, the main thread allocates an array with the number of elements equals
to that  of underlying hardware cores.
Each element is a \texttt{lreg\_args} type with $64$ bytes. 
Then this array is passed to different threads (\texttt{lreg\_thread}) 
so that each thread only updates its thread-dependent area, see Figure~\ref{fig:linearregression}.
False sharing occurs if two threads happens to update data in the same cache line. 
However, different fields of \texttt{lreg\_args} has different access pattern:
only those fields between $SX$ and $SXY$ (totally around $40$ bytes) are constantly read and updated.
Consequently, the performance of \texttt{linear\_regression} is very sensitive to 
the starting address of false sharing object (see Figure~\ref{fig:perfsensitive}),
which can be changed by many dynamic properties according
to the discussion in Section~\ref{sec:intro}.

Figure~\ref{fig:perfsensitive} shows performance sensitivity to 
offsets of the starting address between the false sharing object and corresponding cache lines. 
When the offset is $0$ or $56$ bytes, this benchmark achieves its optimal performance 
and has no false sharing at all.
When the offset is $24$ bytes, this benchmarks runs around $15$ times slower 
than its optimal performance.
When we eveluate detection effectiveness on its original code, 
our customized memory manager happens to make the offset $56$ bytes. 
As a result, \Predator{} can not detect false sharing in this benchmark 
without enabling prediction because of no occurrence of false sharing problem.
This situation happens to all existing tools: they can only detect false sharing problems when
they occur. 

In contrast, the prediction mechanism designed in \predator{} 
amis to address this problem.  Results of our evaluations shows 
that \Predator{} can always predict the false sharing problem in this
benchmark no matter what the offset value is. 
This explains the effectiveness of \Predator{}.

\subsection{Performance Overhead}
\label{sec:perfoverhead}

\begin{figure*}[!ht]
\begin{center}
\includegraphics[width=6.5in]{fig/perf}
\end{center}
\caption{
Performance overhead of \Predator{} with and without prediction.
\label{fig:perf}}
\end{figure*}

To avoid the effect caused by extreme outliers, all performance data shown in this section
are based on the average result of $10$ runs while excluding the maximum and minimum values.
Actual performance overhead with and without prediction 
can be seen in the following figure~\ref{fig:perf}. 

From this figure, we can see for all $16$ benchmarks from Phoenix and PARSEC
benchmark suites that \Predator{} with prediction imposes around $6.7\times$
performance overhead. 
If we remove prediction from \Predator{}, we cannot observe significant performance difference.
This means that prediction of \Predator{} only introduces very minimum performance overhead. 

Among these programs, five of them have more than $8\times$ performance overhead, 
including \texttt{histogram, kmeans, bodytrack, ferret} and \texttt{swaptions}. 
Program \texttt{histogram} has the most performance overhead and 
runs more than $26$ slower than original executions. 
It has a severe false sharing problem inside, and tracking detailed access for those
problematic cache lines exacerbates the 
false sharing effect (see more discussion on this in Section~\ref{sec:sample}). 
For \texttt{bodytrack} and \texttt{ferret}, \Predator{} found a large amount of cache lines with 
writes larger than {\it Tracking-Threshold}. 
Tracking all accesses details for those cache lines 
imposes significant performance overhead. 
Currently, we have not identified the reasons 
why \texttt{kmeans} runs much slower in \Predator{}.   

In our evaluation, we do not observe signficant performance overhead on 
\texttt{matrix\_multiply, blackscholes} and 
\texttt{x264}.
Possibly a large portion of computations operates on stack variables, which are
not tracked by \Predator{}. 

\subsection{Memory Overhead}
\label{sec:memoverhead}
Since \Predator{} pre-allocates a huge block of memory ({\it virtual memory}) 
using \texttt{mmap} system call for its heap usage, 
virtual memory can not be used to evaluate actual memory overhead imposed by our tool. 
Hence, we only evaluate physical memory used for each application. 
According to the discussion of Justin et al. ~\cite{memusage}, proportional set size (PSS) 
in \texttt{/proc/self/smaps} is a suitable number since it reflects memory increase to the system
by running this application. 

To get PSS data, we start a script program to save 
corresponding \texttt{smaps} files periodically.
For each \texttt{smaps} file, we calculate the sum of PSSs for different
memory mappings and uses it as total physical memory usage for this application.
Among all collected \texttt{smaps} files, we choose the maximum number of
different files for comparison since it represents the maximum memory overhead to run this application.
%It is noted that we remove the physical memory usage of   
Results of maximum memory usage is shown in Figure~\ref{fig:memusage}. As we can see,
\Predator{} does not introduce substantial memory usage overhead 
for all eveluated bencharmks, except for \texttt{swaptions}. 
Removing \texttt{swaptions} from comparisons reduces 
the average memory overhead from 64\% to 22\%. 

The reason why \texttt{swaptions} introduces $7.8\times$ memory overhead is that 
its original memory usage is too small (only $3KB$).
Adding the code of detection, prediction and
reporting contributes to a large portion of memory overhead. 

\begin{figure*}
\begin{center} 
\includegraphics[width=6.5in]{fig/memusage}
\end{center}
%\includegraphics{fig/potential.pdf}
\caption{Memory usage overhead}
\label{fig:memusage}
\end{figure*}




\section{Related Work}
\label{sec:relatedwork}

The proliferation of multicore systems has increased interest in tool
support to detect false sharing, since standard profilers like
OProfile~\cite{oprofile} or gprof~\cite{gprof} only report overall
cache misses.

\paragraph{Simulation and Instrumentation Approaches:} Schindewolf describes a system based on the SIMICS functional
simulator that reports full cache miss information, including
invalidations caused by
sharing~\cite{falseshare:simulator}. Pluto
uses the Valgrind framework to track the sequence of load and store
events on different threads and reports a worst-case estimate of
possible false sharing~\cite{falseshare:binaryinstrumentation1}. Similarly,
Liu uses Pin to collect
memory access information and then reports total false sharing miss
information~\cite{falseshare:binaryinstrumentation2}.

Independently and in parallel with this work, Zhao et al.\ developed a
tool designed to detect false sharing and other sources of cache
contention in multithreaded applications~\cite{zhao:vee:2011}. This
tool uses shadow memory to track ownership of cache lines and cache
access patterns. It is currently limited to at most 8 simultaneous
threads. Like Liu above, it only reports an overall false sharing rate
for the whole program, placing the burden on programmers to examine
the entire source base to locate any instances of false
sharing.

Unlike \sheriffdetect{}, these systems generally suffer from high
performance overhead ($5-200\times$ slower) or memory overheads.
They cannot differentiate true sharing
from false sharing, yielding numerous false positives. Because they
operate at the binary level, they can all be misled by aliasing due to
memory object reuse. Finally, and most importantly, they do not point
to the objects responsible for false sharing, limiting their value to
the programmer.

\paragraph{Sampling-Based Approaches:}
Intel's performance tuning utility (PTU)~\cite{detect:ptu,
detect:intel} uses event-based sampling, allowing it to operate
efficiently. PTU can discover shared physical cache lines, and can
identify possible false sharing at the grain of individual function
calls. PTU suffers from a high false positive rate caused by aliasing
due to reuse of heap objects, and reports false sharing instances that
have no impact on performance. PTU cannot differentiate true from
false sharing or pinpoint the source of false sharing problems,
unlike \sheriffdetect{}. Section~\ref{sec:evaluation} contains an
extensive empirical comparison of PTU to \sheriffdetect{}
demonstrating PTU's relative shortcomings.

Pesterev et al.\ describe DProf, a tool that leverages AMD's
instruction-based sampling hardware to help programmers identify the
sources of cache misses~\cite{DProf}. DProf requires manual annotation
to locate data types and object fields, and cannot detect false
sharing when multiple objects reside on the same cache line. By
contrast, \sheriffdetect{} is architecture independent, requires no
manual intervention, and precisely identifies false sharing regardless
of the flow of data or which data types are involved.

\paragraph{False Sharing Avoidance: }
In some restricted cases, it is possible to eliminate false sharing,
obviating the need for detection. Jeremiassen and Eggers describe a
compiler transformation that adjusts the memory layout of applications
through padding and alignment~\cite{falseshare:compile}.  Chow et al.\
describe an approach that alters parallel loop scheduling to avoid
sharing~\cite{falseshare:schedule}. The effectiveness of these
static analysis based approaches is primarily limited to regular,
array-based scientific codes, while \sheriffprotect{} can prevent false
sharing in any application.

Berger et al.\ describe Hoard, a scalable memory allocator can reduce
the likelihood of false sharing of distinct heap
objects~\cite{BergerMcKinleyBlumofeWilson:ASPLOS2000}. Hoard limits
accidental false sharing of entire heap objects by making it unlikely
that two threads will use the same cache lines to satisfy memory
requests, but this has no effect on false sharing within individual heap
objects, which \sheriffprotect{} avoids.

\paragraph{Other Related Work:}
\sheriff{} borrows the
process-as-thread model pioneered by Grace~\cite{grace}, but otherwise
differs from it in its semantics, generality, and goals.

Grace is a process-based approach designed to prevent concurrency
errors, such as deadlock, race conditions, and atomicity errors by
imposing a sequential semantics on speculatively-executed
threads. Grace supports only fork-join programs without inter-thread
communication (e.g., condition variables or barriers), and rolls back
threads when their effects would violate sequential semantics.

\sheriff{} extends Grace to handle
arbitrary multithreaded programs; for example, Grace is incompatible
with any applications that employ inter-thread communication,
including the PARSEC benchmarks we examine here. \sheriff{} applies
diffs at synchronization points in the program to enable updates
without rollback, giving it far greater performance (but different
semantics) than Grace. \sheriff{} does not eliminate concurrency
errors, but instead allows applications to selectively track updates
and isolate memory, enabling tools like \sheriffdetect{}
and \sheriffprotect{}.


\section{Future Work}
\label{sec:futurework}

\subsection{Approach Applicability}
\Predator{} is an approach that does not rely on the support of specific hardware, OS and libraries.
Hence, it is applicable to detect and predict false sharing in the entire software stack in theory, including hypervisors, operating systems, libraries or applications using different threading libraries. It is the future work for us to detect false sharing problems in other level's software.


\subsection{Performance Improvement}
\Predator{} runs around $6\times$ slower on average for all evaluated applications. In the current implementation, every memory access is instrumented with a library call to notify the runtime system. A library call entails not only normal function call overhead but also Global Offset Table(GOT) and/or Procedure Linkage Table (PLT) look-up overhead. 
It is too heavy for simple computations in the following:

\begin{itemize}
\item
Before the number of writes on a cache line reaches {\it Tracking-Threshold},  \Predator{} simply increments its write counter.

\item
In the sampling mechanism discussed in Section~\ref{sec:sample}, most accesses (99\%) outside sampling period only needs to increment access counters of cache lines.
\end{itemize}

These computations listed above only need simple checking or updating operations, normally taking only a few CPU cycles to finish. 
In the future, we plan to instrument those simple computations directly, instead of using library calls.

\subsection{Valuable Suggestions}
\Predator{} can be extended to provide suggestions for fixing false sharing problems based on the memory trace information, which can help reduce the manual overhead of fixing false sharing problems.  

\label{sec:future}
\begin{comment}
(1) Pinpoint the line number to access the cache line by using the "watch point" technique. \\
(2) Figure out the problem caused by read-write false sharing problem by using the "watch point" technique. \\
(3) Design a run-time system which can tolerate the false sharing problem in a very low overhead. Profiling
on specified input should be very helpful to find the problem.
\end{comment}

\section{Conclusion}
\label{sec:conclusion}
This paper presents Sheriff, one software-only system to precisely detect false sharing problems
in multithreaded programs. 
Sheriff's key contribution is its runtime system framework, which simulating the running 
of threads using process. 
Sheriff uses the process's complete separation mechanism and page fault mechanism to capture writes
in one transaction by combining with twin page mechanism. 
In order to capture continuous writes in the same transaction, 
Sheriff introduce the sampling mechanism for those long-running
transactions. 
Unlike previous tools, Sheriff intercepts those memory allocation and de-allocation function calls so that
those false positives caused by re-use of dynamic objects can be avoided completely in Sheriff and also
Sheriff can attach callsite on those memory allocations, which can help to report those callsite 
information about one false sharing objects and save manual effort to locate the false sharing problems.

Sheriff can use preload attributes of library, which can work on unaltered binary files and makes the deployment
simple. Since Sheriff don't use the advanced hardware support, which can be used to find false sharing problems
for those legacy applications running on original hardware. 
For most of applications, sheriff only incurs reasonable runtime overhead, making it a 
practical choice even for those complex system which should run long time. 

{
% \footnotesize
\small
\bibliographystyle{abbrv}
\bibliography{ref,emery}
}

\end{document}
